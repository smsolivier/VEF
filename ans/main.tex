\documentclass{anstrans}
%%%%%%%%%%%%%%%%%%%%%%%%%%%%%%%%%%%
\title{Eddington Acceleration}
\author{Samuel S. Olivier*}

\institute{Department of Nuclear Engineering, Texas A\&M University, College Station, TX 77843}

\email{smsolivier@tamu.edu}


%%%% packages and definitions (optional)
\usepackage{graphicx} % allows inclusion of graphics
\usepackage{booktabs} % nice rules (thick lines) for tables
\usepackage{microtype} % improves typography for PDF

\usepackage{xspace}

\newcommand{\SN}{S$_N$\xspace}
\renewcommand{\vec}[1]{\bm{#1}} %vector is bold italic
\newcommand{\vd}{\bm{\cdot}} % slightly bold vector dot
\newcommand{\grad}{\vec{\nabla}} % gradient
\newcommand{\ud}{\mathop{}\!\mathrm{d}} % upright derivative symbol
\newcommand{\pderiv}[2]{\frac{\partial #1}{\partial #2}}
\newcommand{\dderiv}[2]{\frac{\ud #1}{\ud #2}}
\newcommand{\edd}{\langle \mu^2 \rangle} 

% spatial discretization of SN equations? 

\begin{document}
\section{Introduction}
	One of the most challenging computational tasks is to simulate the interaction of radiation with matter. The steady--state, one--group, isotropically--scattering, fixed--source Linear Boltzmann Equation in planar geometry is: 
		\begin{equation} \label{eq:bte}
			\mu \pderiv{\psi}{x}(x, \mu) + \Sigma_t(x) \psi(x,\mu) = 
			\frac{\Sigma_s(x)}{2} \int_{-1}^{1} \psi(x, \mu') d\mu' + \frac{Q(x)}{2}
		\end{equation}
	where $\mu = \cos\theta$ is the cosine of the angle of flight $\theta$ relative to the $x$--axis, $\Sigma_t(x)$ and $\Sigma_s(x)$ the total and scattering cross sections, $Q(x)$ the isotropic fixed--source and $\psi(x, \mu)$ the angular flux \cite{adams}. In the Discrete--Ordinates (\SN) angular discretization, $\mu$ takes values from Gauss Legendre quadrature. The scalar flux, $\phi$, is then 
		\begin{equation} \label{eq:quad}
			\phi(x) = \sum_{n=1}^N w_n \psi_n(x)
		\end{equation}
	where $\psi_n(x) = \psi(x,\mu_n)$ and $w_n$ the quadrature weights \cite{llnl}. The \SN equations are then 
		\begin{equation} \label{eq:sn}
			\mu_n \dderiv{\psi_n}{x}(x) + \Sigma_t(x) \psi_n(x) = 
			\frac{\Sigma_s(x)}{2} \sum_{n=1}^N w_n \psi_n(x) + \frac{Q(x)}{2}. 
		\end{equation}

	In the Source Iteration (SI) scheme, the right hand side of Eq. \ref{eq:sn} is lagged. In other words, 
		\begin{equation} \label{eq:si}
			\mu_n \dderiv{\psi_n^{\ell+1}}{x}(x) + \Sigma_t(x) \psi_n^{\ell+1}(x) = 
			\frac{\Sigma_s(x)}{2} \sum_{n=1}^N w_n \psi_n^\ell(x) + \frac{Q(x)}{2}. 
		\end{equation}
	Equation \ref{eq:si} is iterated until the flux converges. If $\phi^0(x) = 0$ then, $\phi^\ell$ is the scalar flux of particles that have undergone $\ell - 1$ collisions \cite{adams}. Thus, the number of iterations until convergence is directly linked to the number of collisions in a particle's lifetime. Typically, SI becomes increasingly slow to converge as the ratio of $\Sigma_s$ to $\Sigma_t$ approaches unity. 

	Fortunately, the regime where SI is slow to converge is also the regime where Diffusion Theory is more accurate. A popular method of accelerating SI is Diffusion Synthetic Acceleration (DSA) where a transport sweep is conducted and then a diffusion solve is used to generate a correction factor. Standard DSA requires correction schemes such as the Source Correction, Diffusion Coefficient, and Removal Correction schemes presented in \cite{alcouffe} to prevent instability in highly scattering regimes with coarse spatial grids. 

	Lawrence Livermore National Laboratory (LLNL) is developing a high--order radiation--hydrodynamics code. The hydrodynamics portion is discretized using the Mixed--Hybrid Finite Element Method (MHFEM), where values are taken to be constant within a cell with discontinuous jumps at the cell edges. MHFEM is particularly suited for hydrodynamics but not for radiation transport. This work seeks to efficiently accelerate \SN calculations with a scheme that is both robust and compatible with MHFEM hydrodynamics. 

\section{Eddington Acceleration}
	The zeroth and first angular moments of Eq. \ref{eq:bte} are 
		\begin{subequations} 
		\begin{equation} \label{eq:zero}
			\dderiv{}{x} J(x) + \Sigma_a(x) \phi(x) = Q(x) 
		\end{equation} 
		\begin{equation} \label{eq:first}
			\frac{\ud}{\ud x} \edd(x) \phi(x) + \Sigma_t J = 0  
		\end{equation}
		\end{subequations}
	where $J = \int_{-1}^{1} \mu \ \psi(x, \mu) \ud \mu$ is the current and 
		\begin{equation} \label{eq:eddington} 
			\edd(x) = \frac{\int_{-1}^1 \mu^2 \psi(x, \mu) \ud \mu}{\int_{-1}^1 \psi(x, \mu) \ud \mu}
		\end{equation}
	the Eddington factor. When $\edd = \frac{1}{3}$, Eqs. \ref{eq:zero} and \ref{eq:first} are equivalent to Diffusion Theory. 

	The proposed acceleration scheme is: 
		\begin{enumerate}
			\item Compute $\psi_n(x)$ with \SN with an arbitrary spatial discretization
			\item Compute $\edd(x)$ 
			\item Interpolate $\edd(x)$ onto the MHFEM grid 
			\item Solve the moment equations with the preconditioned $\edd(x)$ using MHFEM. 
		\end{enumerate}
	This scheme allows the \SN equations and moment equations to be solved with different spatial discretizations. This means \SN can be discretized using normal methods such as Linear Discontinuous Galerkin or Diamond Differencing while the moment equations can be solved on the same grid as the hydrodynamics. 

	This method differs from DSA in that two solutions are generated: one from \SN and one from the moment equations. The solution of the moment equations will be used because the moment equations are conservative while \SN is not. 
\section{Results}

\section{Acknowledgments}

\bibliographystyle{ans}
\bibliography{bibliography.bib}
\end{document}