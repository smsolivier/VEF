\documentclass[10pt]{beamer}

\usetheme[progressbar=foot, sectionpage=none]{metropolis}
% font options
\usefonttheme{professionalfonts}
\usepackage{appendixnumberbeamer}

\usepackage{booktabs}
\usepackage[scale=2]{ccicons}

\usepackage{pgfplots}
\usepgfplotslibrary{dateplot}

\usepackage{lmodern}

\usepackage{cancel}

\usepackage{color}

\usepackage{xspace}
\usepackage{siunitx}
\newcommand{\themename}{\textbf{\textsc{metropolis}}\xspace}

\usepackage[usestackEOL]{stackengine}

\title{\color{berkeleyblue} Variable Eddington Factor Method for Inconsistently Differenced Source Iteration}
\subtitle{\normalsize International Conference on Transport Theory \\ Novel Numerical Methods}
% \date{\today}
\date{October 19, 2017}
\author{Samuel S. Olivier$^1$, Jim E. Morel$^2$}
\institute{$^1$Department of Nuclear Engineering, University of California, Berkeley \\$^2$Department of Nuclear Engineering, Texas A\&M University \\ \\ 
\scriptsize
% \url{https://github.com/smsolivier/EddingtonAcceleration.git} \\ 
\vfill
% \centerline{\includegraphics[height=.75cm]{nuen-logo.png}}
% \titlegraphic{\hfill\includegraphics[height=1.5cm]{nuen-logo.png}}
}

\newcommand{\SN}{S$_N$\xspace}
\renewcommand{\vec}[1]{\bm{#1}} %vector is bold italic
\newcommand{\vd}{\bm{\cdot}} % slightly bold vector dot
\newcommand{\ud}{\mathop{}\!\mathrm{d}} % upright derivative symbol
\newcommand{\pderiv}[2]{\frac{\partial #1}{\partial #2}}
\newcommand{\dderiv}[2]{\frac{\ud #1}{\ud #2}}
\newcommand{\edd}{\langle \mu^2 \rangle} 
\newcommand{\rell}{^\ell} % raise to ellth power 
\newcommand{\relll}{^{\ell+1}} % raise to ell + 1 th power 
\newcommand{\rellh}{^{\ell+1/2}} % raise to ell + 1/2 power
\newcommand{\bracket}[1]{\left[ #1 \right]}

\newcommand{\paren}[1]{\left(#1\right)} 
\newcommand{\br}[1]{\left[#1\right]}
\newcommand{\curl}[1]{\left\{#1\right\}}

\newcommand{\eddphi}[1]{\edd_{#1}\phi_{#1}}
\newcommand{\ALPHA}[2]{\frac{#1}{\sigma_{t,#2} h_{#2}}}

% make blocks fill 
\metroset{block=fill}

% dark background 
% \metroset{background=dark}

% color options
\definecolor{maroon}{RGB}{80,0,0}
\definecolor{berkeleyblue}{HTML}{003262}
\definecolor{calgold}{HTML}{FDB515}
\setbeamercolor{progress bar}{fg=calgold, bg=berkeleyblue!30}
\setbeamercolor{progress bar in head/foot}{fg=calgold, bg=berkeleyblue!30}
\setbeamercolor{progress bar in section page}{fg=calgold, bg=berkeleyblue!30}
\setbeamercolor{palette primary}{bg=berkeleyblue}

\setbeamercolor{alerted text}{fg=calgold}

\setbeamertemplate{frametitle continuation}{}

\setbeamertemplate{frame numbering}[fraction]


% ---------------------------------------
\begin{document}

\maketitle

\begin{frame}[plain,noframenumbering]{Overview}
  \setbeamertemplate{section in toc}[sections numbered]
  \tableofcontents[hideallsubsections]
\end{frame}

\section{Background}

\begin{frame}{Variable Eddington Factor Method/Quasi Diffusion Background}

	\begin{itemize}
	
		\item One of the first nonlinear methods for accelerating source iterations 

		\item Produces 2 solutions: one from \SN and one from a transport informed diffusion equation (drift diffusion)
		\begin{itemize}
			\item Do not necessarily become identical when the iterative process converges 
			\item Solutions do converge as the mesh is refined $\Rightarrow$ built in truncation estimator 
		\end{itemize}


	\end{itemize}

\end{frame}

\begin{frame}{Why Nonlinear Acceleration?}

	\begin{itemize}

		\item Classic discretizations (step, diamond) are not suitable for radiative transfer in High Energy Density Physics regime $\Rightarrow$ Discontinuous Galerkin (DG) 

		\item Linear acceleration of Discontinuous Finite Element \SN is problematic 
		\begin{itemize}
			\item Transport and diffusion must be consistently differenced to prevent instability 

			\item Requires the diffusion equation to be expressed in $P_1$ form which is more difficult to solve (Warsa, Wareing, Morel, NSE 2002) 

			\item Partially consistent linear acceleration methods are generally difficult to develop (Wang and Ragusa, NSE 2010)

		\end{itemize}

	\end{itemize}

\end{frame}

\begin{frame}{Why Nonlinear Acceleration? (Cont)}

	\begin{itemize}

		\item Nonlinear acceleration has relaxed consistency requirements 
		\begin{itemize}
			\item Drift diffusion acceleration equation can be discretized in any valid manner without regard for consistency with \SN  

			\item Preserves the thick diffusion limit regardless of discretization consistency 
		\end{itemize}

		\item Can use VEF drift diffusion in multiphysics iterations 
		\begin{itemize}

			\item VEF drift diffusion is conservative and inexpensive (compared to an \SN sweep) 

			\item Couple drift diffusion to other physics components 

			\item Can use discretization compatible with other physics while still retaining benefits of DG \SN 

		\end{itemize}

	\end{itemize}

\end{frame}

\begin{frame}{Motivation}

	\begin{itemize}

		\item Mixed Finite Element Method (MFEM) is being used for high order hydrodynamics calculations (Dobrev, Kolev, Rieben, SIAM 2012)

		\item MFEM is not appropriate for standard, first-order form of transport equation 

		\item $\Rightarrow$ VEF method with DG \SN discretization + MFEM drift diffusion discretization 

	\end{itemize}

	\begin{alertblock}{Goals}
		
		Show Lumped Linear Discontinuous Galerkin (LLDG) \SN can be paired with MFEM drift diffusion for one group, 1D neutron transport 

	\end{alertblock}

\end{frame}

\section{Description of VEF Method}

\begin{frame}{\SN Equations}

	Planar geometry, fixed-source, 1-D, one group, neutron transport equation 
	\begin{equation*} 
		\mu \pderiv{\psi}{x} \paren{x, \mu} + \sigma_t(x) \psi(x,\mu) = 
			\frac{\sigma_s(x)}{2} \int_{-1}^1 \psi(x,\mu') \ud \mu' + \frac{Q(x)}{2}
	\end{equation*}

	\pause
	\SN angular discretization 
	\begin{equation*} \label{eq:sn}
		\mu_n \dderiv{\psi_n}{x}(x) + \sigma_t(x) \psi_n(x) = 
		\frac{\sigma_s(x)}{2} \phi(x) + \frac{Q(x)}{2} \,, \quad 1 \leq n \leq N
	\end{equation*}

	where 
	\begin{equation*}
		\phi(x) = \sum_{n=1}^N w_n \psi_n(x) \,, \psi_n(x) = \psi(x, \mu_n)
	\end{equation*}

\end{frame}

\begin{frame}{Source Iteration}

	Lag scattering term 
	\begin{equation*} \label{eq:si}
		\mu_n \dderiv{}{x}\psi_n\rellh(x) + \sigma_t(x) \psi_n\rellh(x) = 
		\frac{\sigma_s(x)}{2} \phi^\ell(x) + \frac{Q(x)}{2} \,, \quad 1 \leq n \leq N 
	\end{equation*}

	\pause
	Source Iteration 
	\begin{equation*}
		\phi^{\ell+1} = \phi\rellh
	\end{equation*}

	\pause
	Slow to converge in optically thick and highly scattering systems 

\end{frame}

\begin{frame}{VEF Drift Diffusion}

	Instead, solve 
	\begin{equation*} \label{eq:drift}
	-\dderiv{}{x} \frac{1}{\sigma_t(x)} \dderiv{}{x} \bracket{\edd\rellh(x)\phi\relll(x)} + \sigma_a(x) \phi\relll(x) = Q(x) \,,
	\end{equation*}
	for $\phi\relll(x)$ using transport information from iteration $\ell+1/2$

	\pause 
	Variable Eddington Factor:
	\begin{equation*} \label{eq:eddington} 
		\edd\rellh(x) = \frac{\int_{-1}^1 \mu^2 \psi\rellh(x, \mu) \ud \mu}{\int_{-1}^1 \psi\rellh(x, \mu) \ud \mu}
	\end{equation*}

	Angular flux weighted average of $\mu^2$ 

	Depends on angular shape of the angular flux, not its magnitude 

\end{frame}

\section{Discretization}

\section{Scattering Update Methods}

\section{Computational Results}

\section{Conclusions}


% begin uncounted slides ---------------------------
\appendix

\begin{frame}[allowframebreaks]{References}

	\nocite{*}
	\setbeamerfont{bibliography item}{size=\scriptsize}
	\setbeamerfont{bibliography entry author}{size=\scriptsize}
	\setbeamerfont{bibliography entry title}{size=\scriptsize}
	\setbeamerfont{bibliography entry location}{size=\scriptsize}
	\setbeamerfont{bibliography entry note}{size=\scriptsize}
	\setbeamertemplate{bibliography item}{\insertbiblabel}
	\bibliographystyle{siam}
	\bibliography{references}

\end{frame}

\begin{frame}[standout]
  Questions?
\end{frame}

\end{document}
