\documentclass[10pt]{beamer}

\usetheme[progressbar=foot]{metropolis}
% font options
\usefonttheme{professionalfonts}
\usepackage{appendixnumberbeamer}

\usepackage{booktabs}
\usepackage[scale=2]{ccicons}

\usepackage{pgfplots}
\usepgfplotslibrary{dateplot}

\usepackage{lmodern}

\usepackage{cancel}

\usepackage{color}

\usepackage{subcaption}

\usepackage{xstring}

\usepackage{xspace}
\usepackage{siunitx}
\newcommand{\themename}{\textbf{\textsc{metropolis}}\xspace}

\usepackage[usestackEOL]{stackengine}

\title{\color{berkeleyblue} Variable Eddington Factor Method for Inconsistently Differenced Source Iteration}
\subtitle{\normalsize International Conference on Transport Theory \\ Novel Numerical Methods}
% \date{\today}
\date{October 19, 2017}
\author{Samuel S. Olivier$^1$, Jim E. Morel$^2$}
\institute{$^1$Department of Nuclear Engineering, University of California, Berkeley \\$^2$Department of Nuclear Engineering, Texas A\&M University \\ \\ 
\scriptsize
% \url{https://github.com/smsolivier/EddingtonAcceleration.git} \\ 
\vfill
% \centerline{\includegraphics[height=.75cm]{nuen-logo.png}}
% \titlegraphic{\hfill\includegraphics[height=1.5cm]{nuen-logo.png}}
}

\newcommand{\SN}{S$_N$\xspace}
\renewcommand{\vec}[1]{\bm{#1}} %vector is bold italic
\newcommand{\vd}{\bm{\cdot}} % slightly bold vector dot
\newcommand{\ud}{\mathop{}\!\mathrm{d}} % upright derivative symbol
\newcommand{\pderiv}[2]{\frac{\partial #1}{\partial #2}}
\newcommand{\dderiv}[2]{\frac{\ud #1}{\ud #2}}
\newcommand{\edd}{\langle \mu^2 \rangle} 
\newcommand{\rell}{^\ell} % raise to ellth power 
\newcommand{\relll}{^{\ell+1}} % raise to ell + 1 th power 
\newcommand{\rellh}{^{\ell+1/2}} % raise to ell + 1/2 power
\newcommand{\bracket}[1]{\left[ #1 \right]}

\newcommand{\paren}[1]{\left(#1\right)} 
\newcommand{\br}[1]{\left[#1\right]}
\newcommand{\curl}[1]{\left\{#1\right\}}

\newcommand{\eddphi}[1]{\edd_{#1}\phi_{#1}}
\newcommand{\ALPHA}[2]{\frac{#1}{\sigma_{t,#2} h_{#2}}}

% make blocks fill 
\metroset{block=fill}

% dark background 
% \metroset{background=dark}

% color options
\definecolor{maroon}{RGB}{80,0,0}
\definecolor{berkeleyblue}{HTML}{003262}
\definecolor{calgold}{HTML}{FDB515}
\definecolor{founder}{HTML}{3B7EA1}
\definecolor{medalist}{HTML}{C4820E}
\definecolor{goldengate}{HTML}{ED4E33}
\setbeamercolor{progress bar}{fg=founder, bg=berkeleyblue!30}
\setbeamercolor{progress bar in head/foot}{fg=founder, bg=berkeleyblue!30}
\setbeamercolor{progress bar in section page}{fg=founder, bg=berkeleyblue!30}
\setbeamercolor{palette primary}{bg=berkeleyblue}

\setbeamercolor{alerted text}{fg=founder}

\setbeamertemplate{frametitle continuation}{}

\setbeamertemplate{frame numbering}[fraction]

\makeatletter
% \setlength{\metropolis@titleseparator@linewidth}{2pt}
\setlength{\metropolis@progressonsectionpage@linewidth}{1.5pt}
\setlength{\metropolis@progressinheadfoot@linewidth}{1.5pt}
\makeatother

\setbeamertemplate{block begin}
{
  \par\vskip\medskipamount%
  \IfStrEq{\insertblocktitle}{}{}{
      \begin{beamercolorbox}[colsep*=.75ex]{block title}
        \usebeamerfont*{block title}\insertblocktitle%
      \end{beamercolorbox}%
  }
  {\parskip0pt\par}%
  \ifbeamercolorempty[bg]{block title}
  {}
  {\ifbeamercolorempty[bg]{block body}{}{\nointerlineskip\vskip-0.5pt}}%
  \usebeamerfont{block body}%
  \begin{beamercolorbox}[colsep*=.75ex,vmode]{block body}%
    \ifbeamercolorempty[bg]{block body}{\vskip-.25ex}{\vskip-.75ex}\vbox{}%
}


% ---------------------------------------
\begin{document}

\maketitle

\begin{frame}[plain,noframenumbering]{Overview}
  \setbeamertemplate{section in toc}[sections numbered]
  \tableofcontents[hideallsubsections]
\end{frame}

\section{Background}

\begin{frame}{Variable Eddington Factor Method Background}

	% \begin{itemize}
	
		% \item 
		One of the first nonlinear methods for accelerating source iterations

		% \item 
		Use \SN to create a transport informed drift diffusion solution  

		% \item 
		Produces 2 solutions: one from \SN and one from drift diffusion 
		\begin{itemize}
			\item Do not necessarily become identical when the iterative process converges 
			\item Solutions do converge as the mesh is refined $\Rightarrow$ built in truncation estimator 
		\end{itemize}

		Will show that the benefits outweigh producing 2 solutions 


	% \end{itemize}

\end{frame}

\begin{frame}{Why Nonlinear Acceleration?}

	% \begin{itemize}

		% \item 
		Classic discretizations (step, diamond) are not suitable for radiative transfer in High Energy Density Physics regime $\Rightarrow$ Discontinuous Galerkin (DG) 

		% \item 
		Linear acceleration of Discontinuous Finite Element \SN is problematic 
		\begin{itemize}
			\item Consistent differencing required 

			\item Requires the diffusion equation to be expressed in $P_1$ form which is more difficult to solve (Warsa, Wareing, Morel NSE 2002) 

			\item Partially consistent linear acceleration methods are generally difficult to develop (Wang and Ragusa NSE 2010)

		\end{itemize}

	% \end{itemize}

\end{frame}

\begin{frame}{Why Nonlinear Acceleration? (cont.)}

	% \begin{itemize}

		% \item 
		Nonlinear acceleration has relaxed consistency requirements 
		\begin{itemize}
			\item Drift diffusion acceleration equation can be discretized in any valid manner without regard for consistency with \SN  

			\item Preserves the thick diffusion limit regardless of discretization consistency 
		\end{itemize}

		% \item 
		Can use VEF drift diffusion in multiphysics iterations 
		\begin{itemize}

			\item VEF drift diffusion is conservative and inexpensive (compared to an \SN sweep) 

			\item Couple drift diffusion to other physics components 

			\item Can use discretization compatible with other physics while still retaining benefits of DG \SN 

		\end{itemize}

	% \end{itemize}

\end{frame}

\begin{frame}{Motivation}

	% \begin{itemize}

		% \item 
		Mixed Finite Element Method (MFEM) is being used for high order hydrodynamics calculations (Dobrev, Kolev, Rieben SIAM 2012)

		% \item 
		MFEM is not appropriate for standard, first-order form of transport equation 

		% \item 
		$\Rightarrow$ VEF method with DG \SN discretization + MFEM drift diffusion discretization 

	% \end{itemize}

	\vfill
	\begin{alertblock}{Goals}
		
		Show Lumped Linear Discontinuous Galerkin (LLDG) \SN can be paired with MFEM drift diffusion for one group, 1D neutron transport 

	\end{alertblock}

\end{frame}

\section{Description of VEF Method}

\begin{frame}{\SN Equations}

	Planar geometry, fixed-source, 1-D, one group, neutron transport equation 
	\begin{equation*} 
		\mu \pderiv{\psi}{x} \paren{x, \mu} + \sigma_t(x) \psi(x,\mu) = 
			\frac{\sigma_s(x)}{2} \int_{-1}^1 \psi(x,\mu') \ud \mu' + \frac{Q(x)}{2}
	\end{equation*}

	\pause
	\SN angular discretization 
	\begin{equation*} \label{eq:sn}
		\mu_n \dderiv{\psi_n}{x}(x) + \sigma_t(x) \psi_n(x) = 
		\frac{\sigma_s(x)}{2} \phi(x) + \frac{Q(x)}{2} \,, \quad 1 \leq n \leq N
	\end{equation*}

	where 
	\begin{equation*}
		\phi(x) = \sum_{n=1}^N w_n \psi_n(x) \,, \psi_n(x) = \psi(x, \mu_n)
	\end{equation*}

\end{frame}

\begin{frame}{Source Iteration}

	Lag scattering term 
	\begin{equation*} \label{eq:si}
		\mu_n \dderiv{}{x}\psi_n\rellh(x) + \sigma_t(x) \psi_n\rellh(x) = 
		\frac{\sigma_s(x)}{2} \phi^\ell(x) + \frac{Q(x)}{2} \,, \quad 1 \leq n \leq N 
	\end{equation*}

	\pause
	Source Iteration 
	\begin{equation*}
		\phi^{\ell+1} = \phi\rellh
	\end{equation*}

	\pause
	Slow to converge in optically thick and highly scattering systems 

\end{frame}

\begin{frame}{VEF Drift Diffusion}

	Instead, solve 
	\begin{equation*} \label{eq:drift}
	-\dderiv{}{x} \frac{1}{\sigma_t(x)} \dderiv{}{x} \bracket{\edd\rellh(x)\phi\relll(x)} + \sigma_a(x) \phi\relll(x) = Q(x) \,,
	\end{equation*}
	for $\phi\relll(x)$ using transport information from iteration $\ell+1/2$

	\pause 
	Transport information passed through the \alert{Variable Eddington Factor:}
	\begin{equation*} \label{eq:eddington} 
		\edd\rellh(x) = \frac{\int_{-1}^1 \mu^2 \psi\rellh(x, \mu) \ud \mu}{\int_{-1}^1 \psi\rellh(x, \mu) \ud \mu}
	\end{equation*}

	\begin{itemize}
		\pause
		\item Angular flux weighted average of $\mu^2$ 

		\pause
		\item Depends on angular shape of the angular flux, not its magnitude 
	\end{itemize}

	\pause
	Use $\phi\relll$ to update scattering term in \SN sweep or as final solution if converged 

\end{frame}

\begin{frame}{Acceleration Properties}

	Angular shape of the angular flux, and thus the Eddington factor, converges much faster than the scalar flux 

	Drift diffusion includes scattering 

\end{frame}

\begin{frame}{VEF Algorithm}

	\begin{figure}

		\only<1>{\def\svgwidth{.8\textwidth}\input{figs/vef_flow.pdf_tex}}%
		\only<2>{\def\svgwidth{.8\textwidth}\input{figs/vef_discretized.pdf_tex}}

	\end{figure}

\end{frame}

\section{Discretizations}

\begin{frame}{Lumped Linear Discontinuous Galerkin \SN}

	\begin{columns}

		\begin{column}{.5\textwidth}
		\vspace{.1in}
		\begin{itemize}

			\item 
			2 discontinuous, linear basis functions 

			\item 
			Cell edges uniquely defined through upwinding 

			% \item Cell centers through polynomial interpolation (in linear case it is just the average of $\psi_{i,L}$ and $\psi_{i,R}$) 

		\end{itemize}
		\end{column}
		\begin{column}{.5\textwidth}

			\begin{figure}

				\def\svgwidth{\textwidth}
				\input{figs/lldg.pdf_tex}
				% \caption{Unknowns and upwinding for an LLDG cell. }
				
			\end{figure}

		\end{column}

	\end{columns} 

	\begin{columns}
	\begin{column}{1.06\textwidth}
	\begin{itemize}

		\item 
		Within the cell, $\psi$ is a linear combination of the basis functions:
		\begin{equation*}
			\psi_{n,i}(x) = \psi_{n,i,L} B_{i,L}(x) + \psi_{n,i,R} B_{i,R}(x) \,, \quad x \in (x_{i-1/2},x_{i+1/2})
		\end{equation*}
		\item 
		Cell centers through through polynomial interpolation (evaluate at $x_i$) 

		\item 
		Linear case: average of $\psi_{n,i,L}$ and $\psi_{n,i,R}$ 

		\item Sweep through local systems 

	\end{itemize}

	\end{column}
	\end{columns}

\end{frame}

% --- alternate form of LLDG slide --- 
% \begin{frame}{LLDG}

% 	2 discontinuous, linear basis functions 

% 	Cell edges uniquely defined through upwinding 

% 	Within the cell, 
% 	\begin{equation*}
% 		\psi_{n,i}(x) = \psi_{n,i,L} B_{i,L}(x) + \psi_{n,i,R} B_{i,R}(x) \,, \quad x \in (x_{i-1/2},x_{i+1/2}) \,, 
% 	\end{equation*}
% 	Cell centers through through polynomial interpolation (evaluate at $x_i$) 

% 	Linear case $\Rightarrow$ just average of $\psi_{n,i,L}$ and $\psi_{n,i,R}$ 

% 	\vfill
% 	\begin{figure}
% 		\centering
% 		\def\svgwidth{.8\textwidth}
% 		\input{figs/upwind.pdf_tex}
% 	\end{figure}

% \end{frame}

\begin{frame}{Handling Overlap in Eddington Factor}

	For integration by parts in MFEM weak form, need:
	\begin{itemize}
		\item $\edd$ on cell boundary 
		\item $\edd(x)$ on interior of cell 
	\end{itemize}

	Cell edges: use uniquely defined, upwinded cell edge values of $\psi$ in Gauss Quadrature 
	\begin{equation*} \label{lldg:edde}
		\edd_{i\pm 1/2} = \frac{
			\sum_{n=1}^N \mu_n^2 \psi_{n,i\pm 1/2} w_n
		}{
			\sum_{n=1}^N \psi_{n,i\pm 1/2} w_n 
		} 
	\end{equation*}

	Cell centers: use polynomial interpolation function 
	\begin{equation*} \label{lldg:eddi}
			\edd(x) = \frac{
				\sum_{n=1}^N \mu_n^2 \psi_{n}(x) w_n
			}{
				\sum_{n=1}^N \psi_{n}(x) w_n 
			} \,, \quad x\in(x_{i-1/2},x_{i+1/2}),
		\end{equation*}

	\begin{itemize}
		\item This is a rational polynomial $\Rightarrow$ can't be integrated analytically 

		\item Preserves linear spatial dependence of Eddington factor in MFEM formulation 

	\end{itemize}

\end{frame}

\begin{frame}{Constant-Linear Mixed Finite Element Drift Diffusion}

	\begin{columns}
	\begin{column}{.5\textwidth}

		\begin{itemize}

			\item Different basis functions for primary and secondary variables ($\phi$, $J$)

			\item $\phi$: constant with discontinuous jumps at the edges 

			\item $J$: linear discontinuous basis functions (same as in LLDG) 

		\end{itemize}
		
	\end{column}
	\begin{column}{.5\textwidth}

		\begin{figure}

			\def\svgwidth{\textwidth}
			%!TEX root = ./jctt.tex

\subsection{Mixed Finite-Element Method for VEF Equation}
\begin{figure}
	\centering
	% \def\svgwidth{\textwidth}
	\input{figs/mfemgrid.pdf_tex} 
	\caption{The distribution of unknowns in cell $i$ for MFEM. }
	\label{fig:mfem_grid}
\end{figure}
We apply the MFEM method to Eqs.~\ref{eq:zero} and \ref{eq:first} and then eliminate the currents to obtain a discretization for Eq.~\ref{eq:drift}.  In this 
method, the grid is identical to that used in the LLDG \SN discretization. The unknowns in an MFEM cell are depicted in Fig.~\ref{fig:mfem_grid}. In MFEM, separate basis functions are used for the scalar flux and 
current. The scalar flux is constant within the cell with discontinuous jumps at the cell edges and the current is a linear function defined by: 
	\begin{equation} \label{eq:MFEM_current}
		J_i(x) = J_{i,L} B_{i,L}(x) + J_{i,R} B_{R,i}(x) \,, 
	\end{equation} 
where $J_{i,L/R}$ are the currents at the left and right edges of the cell, and the basis functions are identical to those 
defined by Eqs.~\ref{eq:bfunL} and \ref{eq:bfunR} for the LLDG \SN discretization. The constant-linear MFEM yields second 
order accuracy for both the scalar flux and the current.  

The MFEM representation yields five unknowns per cell: $\phi_{i-1/2}$, $\phi_i$, $\phi_{i+1/2}$, $J_{i,L}$, and $J_{i,R}$. However, 
each edge flux on the mesh interior is shared by two cells, so with $I$ cells there are $I$ cell-center scalar fluxes, $2I$ currents, 
and $2I-1$ interior-mesh cell-edge scalar fluxes, and 2 boundary cell-edge scalar fluxes. An equation for $\phi_i$ is found by integrating Eq.~\ref{eq:zero} over cell $i$: 
	\begin{equation} \label{mfem:balance}
		J_{i,R} - J_{i,L} + \sigma_{a,i} h_i \phi_i = Q_i h_i \,,
	\end{equation}
where $\sigma_{a,i}$ and $Q_i$ are the absorption cross section and source in cell $i$. Equations for $J_{i,L/R}$ are found by multiplying Eq.~\ref{eq:first} by $B_{i,L/R}$ and integrating over cell $i$: 
	\begin{subequations}
		\begin{equation} \label{mfem:bli}
			-\edd_{i-1/2} \phi_{i-1/2} + \edd_i \phi_i + \sigma_{t,i} h_i \left(\frac{1}{3} J_{i,L} + \frac{1}{6}J_{i,R}\right) = 0 \,,
		\end{equation}
		\begin{equation} \label{mfem:bri}
			\edd_{i+1/2} \phi_{i+1/2} - \edd_i \phi_i + \sigma_{t,i} h_i \left(\frac{1}{6} J_{i,L} + \frac{1}{3} J_{i,R}\right) = 0 \,, 
		\end{equation}
	\end{subequations}
where the fixed source has been assumed to be isotropic. All Eddington factors are computed using the angular fluxes from the LLDG \SN step. Note that $\edd_{i\pm 1/2}$ denotes cell edge Eddington factors, while 
$\edd_{i}$ denotes an average over cell $i$ of the Eddington factors. The edge Eddington factors are defined by Eq.~\ref{lldg:edde}, while the Eddington factors within each cell 
are defined by Eq.~\ref{lldg:eddi}. We stress that evaluating Eq.~\ref{lldg:eddi} at $x_{i\pm1/2}$ does not yield $\edd_{i\pm 1/2}$ 
because of the upwinding used to define the cell edge angular fluxes. The spatial dependence of the Eddington factors within each cell takes the form of a rational polynomial prompting the use of numerical quadrature to compute the average. Two point Gauss quadrature was used:
	\be
	\edd_{i} = \frac{1}{2} \bracket{ \langle \mu^2 \rangle (x^G_{i,L}) + \langle \mu^2 \rangle (x^G_{i,R}) } 
	\ee
where 
\begin{equation} 
		% x^G_{i,L/R} = \frac{h_i}{2} \mp \frac{x_{i+1/2} + x_{i-1/2}}{2\sqrt{3}} \,.
		x^G_{i,L/R} = \frac{x_{i+1/2} + x_{i-1/2}}{2} \mp \frac{h_i}{2\sqrt{3}} \,.
\end{equation}

Eliminating $J_{i,R}$ from Eq.~\ref{mfem:bli} and $J_{i,L}$ from Eq.~\ref{mfem:bri} yields: 
	\begin{subequations}
		\begin{equation} \label{mfem:jli}
			J_{i,L} = \frac{-2}{\sigma_{t,i} h_i} \bigg\{
				2\br{\eddphi{i} - \eddphi{i-1/2}}
				- \br{\eddphi{i+1/2} - \eddphi{i}}
			\bigg\} \,,
		\end{equation}
		\begin{equation} \label{mfem:jri}
			J_{i,R} = \frac{-2}{\sigma_{t,i} h_i} \bigg\{
				2\br{\eddphi{i+1/2} - \eddphi{i}} 
				- \br{\eddphi{i} - \eddphi{i-1/2}}
			\bigg\} \,.
		\end{equation}
	\end{subequations}
An equation for $\phi_{i+1/2}$ on the mesh interior is found by enforcing continuity of current at the cell edges: 
	\begin{equation} \label{mfem:continuity}
		J_{i,R} = J_{i+1, L} \,. 
	\end{equation}

Using the definitions of $J_{i,L}$ and $J_{i,R}$ from Eqs.~\ref{mfem:jli} and \ref{mfem:jri} in the balance equation (Eq.~\ref{mfem:balance}) and continuity equation (Eq.~\ref{mfem:continuity}) yields equations for all cell-center fluxes and 
interior-mesh cell-edge fluxes.  The resulting balance and continuity equations are:
	\begin{subequations}
		\begin{equation} \label{mfem:center}
			-\frac{6}{\sigma_{t,i}h_i} \edd_{i-1/2} \phi_{i-1/2}
			+ \left(\frac{12}{\sigma_{t,i}h_i} \edd_i + \sigma_{a,i} h_i\right) \phi_i 
			- \frac{6}{\sigma_{t,i} h_i} \edd_{i+1/2} \phi_{i+1/2} 
			= Q_i h_i \,,
		\end{equation}
		\begin{multline} \label{mfem:edge}
			-\ALPHA{2}{i} \eddphi{i-1/2} + \ALPHA{6}{i} \eddphi{i} 
			- 4\paren{\ALPHA{1}{i} + \ALPHA{1}{i+1}} \eddphi{i+1/2} \\
			+ \ALPHA{6}{i+1}\eddphi{i+1} 
			- \ALPHA{2}{i+1} \eddphi{i+3/2}
			= 0 \,. 
		\end{multline}
	\end{subequations}
The equations for the outer boundary fluxes, $\phi_{1/2}$ and $\phi_{I+1/2}$, involve boundary conditions together with continuity conditions.  For instance,
the equation for $\phi_{1/2}$ is 
\begin{equation}
		J_{1,L} = J_{1/2} \,,
\end{equation}		
where $J_{1,L}$ is defined in Eq.~\ref{mfem:jli}, and $J_{1/2}$ is the left boundary current defined by a boundary condition.  For a reflective condition,
\begin{equation}
    J_{1/2} = 0 \, .
\end{equation}
For a source condition,
\begin{equation}
		J_{1/2} = 2 \sum_{\mu_n>0} \mu_n \psi_{n,1/2} w_n - B_{1/2} \phi_{1/2} \,,
\end{equation}  
 where  
\begin{equation}
		B_{1/2} = \frac{\sum_{n=1}^N |\mu_n| \psi_{n,1/2} w_n}{
			\sum_{n=1}^N \psi_{n,1/2} w_n 
		} 
\end{equation}
is the boundary Eddington factor \cite{QDBC}.  The equation for  $\phi_{I+1/2}$ is 
\begin{equation}
		J_{I,R} = J_{I+1/2} \, .
\end{equation}		
where $J_{I,R}$ is defined in Eq.~\ref{mfem:jri}, and $J_{I+1/2}$ is the right boundary current.  For a reflective condition,
\begin{equation}
    J_{I+1/2} = 0 \, .
\end{equation}
For a source condition,
\begin{equation}
		J_{I+1/2} = B_{I+1/2} \phi_{I+1/2} - 2 \sum_{\mu_n<0} |\mu_n| \psi_{n,I+1/2} w_n  \,,
\end{equation}  
where 
\begin{equation}
		B_{I+1/2} = \frac{\sum_{n=1}^N |\mu_n| \psi_{n,I+1/2} w_n}{
			\sum_{n=1}^N \psi_{n,I+1/2} w_n 
		} \, .
\end{equation}

These transport-consistent, Marshak-like source boundary conditions are derived starting with the identity
\begin{equation}
J_{1/2}=j^+ - j^- \,,
\end{equation}
where $j^\pm$ denotes the positive half-range currents associated with $\mu >0$ and $\mu <0$, respectively.  For the left boundary condition, we simply perform the following algebraic manipulations:
\begin{equation}
J_{1/2} = j^+ - j^-  = 2j^+ - (j^+ + j^-) = 2j^+ - \frac{j^+ + j^-}{\phi} \phi = 2j^+ - B_{1/2} \phi  \, .
\end{equation}
For the right boundary condition, we similarly obtain
\begin{equation}
J_{I+1/2} = j^+ - j^-  = (j^+ + j^-) - 2j^- = \frac{j^+ + j^-}{\phi} \phi - 2j^- = B_{I+1/2} \phi - 2j^-\, .
\end{equation}
Note that these source boundary conditions become equivalent to the standard Marshak boundary conditions if the \SN angular flux 
is isotropic. 
The resulting system of $2I+1$ equations for the cell-center and cell-edge fluxes can be assembled into a matrix of both cell-center and cell-edge scalar fluxes and solved with a banded matrix solver of bandwidth five. The resulting drift-diffusion scalar flux can either be used as the final solution if the solution has converged or as an update to the LLDG \SN scattering term. 

% Applying the MFEM to Eqs.~\ref{eq:zero} and \ref{eq:first} and enforcing continuity of current yields: 
% 	\begin{subequations} \label{eq:mfem}
	
% 	\begin{multline}
% 		-\frac{2}{\sigma_{t,i} h_i} \edd_{i-1/2}\phi_{i-1/2} + 
% 		\frac{6}{\sigma_{t,i} h_i} \edd_i \phi_i 
% 		- 4\left(\frac{1}{\sigma_{t,i} h_i} + \frac{1}{\sigma_{t,i+1} h_{i+1}}\right) 
% 			\edd_{i+1/2} \phi_{i+1/2}
% 		\\ + \frac{6}{\sigma_{t,i+1} h_{i+1}} \edd_{i+1} \phi_{i+1} 
% 		- \frac{2}{\sigma_{t,i+1} h_{i+1}} \edd_{i+3/2} \phi_{i+3/2} 
% 		= 0 \,,
% 	\end{multline}
% 	\end{subequations}
% where the Eddington factor is evaluated at iteration $\ell+1/2$ and the scalar flux at $\ell+1$. 
% Here, the Eddington factor has been assumed to be constant in each cell with discontinuous jumps at the edges. 
% The simplest method of converting the Eddington factor from LLDG to MFEM is to compute the Eddington factor using the cell centered and cell edged angular fluxes using Eqs.~\ref{eq:lldg_i}, \ref{eq:downwind}, and \ref{eq:upwind}. A more consistent way to transfer the Eddington factor is to represent the LLDG angular flux as a linear function using the MFEM basis functions: 
% 	\begin{equation} \label{eq:eddquad}
% 		\edd_i(x) = \frac{
% 			\sum_{n=1}^N \mu_n^2 \left[\psi_{n,i,L}B_{i,L}(x) + \psi_{n,i,R} B_{i,R}(x)\right]
% 		}
% 		{
% 			B_{i,L}(x) \sum_{n=1}^N w_n \psi_{n,i,L} + B_{i,R}(x) \sum_{n=1}^N w_n \psi_{n,i,R} 
% 		} \,,
% 	\end{equation}
% where 
	
% and 

% When MFEM is applied, the integral over cell $i$ of the rational polynomial given in Eq.~\ref{eq:eddquad} is approximated with 2 point Gauss quadrature. The cell centered Eddington factors used in Eq.~\ref{eq:mfem} are then: 
% 	\begin{equation} 
% 		\edd_i = \half \left[ \edd_i(x_{i,L}) + \edd_i(x_{i,R}) \right] \,,
% 	\end{equation}
% where 
% 	\begin{equation}
% 		x_{i,L/R} = \frac{x_{i+1/2} - x_{i-1/2}}{2} \mp \frac{x_{i+1/2} + x_{i-1/2}}{2\sqrt{3}}
% 	\end{equation}
% are the quadrature points in cell $i$. 

% Transport consistent vacuum boundary conditions are applied through a modified Marshak boundary condition: 
% 	\begin{equation} 
% 		J(x) = B(x) \phi(x) \,,
% 	\end{equation} 
% where 
% 	\begin{equation} 
% 		B(x) = \frac{\int_{-1}^1 |\mu| \psi(x, \mu) \ud \mu}
% 		{\int_{-1}^1 \psi(x, \mu) \ud \mu} \,. 
% 	\end{equation}

\subsection{Increased Consistency Between LLDG and MFEM}

The MFEM representation for the scalar flux is constant within a cell, but the LLDG representation for the scalar flux is linear.  This suggests that improved 
accuracy of the \SN solution could be achieved by somehow constructing a linear scalar flux dependence from the MFEM solution.  One simple method for doing 
this is to use the MFEM cell-edge scalar fluxes to compute a slope.  This works quite well, for neutronics.  However, it will be inadequate in a radiative 
transfer calculation because slopes must also be generated for the material temperatures, and an MFEM approximation for the temperatures will not include 
edge temperatures.  We have chosen to use a more generally applicable approach based upon standard data reconstruction techniques 
that require only cell-centered values to compute slopes \cite{vanLeer}.  We also limit such slopes to avoid non-physical scalar fluxes.  For example, the reconstructed left and right scalar fluxes in cell $i$ are given by 
	\begin{equation} \label{consistent:reconstruction}
		\phi_{i,L/R} = \phi_i \mp \frac{1}{4} \xi_i \left(\Delta \phi_{i+1/2} + \Delta \phi_{i-1/2}\right) \,,
	\end{equation}
where $\xi$ is a van Leer-type slope limiter \cite{vanLeer}:
\begin{subequations}
	\begin{equation} 
		\xi_i = \begin{cases}
			0, & r_i \leq 0 \,, \\
			\text{min}\bracet{\frac{2r_i}{1+r_i} , \frac{2}{1+r_i}} \,, & r_i > 0
		\end{cases} \,,
	\end{equation}
	\begin{equation}
		r_i = \frac{\Delta\phi_{i-1/2}}{\Delta \phi_{i+1/2}} \,,
	\end{equation}
\end{subequations}
and
	\begin{subequations}
		\begin{equation}
			\Delta \phi_{i+1/2} = \phi_{i+1} - \phi_i \,, 
		\end{equation}
		\begin{equation}
			\Delta \phi_{i-1/2} = \phi_i - \phi_{i-1} \,.
		\end{equation}
	\end{subequations}
On the boundaries, we use 
	\begin{subequations}
		\begin{equation}
			\phi_{1,L/R} = \phi_1 \mp \frac{1}{2} \Delta \phi_{3/2} \,,
		\end{equation}
		\begin{equation}
			\phi_{I,L/R} = \phi_I \mp \frac{1}{2} \Delta \phi_{I-1/2} \,.
		\end{equation}
	\end{subequations}
We also set any negative left or right flux values in the boundary cells to zero by appropriately rotating the slopes.
% \URL{I'm not sure if this was how the slopes were really calculated, but we need to fully explain what was done, 
% and we should do something to prevent negativities from arising.}
			% \caption{Distribution of unknowns for an MFEM cell. }

		\end{figure}

	\end{column}

	\end{columns}

	\begin{columns}
	\begin{column}{1.06\textwidth}

		\begin{itemize}

			\item 5 unknowns per cell 

			\item $\phi$ and $J$ are doubly defined on the edges but will later be made continuous through enforcing continuity of flux and current 

		\end{itemize}

	\end{column}
	\end{columns}

\end{frame}

\begin{frame}{Weak Form}

	System of first order equations form of drift diffusion:
	\begin{subequations} 
	\begin{equation*} \label{eq:zero}
		\dderiv{}{x} J (x) + \sigma_a(x) \phi(x) = Q(x)
	\end{equation*} 
	\begin{equation*} \label{eq:first}
		\dderiv{}{x} \bracket{\edd(x) \phi (x)} + \sigma_t(x) J(x) = 0
	\end{equation*}
	\end{subequations}

	\pause
	Multiply by $\phi$ basis function and integrate over cell $i$: 
	\begin{equation*}
		\int_{x_{i-1/2}}^{x_{i+1/2}} \dderiv{}{x} J (x) + \sigma_a(x) \phi(x) \ \ud x 
		= \int_{x_{i-1/2}}^{x_{i+1/2}} Q(x) \ \ud x
	\end{equation*}

	\pause
	Multiply by $J$ basis functions ($B_{i,L}$ and $B_{i,R}$) and integrate: 
	\begin{equation*}
		\int_{x_{i-1/2}}^{x_{i+1/2}} 
		B_{i,L/R}(x)\dderiv{}{x} \bracket{\edd(x) \phi (x)} + B_{i,L/R}(x)\sigma_t(x) J(x) \ud x = 0
	\end{equation*}

	

\end{frame}

\begin{frame}{Weak Form (cont.)}

	Integrate by parts:
	\begin{multline*}
		\int_{x_{i-1/2}}^{x_{i+1/2}} 
			B_{i,L/R}(x)\dderiv{}{x} \bracket{\edd(x) \phi (x)} \ud x = \\ 
		\underbrace{\vphantom{\int_{x_{i-1/2}}}
		\bracket{B_{i,L/R}(x)\edd(x)\phi(x)}_{x_{i-1/2}}^{x_{i+1/2}}}_\text{Edge Eddington Factors}
		- 
		\underbrace{\int_{x_{i-1/2}}^{x_{i+1/2}}
		\edd(x) \phi(x) \dderiv{B_{i,L/R}}{x} \ud x}_\text{Interior Eddington Factors}
	\end{multline*}

	\pause
	On the interior: 
	\begin{itemize}

		\item $\phi(x)$ and $\dderiv{B_{i,L/R}}{x}$ are constant (for linear case)

		\item Use Gauss Quadrature to approximate 
		\begin{equation*}
			\edd_i = \int_{x_{i-1/2}}^{x_{i+1/2}} \edd(x) \ud x 
		\end{equation*}

	\end{itemize}

\end{frame}

\begin{frame}{MFEM Closure}
	
	3 equations from weak form but 5 unknowns per cell 

	Enforce continuity of $\phi$ and $J$ at the interior cell edges:
	\begin{equation*}
		\phi_{i+1/2} = \phi_{(i+1)-1/2}
	\end{equation*}
	\begin{equation*}
		J_{i,R} = J_{i+1,L}
	\end{equation*}

	Use transport consistent, Marshak-like boundary conditions
	
	Can then eliminate $J$ and assemble a system of equations of cell centers and edges of $\phi$ only 

	Solve resulting Symmetric Positive Definite Matrix with a 5 band solver  

\end{frame}

\section{Scattering Update Methods}

\begin{frame}{Scattering Update Overlap}

	Must reconstruct an LLDG-like $\phi$ from the MFEM drift diffusion $\phi$ 

	\begin{figure}

		\def\svgwidth{\textwidth}
		% \input{figs/scattering.pdf_tex}
		\input{figs/scattering_color.pdf_tex}

	\end{figure}

	To remain general, reconstruct from cell centers only 

	\begin{itemize}
		\item Temperature equation will not have cell edges (no continuity of temperature)
	\end{itemize}


\end{frame}

\begin{frame}{Flat Scattering Update}

	Naive: flat update 
	\begin{equation*}
		{\color{founder}\phi_{i,L/R}} = {\color{medalist}\phi_i} 
	\end{equation*}

	\begin{figure}

		\def\svgwidth{\textwidth}
		% \input{figs/naive.pdf_tex}
		\input{figs/naive_color.pdf_tex}

	\end{figure}

	Converts constant MFEM to discontinuous constant in scattering term 

	Better: construct a linear dependence from neighboring MFEM cell centers 

\end{frame}

\begin{frame}{Linear Reconstruction}

	\onslide<1->{Compute slopes from neighboring cell centers}

	\onslide<2->{Generate an average slope from left and right slopes, apply van Leer-type slope limiting }

	\onslide<3->{Interpolate to cell edge }

	\begin{figure}

		% \def\svgwidth{\textwidth}
		% \input{figs/slopes.pdf_tex}
		\only<1>{\def\svgwidth{\textwidth}\input{figs/slopes_color.pdf_tex}}%
		\only<2>{\def\svgwidth{\textwidth}\input{figs/average_color.pdf_tex}}%
		\only<3,4>{\def\svgwidth{\textwidth}\input{figs/interpolate_color.pdf_tex}}

	\end{figure}

	\onslide<4->{
	This method: 
	\begin{itemize}

		\item Preserves the cell center value from MFEM 

		\item Reconstructs a linear, discontinuous $\phi$ from MFEM cell centers only 

		\item Uses slope limiting to prevent unphysical oscillations 

	\end{itemize}
	}

\end{frame}

\section{Computational Results}

\begin{frame}{Iterative Convergence Comparison}

	Relative iterative change (crude measure of iterative convergence)
	\begin{equation*}
		\frac{\| f^{\ell+1} - f^\ell \|_2}{\| f^{\ell+1} \|_2}
	\end{equation*}

	\vspace{-.2in}
	\begin{figure}[htb]
	\centering
	\begin{subfigure}{.515\textwidth}
		\centering
		\includegraphics[width=\textwidth]{figs/si.pdf}
		\caption{}
		\label{fig:si}
	\end{subfigure}
	\hspace{-2em}
	\begin{subfigure}{.515\textwidth}
		\centering
		\includegraphics[width=\textwidth]{figs/vef.pdf}  
		\caption{}
		\label{fig:vef}
	\end{subfigure}
	\caption{Relative iterative change for $\phi(x)$ and $\edd(x)$ for (a) unaccelerated and (b) VEF accelerated SI. }
	\end{figure}

\end{frame}

\begin{frame}{Comparison to SI and Consistently Differenced S$_2$SA}

	\begin{figure}[htb]
		\centering
		\includegraphics[width=.75\textwidth]{figs/si_vef_s2sa.pdf} 
		\caption{A comparison of the number of iterations required to converge for Source Iteration, VEF acceleration, and S$_2$SA for varying ratios of $\sigma_s$ to $\sigma_t$. } 
		\label{fig:si_vef_s2sa}
	\end{figure}

\end{frame}

\begin{frame}{Method of Manufactured Solutions}

	Set $Q(x, \mu_n)$ to force solution to 
	\begin{equation*}
		\phi(x) = \sin\left(\frac{\pi x}{x_b}\right)
	\end{equation*}

	Fit error to 
	\begin{equation*}
		E = C h^p 
	\end{equation*}

	\begin{table}[htb]
	\centering
	\begin{tabular}{|c|c|c|c|}
	\hline
	Update Method & $p$ & $C$ & $R^2$ \\ 
	\hline
		None & Constant & \num{1.997} & \num{0.682} & \num{9.9999e-01} \\
None & Linear & \num{1.998} & \num{0.687} & \num{1.0000e+00} \\
Center & Constant & \num{2.007} & \num{0.726} & \num{9.9992e-01} \\
Center & Linear & \num{2.009} & \num{0.732} & \num{9.9991e-01} \\

	\hline
	\end{tabular}
	% \caption{The order of accuracy, error, and $R^2$ values for flat and linear slope reconstruction scattering source update methods. }
	\label{tab:mms}
	\end{table}

	\pause 
	\begin{block}{}
		\centering Same order of accuracy but linear reconstruction is more accurate 
	\end{block}

\end{frame}

\begin{frame}{VEF Drift Diffusion/\SN Solution Convergence}

	Compare the L$_2$ norm of the difference between \SN and drift diffusion solutions for:
	\begin{itemize} 

		\item Homogeneous system with $\frac{\sigma_s}{\sigma_t} = .99$ 

		\item Reed's problem 

	\end{itemize}

	\begin{figure}
		\centering
		\includegraphics[width=\textwidth]{figs/reed_solution.pdf}
		\caption{VEF solution for Reed's problem. }
	\end{figure}

\end{frame}

\begin{frame}{VEF Drift Diffusion/\SN Solution Convergence (cont.)}

	\begin{figure}[htb]
		\centering
		\begin{subfigure}{.5\textwidth}
			\centering
			\includegraphics[width=\textwidth]{figs/solconv_homo.pdf}
			\caption{}
			\label{fig:homo}
		\end{subfigure}
		\hspace{-2em}
		\begin{subfigure}{.5\textwidth}
			\centering
			\includegraphics[width=\textwidth]{figs/solconv_reed.pdf}
			\caption{}
			\label{fig:reed}
		\end{subfigure}
		\caption{Comparison of difference between solutions for both scattering update methods for (a) homogeneous problem and (b) Reed's problem. }
	\end{figure}

	\pause
	\begin{block}{}
		\centering Linearly reconstructed solution is more convergent to \SN for homogeneous problem only 
	\end{block}

\end{frame}

\begin{frame}{Thick Diffusion Limit Test}

	Scale cross sections and source according to:
	\begin{equation*}
		\sigma_t(x) \rightarrow \sigma_t(x)/\epsilon, \
		\sigma_s(x) \rightarrow \epsilon \sigma_s(x), \
		Q(x) \rightarrow \epsilon Q(x)
	\end{equation*}

	Diffusion length is invariant 
	\begin{equation*}
		L^2 = \frac{D}{\sigma_a} = \frac{1}{3\sigma_t\sigma_a} \rightarrow
			\frac{1}{3 \sigma_t /\epsilon \sigma_a \epsilon}
	\end{equation*}

	As $\epsilon \rightarrow 0$, the system becomes diffusive 

	\vspace{-.2in}
	\begin{figure}[htb]
		\centering
		\includegraphics[height=.55\textheight]{figs/dl_it.pdf}
		% \caption{The number of iterations required for convergence in the diffusion limit ($\epsilon \rightarrow 0$). }
		\label{fig:dl_it}
	\end{figure}
	% \begin{figure}
	% 	\centering
	% 	\includegraphics[height=.3\textheight]{figs/dl_err.pdf}
	% 	\caption{The error between the VEF methods and the exact diffusion solution as $\epsilon \rightarrow 0$. }
	% 	\label{fig:dl_err}
	% \end{figure}

\end{frame}

\section{Conclusions and Future Work }

\begin{frame}{Conclusions}

	Successfully paired Lumped Linear Discontinuous Galerkin \SN with constant-linear Mixed Finite Element drift diffusion 

	Acceleration is as effective as consistently differenced S$_2$SA 

	Thick diffusion limit is preserved 

	Overlap between discretizations:
	\begin{itemize}
		\item Carried linear dependence from LLDG into MFEM 

		\item Used slope reconstruction with limiting to regenerate a linear dependence from MFEM 

	\end{itemize}

	Conservative drift diffusion equation can be coupled to other physics components

	Drift diffusion discretization can match other physics components while retaining benefits of DG \SN 

	Built in error estimator 

\end{frame}

\begin{frame}{Future Work}

	Extend to high order finite elements in 2/3D 

	Radiative transfer 

	Investigate the impact of the linear reconstruction method on the "teleportation effect"

\end{frame}


% begin uncounted slides ---------------------------
\appendix

\begin{frame}[allowframebreaks]{References}

	\nocite{*}
	\setbeamerfont{bibliography item}{size=\scriptsize}
	\setbeamerfont{bibliography entry author}{size=\scriptsize}
	\setbeamerfont{bibliography entry title}{size=\scriptsize}
	\setbeamerfont{bibliography entry location}{size=\scriptsize}
	\setbeamerfont{bibliography entry note}{size=\scriptsize}
	\setbeamertemplate{bibliography item}{\insertbiblabel}
	\bibliographystyle{siam}
	\bibliography{references}

\end{frame}

\begin{frame}[standout]
  Questions?
\end{frame}

\end{document}
