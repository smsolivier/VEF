%!TEX root = ./jctt.tex

\subsection{Mixed Finite-Element Method for VEF Equation}
\begin{figure}
	\centering
	% \def\svgwidth{\textwidth}
	\input{figs/mfemgrid.pdf_tex} 
	\caption{The distribution of unknowns in cell $i$ for MFEM. }
	\label{fig:mfem_grid}
\end{figure}
We apply the MFEM method to Eqs.~\ref{eq:zero} and \ref{eq:first} and then eliminate the currents to obtain a discretization for Eq.~\ref{eq:drift}.  In this 
method, the grid is identical to that used in the LLDG \SN discretization. The unknowns in an MFEM cell are depicted in Fig.~\ref{fig:mfem_grid}. In MFEM, separate basis functions are used for the scalar flux and 
current. The scalar flux is constant within the cell with discontinuous jumps at the cell edges and the current is a linear function defined by: 
	\begin{equation} \label{eq:MFEM_current}
		J_i(x) = J_{i,L} B_{i,L}(x) + J_{i,R} B_{R,i}(x) \,, 
	\end{equation} 
where $J_{i,L/R}$ are the currents at the left and right edges of the cell, and the basis functions are identical to those 
defined by Eqs.~\ref{eq:bfunL} and \ref{eq:bfunR} for the LLDG \SN discretization. The constant-linear MFEM yields second 
order accuracy for both the scalar flux and the current.  

The MFEM representation yields five unknowns per cell: $\phi_{i-1/2}$, $\phi_i$, $\phi_{i+1/2}$, $J_{i,L}$, and $J_{i,R}$. However, 
each edge flux on the mesh interior is shared by two cells, so with $I$ cells there are $I$ cell-center scalar fluxes, $2I$ currents, 
$2I-1$ interior-mesh cell-edge scalar fluxes, and 2 boundary cell-edge scalar fluxes. An equation for $\phi_i$ is found by integrating Eq.~\ref{eq:zero} over cell $i$: 
	\begin{equation} \label{mfem:balance}
		J_{i,R} - J_{i,L} + \sigma_{a,i} h_i \phi_i = Q_i h_i \,,
	\end{equation}
where $\sigma_{a,i}$ and $Q_i$ are the absorption cross section and source in cell $i$. Equations for $J_{i,L/R}$ are found by multiplying Eq.~\ref{eq:first} by $B_{i,L/R}$ and integrating over cell $i$: 
	\begin{subequations}
		\begin{equation} \label{mfem:bli}
			-\edd_{i-1/2} \phi_{i-1/2} + \edd_i \phi_i + \sigma_{t,i} h_i \left(\frac{1}{3} J_{i,L} + \frac{1}{6}J_{i,R}\right) = 0 \,,
		\end{equation}
		\begin{equation} \label{mfem:bri}
			\edd_{i+1/2} \phi_{i+1/2} - \edd_i \phi_i + \sigma_{t,i} h_i \left(\frac{1}{6} J_{i,L} + \frac{1}{3} J_{i,R}\right) = 0 \,. 
		\end{equation}
	\end{subequations}
All Eddington factors are computed using the angular fluxes from the LLDG \SN step. Note that $\edd_{i\pm 1/2}$ denotes cell edge Eddington factors, while 
$\edd_{i}$ denotes an average over cell $i$ of the Eddington factors. The edge Eddington factors are defined by Eq.~\ref{lldg:edde}, while the Eddington factors within each cell 
are defined by Eq.~\ref{lldg:eddi}. We stress that evaluating Eq.~\ref{lldg:eddi} at $x_{i\pm1/2}$ does not yield $\edd_{i\pm 1/2}$ 
because of the upwinding used to define the cell edge angular fluxes. The spatial dependence of the Eddington factors within each cell takes the form of a rational polynomial prompting the use of numerical quadrature to compute the average. Two point Gauss quadrature was used:
	\be
	\edd_{i} = \frac{1}{2} \bracket{ \langle \mu^2 \rangle (x^G_{i,L}) + \langle \mu^2 \rangle (x^G_{i,R}) } 
	\ee
where 
\begin{equation} 
		% x^G_{i,L/R} = \frac{h_i}{2} \mp \frac{x_{i+1/2} + x_{i-1/2}}{2\sqrt{3}} \,.
		x^G_{i,L/R} = \frac{x_{i+1/2} + x_{i-1/2}}{2} \mp \frac{h_i}{2\sqrt{3}} \,.
\end{equation}

Eliminating $J_{i,R}$ from Eq.~\ref{mfem:bli} and $J_{i,L}$ from Eq.~\ref{mfem:bri} yields: 
	\begin{subequations}
		\begin{equation} \label{mfem:jli}
			J_{i,L} = \frac{-2}{\sigma_{t,i} h_i} \bigg\{
				2\br{\eddphi{i} - \eddphi{i-1/2}}
				- \br{\eddphi{i+1/2} - \eddphi{i}}
			\bigg\} \,,
		\end{equation}
		\begin{equation} \label{mfem:jri}
			J_{i,R} = \frac{-2}{\sigma_{t,i} h_i} \bigg\{
				2\br{\eddphi{i+1/2} - \eddphi{i}} 
				- \br{\eddphi{i} - \eddphi{i-1/2}}
			\bigg\} \,.
		\end{equation}
	\end{subequations}
An equation for $\phi_{i+1/2}$ on the mesh interior is found by enforcing continuity of current at the cell edges: 
	\begin{equation} \label{mfem:continuity}
		J_{i,R} = J_{i+1, L} \,. 
	\end{equation}

Using the definitions of $J_{i,L}$ and $J_{i,R}$ from Eqs.~\ref{mfem:jli} and \ref{mfem:jri} in the balance equation (Eq.~\ref{mfem:balance}) and continuity equation (Eq.~\ref{mfem:continuity}) yields equations for all cell-center fluxes and 
interior-mesh cell-edge fluxes, respectively.  The resulting balance equation for cell $i$ and continuity equation for 
edge $i+1/2$ are respectively:
	\begin{subequations}
		\begin{equation} \label{mfem:center}
			-\frac{6}{\sigma_{t,i}h_i} \edd_{i-1/2} \phi_{i-1/2}
			+ \left(\frac{12}{\sigma_{t,i}h_i} \edd_i + \sigma_{a,i} h_i\right) \phi_i 
			- \frac{6}{\sigma_{t,i} h_i} \edd_{i+1/2} \phi_{i+1/2} 
			= Q_i h_i \,,
		\end{equation}
		and 
		\begin{multline} \label{mfem:edge}
			-\ALPHA{2}{i} \eddphi{i-1/2} + \ALPHA{6}{i} \eddphi{i} 
			- 4\paren{\ALPHA{1}{i} + \ALPHA{1}{i+1}} \eddphi{i+1/2} \\
			+ \ALPHA{6}{i+1}\eddphi{i+1} 
			- \ALPHA{2}{i+1} \eddphi{i+3/2}
			= 0 \,. 
		\end{multline}
	\end{subequations}
The equations for the outer boundary fluxes, $\phi_{1/2}$ and $\phi_{I+1/2}$, involve boundary conditions together with continuity conditions.  For instance,
the equation for $\phi_{1/2}$ is 
\begin{equation}
		J_{1,L} = J_{1/2} \,,
\end{equation}		
where $J_{1,L}$ is defined by Eq.~\ref{mfem:jli}, and $J_{1/2}$ is the left boundary current defined by a boundary condition.  For a reflective condition,
\begin{equation}
    J_{1/2} = 0 \, .
\end{equation}
For a source condition,
\begin{equation}
		J_{1/2} = 2 \sum_{\mu_n>0} \mu_n \psi_{n,1/2} w_n - B_{1/2} \phi_{1/2} \,,
\end{equation}  
 where  
\begin{equation}
		B_{1/2} = \frac{\sum_{n=1}^N |\mu_n| \psi_{n,1/2} w_n}{
			\sum_{n=1}^N \psi_{n,1/2} w_n 
		} 
\end{equation}
is the boundary Eddington factor \cite{QDBC}.  The equation for  $\phi_{I+1/2}$ is 
\begin{equation}
		J_{I,R} = J_{I+1/2} \, .
\end{equation}		
where $J_{I,R}$ is defined by Eq.~\ref{mfem:jri}, and $J_{I+1/2}$ is the right boundary current.  For a reflective condition,
\begin{equation}
    J_{I+1/2} = 0 \, .
\end{equation}
For a source condition,
\begin{equation}
		J_{I+1/2} = B_{I+1/2} \phi_{I+1/2} - 2 \sum_{\mu_n<0} |\mu_n| \psi_{n,I+1/2} w_n  \,,
\end{equation}  
where 
\begin{equation}
		B_{I+1/2} = \frac{\sum_{n=1}^N |\mu_n| \psi_{n,I+1/2} w_n}{
			\sum_{n=1}^N \psi_{n,I+1/2} w_n 
		} \, .
\end{equation}

These transport-consistent, Marshak-like source boundary conditions are derived starting with the identity
\begin{equation}
J_{1/2}=j^+ - j^- \,,
\end{equation}
where $j^\pm$ denotes the positive half-range currents associated with $\mu >0$ and $\mu <0$, respectively.  For the left boundary condition, we simply perform the following algebraic manipulations:
\begin{equation}
J_{1/2} = j^+ - j^-  = 2j^+ - (j^+ + j^-) = 2j^+ - \frac{j^+ + j^-}{\phi} \phi = 2j^+ - B_{1/2} \phi  \, .
\end{equation}
For the right boundary condition, we similarly obtain
\begin{equation}
J_{I+1/2} = j^+ - j^-  = (j^+ + j^-) - 2j^- = \frac{j^+ + j^-}{\phi} \phi - 2j^- = B_{I+1/2} \phi - 2j^-\, .
\end{equation}
Note that these source boundary conditions become equivalent to the standard Marshak boundary conditions if the \SN angular flux 
is isotropic. 
The resulting system of $2I+1$ equations for the cell-center and cell-edge fluxes can be assembled into a matrix of both cell-center and cell-edge scalar fluxes and solved with a banded matrix solver of bandwidth five. The resulting drift-diffusion scalar flux can either be used as the final solution if the solution has converged or as an update to the LLDG \SN scattering source. Use of these piecewise-constant fluxes to represent the LLDG \SN scattering source is the default method, and is referred to as the flat update method.

% Applying the MFEM to Eqs.~\ref{eq:zero} and \ref{eq:first} and enforcing continuity of current yields: 
% 	\begin{subequations} \label{eq:mfem}
	
% 	\begin{multline}
% 		-\frac{2}{\sigma_{t,i} h_i} \edd_{i-1/2}\phi_{i-1/2} + 
% 		\frac{6}{\sigma_{t,i} h_i} \edd_i \phi_i 
% 		- 4\left(\frac{1}{\sigma_{t,i} h_i} + \frac{1}{\sigma_{t,i+1} h_{i+1}}\right) 
% 			\edd_{i+1/2} \phi_{i+1/2}
% 		\\ + \frac{6}{\sigma_{t,i+1} h_{i+1}} \edd_{i+1} \phi_{i+1} 
% 		- \frac{2}{\sigma_{t,i+1} h_{i+1}} \edd_{i+3/2} \phi_{i+3/2} 
% 		= 0 \,,
% 	\end{multline}
% 	\end{subequations}
% where the Eddington factor is evaluated at iteration $\ell+1/2$ and the scalar flux at $\ell+1$. 
% Here, the Eddington factor has been assumed to be constant in each cell with discontinuous jumps at the edges. 
% The simplest method of converting the Eddington factor from LLDG to MFEM is to compute the Eddington factor using the cell centered and cell edged angular fluxes using Eqs.~\ref{eq:lldg_i}, \ref{eq:downwind}, and \ref{eq:upwind}. A more consistent way to transfer the Eddington factor is to represent the LLDG angular flux as a linear function using the MFEM basis functions: 
% 	\begin{equation} \label{eq:eddquad}
% 		\edd_i(x) = \frac{
% 			\sum_{n=1}^N \mu_n^2 \left[\psi_{n,i,L}B_{i,L}(x) + \psi_{n,i,R} B_{i,R}(x)\right]
% 		}
% 		{
% 			B_{i,L}(x) \sum_{n=1}^N w_n \psi_{n,i,L} + B_{i,R}(x) \sum_{n=1}^N w_n \psi_{n,i,R} 
% 		} \,,
% 	\end{equation}
% where 
	
% and 

% When MFEM is applied, the integral over cell $i$ of the rational polynomial given in Eq.~\ref{eq:eddquad} is approximated with 2 point Gauss quadrature. The cell centered Eddington factors used in Eq.~\ref{eq:mfem} are then: 
% 	\begin{equation} 
% 		\edd_i = \half \left[ \edd_i(x_{i,L}) + \edd_i(x_{i,R}) \right] \,,
% 	\end{equation}
% where 
% 	\begin{equation}
% 		x_{i,L/R} = \frac{x_{i+1/2} - x_{i-1/2}}{2} \mp \frac{x_{i+1/2} + x_{i-1/2}}{2\sqrt{3}}
% 	\end{equation}
% are the quadrature points in cell $i$. 

% Transport consistent vacuum boundary conditions are applied through a modified Marshak boundary condition: 
% 	\begin{equation} 
% 		J(x) = B(x) \phi(x) \,,
% 	\end{equation} 
% where 
% 	\begin{equation} 
% 		B(x) = \frac{\int_{-1}^1 |\mu| \psi(x, \mu) \ud \mu}
% 		{\int_{-1}^1 \psi(x, \mu) \ud \mu} \,. 
% 	\end{equation}

\subsection{Increased Consistency Between LLDG and MFEM}

The MFEM representation for the scalar flux is constant within a cell, but the LLDG representation for the scalar flux is linear.  This suggests that improved 
accuracy of the \SN solution could be achieved by somehow constructing a linear scalar flux dependence from the MFEM solution.  One simple method for doing 
this is to use the MFEM cell-edge scalar fluxes to compute a slope, which is then combined with the MFEM cell-center flux value to 
obtain a linear dependence.   This works quite well for neutronics.  However, it will be inadequate in a radiative transfer calculation because slopes must also be generated for the material temperatures, and an MFEM approximation for the temperatures will not include 
edge temperatures.  We have chosen to use a more generally applicable approach based upon standard data reconstruction techniques 
that require only cell-centered values to compute slopes \cite{vanLeer}.  We also limit such slopes to avoid non-physical scalar fluxes.  For example, the reconstructed left and right scalar fluxes in cell $i$ are given by 
	\begin{equation} \label{consistent:reconstruction}
		\phi_{i,L/R} = \phi_i \mp \frac{1}{4} \xi_i \left(\Delta \phi_{i+1/2} + \Delta \phi_{i-1/2}\right) \,,
	\end{equation}
where $\xi$ is a van Leer-type slope limiter \cite{vanLeer}: 
\begin{subequations}
	\begin{equation} 
		\xi_i = \begin{cases}
			0, & r_i \leq 0 \,, \\
			\text{min}\bracet{\frac{2r_i}{1+r_i} , \frac{2}{1+r_i}} \,, & r_i > 0
		\end{cases} \,,
	\end{equation}
	\begin{equation}
		r_i = \frac{\Delta\phi_{i-1/2}}{\Delta \phi_{i+1/2}} \,,
	\end{equation}
\end{subequations}
and
	\begin{subequations}
		\begin{equation}
			\Delta \phi_{i+1/2} = \phi_{i+1} - \phi_i \,, 
		\end{equation}
		\begin{equation}
			\Delta \phi_{i-1/2} = \phi_i - \phi_{i-1} \,.
		\end{equation}
	\end{subequations}

On the boundaries, we use 
	\begin{subequations}
		\begin{equation}
			\phi_{1,L/R} = \phi_1 \mp \frac{1}{2} \Delta \phi_{3/2} \,,
		\end{equation}
		\begin{equation}
			\phi_{I,L/R} = \phi_I \mp \frac{1}{2} \Delta \phi_{I-1/2} \,.
		\end{equation}
	\end{subequations}
We also set any negative left or right flux values in the boundary cells to zero by appropriately rotating the slopes. 
