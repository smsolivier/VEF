%!TEX root = ./jctt.tex

\newcommand{\rell}{^\ell} % raise to ellth power 
\newcommand{\relll}{^{\ell+1}} % raise to ell + 1 th power 
\newcommand{\rellh}{^{\ell+1/2}} % raise to ell + 1/2 power

\newcommand{\paren}[1]{\left(#1\right)} 
\newcommand{\br}[1]{\left[#1\right]}
\newcommand{\curl}[1]{\left\{#1\right\}}

\newcommand{\eddphi}[1]{\edd_{#1}\phi_{#1}}
\newcommand{\ALPHA}[2]{\frac{#1}{\sigma_{t,#2} h_{#2}}}

\section{The VEF Method}
\subsection{The Algorithm}
Here, we describe the VEF method for a planar geometry, fixed-source problem:
	\begin{equation} 
		\mu \pderiv{\psi}{x} \paren{x, \mu} + \sigma_t(x) \psi(x,\mu) = 
			\frac{\sigma_s(x)}{2} \int_{-1}^1 \psi(x,\mu') \ud \mu' + \frac{Q(x)}{2} \,,
	\end{equation}
where $\mu = \cos\theta$ is the cosine of the angle of flight $\theta$ relative to the $x$--axis, $\sigma_t(x)$ and $\sigma_s(x)$ the total and scattering macroscopic cross sections, $Q(x)$ the isotropic fixed-source and $\psi(x, \mu)$ the angular flux. Applying the Discrete Ordinates (\SN) angular discretization yields the following set of $N$ coupled, ordinary differential equations: 
	\begin{equation} \label{eq:sn}
		\mu_n \dderiv{\psi_n}{x}(x) + \sigma_t(x) \psi_n(x) = 
		\frac{\sigma_s(x)}{2} \phi(x) + \frac{Q(x)}{2} \,, 1 \leq n \leq N \,,
	\end{equation}
where $\psi_n(x) = \psi(x, \mu_n)$ is the angular flux in direction $\mu_n$. The $\mu_n$ are stipulated by an $N$-point Gauss quadrature rule such that the scalar flux, $\phi(x)$, can be numerically integrated with: 
	\begin{equation} \label{eq:phiquad}
		\phi(x) = \sum_{n=1}^N w_n \psi_n(x) \,,
	\end{equation}
where the $w_n$ are the quadrature weights corresponding to the $\mu_n$. 

The VEF method decouples Eq. \ref{eq:sn} by lagging the scattering term: 
	\begin{equation} \label{eq:si}
		\mu_n \dderiv{\psi_n\rellh}{x}(x) + \sigma_t(x) \psi_n\rellh(x) = 
		\frac{\sigma_s(x)}{2} \phi^\ell(x) + \frac{Q(x)}{2} \,, 1 \leq n \leq N \,,
	\end{equation}
where the superscripts indicate the iteration index. The scalar flux used in the scattering term, $\phi\rell$, is assumed to be known either from the previous iteration or from the initial guess if $\ell=0$. In Source Iteration (SI), the update 
	\begin{equation} \label{eq:siupdate}
		\phi(x)\relll = \phi(x)\rellh
	\end{equation}
is used. However, this is slow to converge in optically thick and highly scattering systems. Instead, the VEF method solves the VEF drift diffusion equations found by taking the first two angular moments of Eq. \ref{eq:sn}: 
	\begin{subequations} 
	\begin{equation} \label{eq:zero}
		\dderiv{}{x} J\relll(x) + \sigma_a(x) \phi\relll(x) = Q(x) \,,
	\end{equation} 
	\begin{equation} \label{eq:first}
		\frac{\ud}{\ud x} \edd\rellh(x) \phi\relll(x) + \sigma_t(x) J\relll(x) = 0 \,,
	\end{equation}
	\end{subequations}
where $J\relll(x)$ is the current and 
	\begin{equation} \label{eq:eddington} 
		\edd\rellh(x) = \frac{\int_{-1}^1 \mu^2 \psi\rellh(x, \mu) \ud \mu}{\int_{-1}^1 \psi\rellh(x, \mu) \ud \mu}
		\xrightarrow{\text{\SN}} \frac{
			\sum_{n=1}^N \mu_n^2 \psi_n\rellh(x) w_n
		}{
			\sum_{n=1}^N \psi_n\rellh(x) w_n 
		}
	\end{equation}
the Eddington factor. The scattering term in Eq. \ref{eq:si} is then updated with the VEF drift diffusion scalar flux found by solving Eqs. \ref{eq:zero} and \ref{eq:first}. This process of solving Eq. \ref{eq:si} for the $\psi_n(x)$, computing the Eddington factor, solving the VEF drift diffusion equation for the scalar flux, and updating the scattering term with the VEF drift diffusion scalar flux is repeated until convergence. 

Acceleration occurs because the angular shape of the angular flux, and thus the Eddington factor, converges much faster than the scalar flux. In addition, the VEF equations model the contributions of all scattering events at once, reducing the dependence on source iterations to introduce scattering information. 
% The solution from the VEF equations is then an approximation for the full flux and not the $\ell - 1$ collided flux as it was without acceleration. 

In addition to acceleration, this scheme allows the \SN equations and drift diffusion equations to be solved with arbitrarily different spatial discretization methods. The following sections  present the application of the Lumped Linear Discontinuous Galerkin (LLDG) spatial discretization to the \SN equations and the Mixed Finite Element Method (MFEM) to the VEF drift diffusion equations. 

\subsection{Lumped Linear Discontinuous Galerkin \SN}
\begin{figure}
	\centering
	\input{figs/lldggrid.pdf_tex}
	\caption{The distribution of unknowns in an LLDG cell. The superscript $+$ and $-$ indicate the angular fluxes for $\mu_n>0$ and $\mu_n<0$, respectively. } 
	\label{fig:lldg_grid}
\end{figure}
The spatial grid and distribution of unknowns for an LLDG cell are shown in Fig. \ref{fig:lldg_grid}. We assume a computational domain of length $x_b$ discretized into $I$ cells. The cell centers are integral and the cell edges are half integral. The two unknowns in each cell for each discrete angle are the left and right edge discontinuous angular fluxes, $\psi_{n,i,L}\rellh$ and $\psi_{n,i,R}\rellh$. 

The LLDG discretization of Eq. \ref{eq:si} is then: 
	\begin{subequations} 
	\begin{equation} \label{eq:lldg_l}
		\mu_n \left(\psi_{n,i}\rellh - \psi_{n, i-1/2}\rellh\right) 
		+ \frac{\sigma_{t,i} h_i}{2} \psi_{n,i,L}\rellh
		= \frac{\sigma_{s,i} h_i}{4} \phi_{i,L}\rell + \frac{h_i}{4} Q_{i,L} \,, 
		% 1 \leq n \leq N \,, 
		% 1 \leq i \leq I\,, 
	\end{equation}
	\begin{equation} \label{eq:lldg_r}
		\mu_n \left(\psi_{n,i+1/2}\rellh - \psi_{n,i}\rellh\right) 
		+ \frac{\sigma_{t,i} h_i}{2} \psi_{n,i,R}\rellh
		= \frac{\sigma_{s,i} h_i}{4} \phi_{i,R}\rell + \frac{h_i}{4} Q_{i,R} \,, 
		% 1 \leq n \leq N \,, 
		% 1 \leq i \leq I\,,
	\end{equation}
	\end{subequations}
where $h_i$, $\sigma_{t,i}$, $\sigma_{s,i}$, and $Q_{i,L/R}$ are the cell width, total cross section, scattering cross section and discontinuous fixed source in cell $i$. The discontinuous scalar fluxes, $\phi_{i,L/R}\rell$, are assumed to be known from the previous iteration or the initial guess when $\ell=0$. The cell edged angular fluxes are uniquely defined by upwinding: 
	\begin{subequations}
	\begin{equation} \label{eq:downwind}
		\psi_{n,i-1/2}\rellh = \begin{cases}
			\psi_{n,i-1,R}\rellh \,, & \mu_n > 0 \\ 
			\psi_{n,i,L}\rellh \,, & \mu_n < 0 
		\end{cases} \,,
	\end{equation}
	\begin{equation} \label{eq:upwind}
		\psi_{n,i+1/2}\rellh = \begin{cases}
			\psi_{n,i,R}\rellh \,, & \mu_n > 0 \\
			\psi_{n,i+1,L}\rellh \,, & \mu_n < 0 
		\end{cases} \,,
	\end{equation}
	\end{subequations} 
and the cell centered angular flux is the average of the left and right discontinuous edge fluxes:
	\begin{equation} \label{eq:lldg_i}
		\psi_{n,i}\rellh = \half\left(\psi_{n,i,L}\rellh + \psi_{n,i,R}\rellh\right) \,.
	\end{equation}
Equations \ref{eq:lldg_l}, \ref{eq:lldg_r}, \ref{eq:downwind}, \ref{eq:upwind}, and \ref{eq:lldg_i} can be combined and rewritten as 
	\begin{equation} \label{eq:sweepLR}
		\left[\begin{matrix}
			\mu_n + \sigma_{t,i} h_i & \mu_n  \\ 
			-\mu_n & \sigma_{t,i} + \mu_n \\ 
		\end{matrix}\right]
		\left[\begin{matrix}
			\psi_{n,i,L}\rellh \\ \psi_{n,i,R}\rellh
		\end{matrix}\right]
		= \left[\begin{matrix}
			\frac{\sigma_{s,i}h_i}{2} \phi_{i,L}\rell + \frac{h_i}{2} Q_{i,L} + 2\mu_n \psi_{n,i-1,R}\rellh \\
			\frac{\sigma_{s,i}h_i}{2} \phi_{i,R}\rell + \frac{h_i}{2} Q_{i,R} 
		\end{matrix}\right] \,, 
	\end{equation}
for sweeping from left to right ($\mu_n > 0$) and 
	\begin{equation} \label{eq:sweepRL}
		\left[\begin{matrix} 
			-\mu_n + \sigma_{t,i}h_i & \mu_n \\ 
			-\mu_n & -\mu_n + \sigma_{t,i}h_i \\ 
		\end{matrix} \right]
		\left[\begin{matrix}
			\psi_{n,i,L}\rellh \\ \psi_{n,i,R}\rellh
		\end{matrix} \right]
		= \left[\begin{matrix}
			\frac{\sigma_{s,i}h_i}{2} \phi_{i,L}\rell + \frac{h_i}{2} Q_{i,L} \\ 
			\frac{\sigma_{s,i}h_i}{2} \phi_{i,R}\rell + \frac{h_i}{2} Q_{i,R} - 2\mu_n \psi_{n,i+1,L}\rellh
		\end{matrix} \right]
		\,, 
	\end{equation}
for sweeping from right to left ($\mu_n < 0$). The right hand sides of Eqs. \ref{eq:sweepLR} and \ref{eq:sweepRL} are known as the scalar flux from the previous iteration, the fixed source, and the angular flux entering from the downwind cell are all known. By supplying the flux entering the left side of the first cell, the positive-angled solution can be propagated from left to right by solving Eq. \ref{eq:sweepLR}. Similarly, supplying the incident flux on the right boundary allows the negative-angled solution to be propagated from right to left with Eq. \ref{eq:sweepRL}. The cell edge and cell centered Eddington factors needed in the VEF acceleration step are computed with: 
	\begin{equation} \label{lldg:edd}
		\edd_{i(\pm1/2)}\rellh = \frac{
			\sum_{n=1}^N \mu_n^2 \psi_{n,i(\pm1/2)}\rellh w_n
		}{
			\sum_{n=1}^N \psi_{n,i(\pm1/2)}\rellh w_n 
		} \,,
	\end{equation}
where the $\psi_{n,i\pm1/2}\rellh$ are defined by Eqs. \ref{eq:downwind} and \ref{eq:upwind} and $\psi_{n,i}$ by Eq. \ref{eq:lldg_i}. 