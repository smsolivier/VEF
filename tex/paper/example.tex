\documentclass[12pt]{article}
\usepackage{afterpage}
\usepackage{amsmath}
\usepackage{amsfonts}
\usepackage{amssymb}
\usepackage{amsbsy}
\usepackage{bm}
\usepackage{epsfig}
\usepackage{rotating}
\usepackage{setspace}
\usepackage{tabls}
\usepackage{hhline}
\usepackage{float}
\usepackage{subfigure}
\usepackage{float}
%\usepackage{subfigmat}
%\usepackage{citesort}
%\usepackage{cites}
%\usepackage{overcite}
%\usepackage[section]{placeins}

% uncomment for submission of manuscript to NSE
\usepackage[nolists, nomarkers]{endfloat}
%
% use to include postscript figures
\usepackage{graphicx,color}
%
%\usepackage[light,firsttwo]{draftcopy}
%\draftcopySetGrey{0.90}

%\usepackage{dbl}

% -----------------------------------------------------------------------------
% define newcommands
% -----------------------------------------------------------------------------

%\setlength{\floatsep}{4pt plus 1pt minus 1pt}
%\setlength{\textfloatsep}{8pt plus 1pt minus 1pt}
%\setlength{\intextsep}{4pt plus 1pt minus 1pt}
%\setlength{\abovedisplayskip}{4pt plus 1pt minus 1pt}
%\setlength{\belowdisplayskip}{4pt plus 1pt minus 1pt}

\makeatletter
\renewcommand{\@thesubfigure}{\thefigure\thesubfigure\space}
\makeatother
% =================================================================================================
% more new commands
% +++++++++++++++++++++++++++++++++++++++++++++++++++++++++++++++++++++++++++++++++++++++++++++++++
\setlength{\textwidth}{6.5in}
\setlength{\textheight}{8.in}
\setlength{\oddsidemargin}{0in}
%\setlength{\topmargin}{0.75in}
\setlength{\headsep}{12pt}
%\addtolength{\oddsidemargin}{-0.5in}
%\addtolength{\textwidth}{1.0in}
%\addtolength{\textheight}{1.0in}
\renewcommand{\thefootnote}{\fnsymbol{footnote}}
%
% -----------------------------------------------------------------------------
% define newcommands
% -----------------------------------------------------------------------------

%\setlength{\floatsep}{4pt plus 1pt minus 1pt}
%\setlength{\textfloatsep}{8pt plus 1pt minus 1pt}
%\setlength{\intextsep}{4pt plus 1pt minus 1pt}
%\setlength{\abovedisplayskip}{4pt plus 1pt minus 1pt}
%\setlength{\belowdisplayskip}{4pt plus 1pt minus 1pt}

% =================================================================================================
% more new commands
% +++++++++++++++++++++++++++++++++++++++++++++++++++++++++++++++++++++++++++++++++++++++++++++++++

% Ways of grouping things
%
\newcommand{\bracket}[1]{\left[ #1 \right]}
\newcommand{\bracet}[1]{\left\{ #1 \right\}}
\newcommand{\fn}[1]{\left( #1 \right)}
\newcommand{\ave}[1]{\left\langle #1 \right\rangle}
%
% Derivative forms
%
\newcommand{\dx}[1]{\,d#1}
\newcommand{\dxdy}[2]{\frac{\partial #1}{\partial #2}}
\newcommand{\dxdt}[1]{\frac{\partial #1}{\partial t}}
\newcommand{\dxdz}[1]{\frac{\partial #1}{\partial z}}
\newcommand{\dddx}[1]{\frac{d #1}{d x}}
\newcommand{\dfdt}[1]{\frac{\partial}{\partial t} \fn{#1}}
\newcommand{\dfdz}[1]{\frac{\partial}{\partial z} \fn{#1}}
\newcommand{\ddt}[1]{\frac{\partial}{\partial t} #1}
\newcommand{\ddz}[1]{\frac{\partial}{\partial z} #1}
\newcommand{\dd}[2]{\frac{\partial}{\partial #1} #2}
\newcommand{\ddx}[1]{\frac{\partial}{\partial x} #1}
\newcommand{\ddy}[1]{\frac{\partial}{\partial y} #1}
%
% Vector forms
%
%\renewcommand{\vec}[1]{\ensuremath{\stackrel{\rightarrow}{#1}}}
%\renewcommand{\div}{\ensuremath{\vec{\nabla} \cdot}}
%\newcommand{\grad}{\ensuremath{\vec{\nabla}}}
\renewcommand{\vec}[1]{\overrightarrow{#1}}
\renewcommand{\div}{\vec{\nabla}\! \cdot \!}
\newcommand{\grad}{\vec{\nabla}}
\newcommand{\oa}[1]{\fn{\frac{1}{3}\hat{\Omega}\!\cdot\!\overrightarrow{A_{#1}}}}

%
% Equation beginnings and endings
%
\newcommand{\bea}{\begin{eqnarray}}
\newcommand{\eea}{\end{eqnarray}}
\newcommand{\be}{\begin{equation}}
\newcommand{\ee}{\end{equation}}
\newcommand{\beas}{\begin{eqnarray*}}
\newcommand{\eeas}{\end{eqnarray*}}
\newcommand{\bdm}{\begin{displaymath}}
\newcommand{\edm}{\end{displaymath}}


%
% Equation punctuation
%
\newcommand{\pec}{\; ,}
\newcommand{\pep}{\; .}
%
% Equation labels and references, figure references, table references
%
\newcommand{\LEQ}[1]{\label{eq:#1}}
\newcommand{\EQ}[1]{Eq.~(\ref{eq:#1})}
\newcommand{\EQS}[1]{Eqs.~(\ref{eq:#1})}
\newcommand{\REQ}[1]{\ref{eq:#1}}
\newcommand{\LFI}[1]{\label{fi:#1}}
\newcommand{\FI}[1]{Fig.~\ref{fi:#1}}
\newcommand{\RFI}[1]{\ref{fi:#1}}
\newcommand{\LTA}[1]{\label{ta:#1}}
\newcommand{\TA}[1]{Table~\ref{ta:#1}}
\newcommand{\RTA}[1]{\ref{ta:#1}}

%
% List beginnings and endings
%
\newcommand{\bl}{\bss\begin{itemize}}
\newcommand{\el}{\vspace{-.5\baselineskip}\end{itemize}\ess}
\newcommand{\ben}{\bss\begin{enumerate}}
\newcommand{\een}{\vspace{-.5\baselineskip}\end{enumerate}\ess}
%
% Figure and table beginnings and endings
%
\newcommand{\bfg}{\begin{figure}}
\newcommand{\efg}{\end{figure}}
\newcommand{\bt}{\begin{table}}
\newcommand{\et}{\end{table}}
%
% Tabular and center beginnings and endings
%
\newcommand{\bc}{\begin{center}}
\newcommand{\ec}{\end{center}}
\newcommand{\btb}{\begin{center}\begin{tabular}}
\newcommand{\etb}{\end{tabular}\end{center}}
%
% Single space command
%
%\newcommand{\bss}{\begin{singlespace}}
%\newcommand{\ess}{\end{singlespace}}
\newcommand{\bss}{\singlespacing}
\newcommand{\ess}{\doublespacing}
%
%---New environment "arbspace". (modeled after singlespace environment
%                                in Doublespace.sty)
%   The baselinestretch only takes effect at a size change, so do one.
%
\def\arbspace#1{\def\baselinestretch{#1}\@normalsize}
\def\endarbspace{}
\newcommand{\bas}{\begin{arbspace}}
\newcommand{\eas}{\end{arbspace}}
%
% An explanation for a function
%
\newcommand{\explain}[1]{\mbox{\hspace{2em} #1}}
%
% Quick commands for symbols
%
\newcommand{\half}{\frac{1}{2}}
\newcommand{\halff}{1/2}
\newcommand{\third}{\frac{1}{3}}
\newcommand{\twothird}{\frac{2}{3}}
\newcommand{\mdot}{\dot{m}}
\newcommand{\ten}[1]{\times 10^{#1}\,}
\newcommand{\cL}{{\cal L}}
\newcommand{\cD}{{\cal D}}
\newcommand{\cF}{{\cal F}}
\newcommand{\cE}{{\cal E}}
\renewcommand{\Re}{\mbox{Re}}
\newcommand{\Ma}{\mbox{Ma}}
%
% Inclusion of Graphics Data
%
%\input{psfig}
%\psfiginit
%
% More Quick Commands
%
\newcommand{\bi}{\begin{itemize}}
\newcommand{\ei}{\end{itemize}}
\newcommand{\dxi}{\Delta x_i}
\newcommand{\dyj}{\Delta y_j}
\newcommand{\ts}[1]{\textstyle #1}
% Mark URL's
\newcommand{\URL}[1]{{\textcolor{blue}{#1}}}



% Alter some LaTeX defaults for better treatment of figures:
    % See p.105 of "TeX Unbound" for suggested values.
    % See pp. 199-200 of Lamport's "LaTeX" book for details.
    %   General parameters, for ALL pages:
    \renewcommand{\topfraction}{0.9}	% max fraction of floats at top
    \renewcommand{\bottomfraction}{0.8}	% max fraction of floats at bottom
    %   Parameters for TEXT pages (not float pages):
    \setcounter{topnumber}{2}
    \setcounter{bottomnumber}{2}
    \setcounter{totalnumber}{2}     % 2 may work better
    \setcounter{dbltopnumber}{2}    % for 2-column pages
    \renewcommand{\dbltopfraction}{0.9}	% fit big float above 2-col. text
    \renewcommand{\textfraction}{0.07}	% allow minimal text w. figs
    %   Parameters for FLOAT pages (not text pages):
    \renewcommand{\floatpagefraction}{0.7}	% require fuller float pages
	% N.B.: floatpagefraction MUST be less than topfraction !!
    \renewcommand{\dblfloatpagefraction}{0.7}	% require fuller float pages

	% remember to use [htp] or [htpb] for placement
	
	
% =================================================================================================
\date{}

\begin{document}

%\bibliographystyle{nse}
%\bibnum{p}

\thispagestyle{empty}
\bc
{\Large \bf ASYMPTOTIC P$_N$-EQUIVALENT S$_{N+1}$ EQUATIONS}\\
\vspace{0.5in}
{\large {\bf J.E. Morel$^{a}$, J.C. Ragusa$^{a}$, M.L.\ Adams$^{a}$, G. Kanschat$^{b}$}\\
$ $\\
$^a$Department of Nuclear Engineering\\
$^b$Department of Mathematics\\
Texas A\&M University\\
College Station, TX 77843}\\
\ec
$ $\\
\bc
{\large \bf Abstract}\\
\ec
\noindent
\emph{
The 1-D one-speed slab-geometry P$_N$ equations with isotropic scattering can be 
modified via an alternative moment closure to preserve the two asymptotic eigenmodes 
associated with the transport equation.  Pomraning referred to these equations as 
the asymptotic P$_N$ equations. It is well-known that the 1-D slab-geometry S$_{N+1}$ 
equations with Gauss quadrature are equivalent to the standard P$_{N}$ equations.  
In this paper, we first show that if any quadrature set meets a certain criterion, 
the corresponding S$_{N+1}$ equations will be equivalent to a set of P$_N$ equations with a 
quadrature-dependent closure.  We then derive a particular family of quadrature sets that make 
the S$_{N+1}$ equations equivalent to the asymptotic P$_N$ equations. Next we theoretically demonstrate 
several of the properties of these sets, relate them to an existing family of quadratures, 
numerically generate several example quadrature sets, and give numerical results that 
confirm several of their theoretically predicted properties.
}\\
$ $\\
\noindent
{\bf Keywords}\\
\noindent P$_N$ equations, S$_N$ equations, asymptotic decay lengths.\\
$ $\\
\noindent {\bf Running Head}\\
\noindent Asymptotic S$_{N+1}$ Equations\\
$ $\\
\noindent{\bf Corresponding Author}\\
\noindent Jim E. Morel, Phone: (979)845-6072, FAX: (979)845-6075, E-mail: \emph{morel@tamu.edu}.
$ $\\
\newpage
\ess
%\bc
%{\large \bf I. Introduction}
%\ec
\section{Introduction}

Pomraning developed a generalized P$_N$ method based upon alternatives to the usual closure that 
assumes the angular flux moment corresponding to P$_{N+1}$ is zero \cite{pomraning}.  This generalized 
approach takes two forms.  One preserves forward-peaked and backward-peaked delta-function solutions, 
and the other preserves the two asymptotic eigenmodes associated with the exact transport 
solution. We refer to the latter approximation as the asymptotic P$_N$ equations.  Both of the asymptotic 
modes have a decay length that is larger than those of all other eigenmodes, so the asymptotic modes 
dominate at deep penetration in a homogeneous medium.   


It is well known that in 1-D slab geometry the S$_{N+1}$ equations with Gauss quadrature are equivalent 
to the standard P$_{N}$ equations.  The purpose of this paper is to demonstrate that an analogous equivalence 
exists between the $S_{N+1}$ equations with appropriate quadrature and the asymptotic P$_N$ equations.  Several 
families of S$_N$ quadrature sets that preserve the transport asymptotic decay length were derived by Ganguley, 
et al. \cite{ganguley}. Although these sets were derived without consideration of P$_N$ equations, we have found 
that our family of asymptotic quadratures is identical to one of those families. We have no proof of this property, 
but rather have simply observed it by comparing quadrature sets that we generated with those published 
by Ganguley, et. al. 

The remainder of this paper is organized as follows.  First we derive Ponraning's asmptotic P$_N$ equations. 
Next we show that under the assumption of a certain property of the quadrature set, the S$_{N+1}$ equations are 
equivalent to the P$_N$ equations with a truncation determined by the quadrature set. This is a general property 
independent of Pomraning's closure.  The particular quadrature sets that produce Ponraning's asymptotic closure 
are next derived.  Various properties of these sets are then demonstrated including certain properties that 
Ganguly, et al., did not make reference to.  Boundary conditions and interface conditions are next given for the 
S$_{N+1}$ equations corresponding to our asymptotic quadrature sets. We then discuss the equations that 
Ganguley, et al., used to generate the quadrature sets that are identical to ours. Finally, we numerically generate 
several of our sets and numerical demonstrate several of their properties. 


%\bc
%{\large \bf II. The Asymptotic P$_N$ Equations}
%\ec
\section{The Asymptotic P$_N$ Equations}

The purpose of this section is to derive Pomraning's asymptotic P$_N$ equations.  We begin with the transport equation:
\be
\mu\dxdy{\psi}{z} + \sigma_t \psi =  \frac{\sigma_s} {2} \phi + \frac{q}{2} \pec
\LEQ{1}
\ee
where $\psi(z,\mu)$ $(p/cm^2-sec-steradian)$ is the angular flux, $\mu$ is the cosine of the polar angle associated with the particle direction, $\sigma_t $ $(cm^{-1})$ is the macroscopic total cross section, $\sigma_s $ $(cm^{-1})$ is the macroscopic scattering 
cross section, $q(z)/2$ $(p/cm^3-sec-steradian)$ is the inhomogeneous source, and $\phi(z)$ $(p/cm^2-sec)$ is the scalar flux:
\be
\phi(z) = \int_{-1}^{+1} \psi(z,\mu) \, d\mu \pep
\LEQ{2}
\ee
It is convenient for our purposes to re-express the spatial variable in \EQ{1} in mean-free-paths.  Let $x$ denote this 
variable, then 
\be
x= \sigma_t z \pec
\LEQ{3}
\ee
and \EQ{1} becomes
\be
\mu\dxdy{\psi}{x} + \psi =   \frac{c}{2} \phi + \frac{\varrho}{2} \pec
\LEQ{4}
\ee
where $c=\sigma_s/\sigma_t$ is the scattering ratio and $\varrho = q/\sigma_t$.
We next take the $k$'th Legendre moment of the transport equation by first multiplying \EQ{1} by the $k$'th Legendre polynomial, 
$P_k(\mu)$, and integrating over all directions.  Using the identity
\be
\mu P_k(\mu) = \frac{k+1}{2k+1} P_{k+1}(\mu) + \frac{k}{2k+1} P_{k-1}(\mu) \pec
\LEQ{4a}
\ee
and taking the moments for $k=0,N$, with $N$ restricted to odd values, we obtain the following system of exact moment equations:
\begin{gather}
\LEQ{5}
\dddx{\Phi_1} + (1-c) \Phi_0 = \varrho \pec \\
\LEQ{6}
\frac{k+1}{2k+1} \dddx{\Phi_{k+1}} + \frac{k}{2k+1} \dddx{\Phi_{k-1}} +  \Phi_k = 0 \pec \quad k=1,N-1, \\
\LEQ{7}
\frac{N+1}{2N+1} \dddx{\Phi_{N+1}} + \frac{N}{2N+1} \dddx{\Phi_{N-1}} + \Phi_N = 0 \pec
\end{gather}
where the $k$'th Legendre moment of the angular flux is given by 
\be
\Phi_k (x) = \int_{-1}^{+1} P_k(x,\mu) \psi(x,\mu) \, d\mu \pec
\LEQ{8}
\ee
and $\sigma_a = \sigma_t-\sigma_s$ $(cm^{-1})$ is the macroscopic absorption cross section.
These moment equations are exact, but the system is open because there are $N+1$ unknowns and 
$N$ equations. The standard P$_N$ approximation is simply to set $\Phi_{N+1}=0$.  Pomraning obtains 
the asymptotic closure by assuming that the solution is given by a linear combination of the Legendre 
polynomials of degree $N-2$ or less and the two asymptotic transport modes. These modes have the following 
form:
\be
\psi^+_{\nu}(x,\mu) = \frac{a^+}{1 + \nu \mu}\exp(\nu x) \pec
\LEQ{10}
\ee
and 
\be
\psi^-_{\nu}(x,\mu) = \frac{a^-}{1 - \nu \mu} \exp(-\nu x) \pec
\LEQ{11}
\ee
where each $a^{\pm}$ is an arbitrary constant, and $\nu$ is the decay constant or the reciprocal of the asymptotic decay length 
that satisfies the following dispersion relation,
\be
\frac{2\nu}{c}=\ln\frac{1+\nu}{1-\nu} \pec
\LEQ{12}
\ee
Because $0 \le c  \le 1$, $\nu$ is real and continuously varies from zero to one as $c$ varies from one to zero. 
Equation~(12) admits negative $\mu$ values as well, but we have accounted for these values by defining $\psi^{\pm}$.
The generalized closure is mechanically expressed in terms of a constant, $\alpha_N$, 
where 
\be
\Phi_{N+1}=\alpha_N \Phi_{N-1} \pep
\LEQ{13}
\ee
Thus \EQ{7} becomes 
\be
\bracket{ \alpha_N \frac{N+1}{2N+1}  +  \frac{N}{2N+1} }\dddx{\Phi_{N-1}} + \Phi_N = 0 \pep
\LEQ{13a}
\ee
Due to the orthogonality of the Legendre polynomials and the assumption that the solution is given by 
a linear combination of the Legendre polynomials of degree $N-2$ or less and the two asymptotic transport modes , 
only the asymptotic modes contribute to $\Phi_{N-1}$ and $\Phi_{N+1}$. Thus 
\be
\alpha_N = \frac{
 \int_{-1}^{+1} P_{N-1}(\mu) \bracket{a^+ \exp(\nu x)(1 + \nu \mu)^{-1} + a^-(x) \exp(-\nu x)(1 + \nu \mu)^{-1}} \, d\mu
}
{
 \int_{-1}^{+1} P_{N+1}(\mu) \bracket{a^+ \exp(\nu x)(1 + \nu \mu)^{-1} + a^- \exp(-\nu x) (1 + \nu \mu)^{-1}} \, d\mu
}
\LEQ{14}
\ee
Since both P$_{N-1}(\mu)$ and P$_{N+1}(\mu)$ are even functions of $\mu$, and $\psi^+_{\nu}(\mu)=\psi^-_{\nu}(-\mu)$, 
\EQ{14} reduces to 
\be
\alpha_N = \frac{\int_{-1}^{+1} P_{N-1}(\mu)(1 + \nu \mu)^{-1} \, d\mu}{\int_{-1}^{+1} P_{N+1}(\mu)(1 + \nu \mu)^{-1} \, d\mu} \pep
\LEQ{15}
\ee 
For a given value of $N$, the constant $\alpha_N$ is purely a function of $c$ and continuously varies from zero to one as $c$ varies from one to zero. 
This constant can be directly evaluated numerically using adaptive quadrature integration for any values of $c$ other than 
zero and one. However, as previously noted, $c=1$ yields $\alpha_N=0$ and $c=0$ yields $\alpha_N=1$.  As $c \rightarrow 1$, the 
modal functions become isotropic and as $c \rightarrow 0$, (with proper normalization) they become delta-functions at 
$\mu=\pm 1$. 

%&\bc
%{\large \bf III. General Equivalence of S$_{N+1}$ and P$_N$ Equations}
%\ec
\section{General Equivalence of S$_{N+1}$ and P$_N$ Equations}
\label{sec:genequiv}

In this section we first show that if any quadrature set meets a certain general criterion, the S$_{N+1}$ equations formed with that set will be equivalent 
to a set of P$_N$ equations with a closure determined by the quadrature.  We begin our consideration of the general S$_{N+1}$ equations by assuming a standard $(N+1)$-point quadrature set for discrete-ordinates 
calculations: $\{\mu_m, w_m\}_{m=1}^{N+1}$, where the weights sum to 2, the quadrature points are symmetric about $\mu=0$, and the 
set exactly integrates at least 1, $\mu$, and $\mu^2$.  The corresponding S$_{N+1}$ equations can be written as follows:
\be
\mu_m\dddx{\psi_m} + \psi_m = \frac{c}{2} \phi + \frac{\varsigma}{2} \pec \quad m=1,N+1,
\LEQ{16}
\ee
where $\psi_m (x)= \psi(x,\mu_m)$ and 
\be
\phi(x) = \sum_{m=1}^{N+1} \psi_m(x) w_m  \pep
\LEQ{17}
\ee
We next take the $k$'th Legendre moment of \EQ{16} using the quadrature formula:
\be
\sum_{m=1}^{N+1} P_k(\mu_m) \bracket{ 
\mu_m\dddx{\psi_m} + \psi_m - \frac{c}{2} \phi + \frac{\varsigma}{2} } w_m = 0 \pep
\LEQ{18}
\ee
Substituting from \EQ{4a} into \EQ{18}, we get
\be
\sum_{m=1}^{N+1} \bracket{ 
\fn{\frac{k+1}{2k+1} P_{k+1}(\mu_m) + \frac{k}{2k+1} P_{k-1}(\mu_m)} \dddx{\psi_m}  + 
\psi_m - \frac{c}{2} \phi + \frac{\varsigma}{2} } w_m = 0 \pep
\LEQ{19}
\ee
Equation~(19) is trivially evaluated for $k=0,N+1$:
\begin{gather}
\LEQ{20}
\dddx{\Phi_1} + (1-c) \Phi_0 = \varsigma \pec \\
\LEQ{21}
\frac{k+1}{2k+1} \dddx{\Phi_{k+1}} + \frac{k}{2k+1} \dddx{\Phi_{k-1}} +  \Phi_k = 0 \pec \quad k=1,N-1, \\
\LEQ{22}
\frac{N+1}{2N+1} \dddx{\Phi_{N+1}} + \frac{N}{2N+1} \dddx{\Phi_{N-1}} + \Phi_N = 0 \pec
\end{gather}
where the $k$'th Legendre moment of the angular flux is given by 
\be
\Phi_k(x) = \sum_{m=1}^{N+1} P_k(\mu_m) \psi_m(x) w_m  \pep
\LEQ{23}
\ee
Note that Eqs.~(\REQ{20}) through (\REQ{22}) are identical to Eqs.~(\REQ{5}) through (\REQ{7}).  However, it can be 
shown that if we assume a certain property of the quadrature formula, Eqs.~(\REQ{20}) through (\REQ{22}) are closed, 
whereas Eqs.~(\REQ{5}) through (\REQ{7}) are open unless a specific closure relating $\Phi_{N+1}$ to the lower-order 
flux moments is defined. The potential implicit closure of Eqs.~(\REQ{20}) through (\REQ{22}) relates to the fact 
that all of the S$_{N+1}$ moments are a function of $N+1$ independent angular flux values. If these angular flux values can 
be uniquely expressed in terms of the first $N+1$ moments, then it follows that $\Phi_{N+1}$ can be expressed in terms of 
these lower-order moments, which in turn closes the system.  We can begin to demonstrate a specific expression for the 
closure by first considering the discrete-to-moment matrix $\mathbf{D}$ defined via \EQ{23} \cite{morel}:
\be
D_{k,m} = P_{k}(\mu_m) w_m \pec \quad \mbox{$k=0,N$, and $m=1,N+1$}.  
\LEQ{24}
\ee
Note that this matrix maps $N+1$ Legendre moments to $N+1$ discrete angular flux values:
\be
\vec{\psi} = \mathbf{D}\vec{\Phi} \pec
\LEQ{25}
\ee
where
\be
\vec{\psi} = \fn{\psi_1,\psi_2, \ldots ,\psi_{N+1}}^t \pec
\LEQ{26}
\ee
and
\be
\vec{\Phi} = \fn{\Phi_0,\Phi_1, \ldots ,\Phi_{N}}^t \pep
\LEQ{27}
\ee
If we assume that $\mathbf{D}$ is invertible, then it follows that 
\be
\psi_m = \sum_{k=0}^{N} \mathbf{M}_{m,k} \Phi_k \pec \quad m=1,N+1.
\LEQ{28}
\ee
where $\mathbf{M}=\mathbf{D}^{-1}$. 
Substituting from \EQ{28} into \EQ{23} for $k=N+1$, we obtain the desired closure expression:
\be
\Phi_{N+1} = \sum_{m=1}^{N+1} P_{N+1}(\mu_m) \bracket{\sum_{k=0}^{N} \mathbf{M}_{m,k} \Phi_k } w_m \pep
\LEQ{29}
\ee
The matrix $\mathbf{M}$ is called the moment-to-discrete matrix \cite{morel}. It is important to realize 
that if the discrete-to-moment matrix is not invertible Eqs.~(\REQ{20}) through (\REQ{22}) are nonetheless 
satisfied by the S$_{N+1}$-generated moments, but these equations cannot be solved for those moments or  
for the discrete angular flux values given those moments. 

%\bc
%{\large \bf IV. Asymptotic P$_{N}$-Equivalent S$_{N+1}$ Equations }
%\ec
\section{Asymptotic P$_{N}$-Equivalent S$_{N+1}$ Equations}


In this section we define a procedure for obtaining quadrature sets that yield Pomraning's closure given 
the value of $\alpha_N$.  In this regard, we want \EQ{13} to be satisfied using the quadrature definition for 
the moments given in \EQ{23}:
\be
\sum_{m=1}^{N+1} P_{N+1}(\mu_m) \psi_m w_m = \alpha \sum_{m=1}^{N+1} P_{N-1}(\mu_m) \psi_m w_m  \pep
\LEQ{30}
\ee
It is useful to rearrange \EQ{30} a bit:
\be
\sum_{m=1}^{N+1} G_N(\mu_m) \psi_m w_m = 0 \pec
\LEQ{31}
\ee
where 
\be
G_N(\mu) = P_{N+1}(\mu) - \alpha_N P_{N-1}(\mu)\pep
\LEQ{32}
\ee
Note from \EQ{31} that if we choose the quadrature points to be the roots of $G_{N}(\mu)$, \EQ{30} is guaranteed 
to be satisfied independent of the values of $\psi$.

We next show that the polynomial $G_N(\mu)$ has $N+1$ real non-zero roots symmetrically 
located about $\mu=0$, and if $\alpha < 1$, these roots are located within the open interval $(-1,1)$, and if $\alpha=1$, 
these roots are located within the closed interval $[-1,1]$.
Using \EQ{4a}, we can rewrite $G_N(\mu)$ as follows:
\begin{gather}
  \label{eq:polynomials:2}
  G_N(\mu) = \frac{2N+1}{n+1} \mu P_N(\mu)
  - \left(\frac{N}{N+1}+\alpha\right) P_{N-1}(\mu).
\end{gather}
We next make the following definitions:
  \begin{gather}
    \label{eq:polynomials:3}
    G_N(\mu) = a(\mu) - b(\mu),
    \qquad a(\mu) = \frac{2N+1}{n+1} \mu P_N(\mu),
    \quad b(\mu) = \left(\frac{N}{N+1}+\alpha\right) P_{N-1}(\mu).
  \end{gather}
  First, nothing needs to be shown for $\alpha = 0$, since $G_N(\mu)$ is equal to 
  $P_{N+1}(\mu)$ in this case, and $P_{N+1}(\mu)$ is known to have the desired properties.
  
  Hence, we consider $\alpha > 0$.  It is known that $P_{N-1}(\mu)$ has one real root 
  within every interval between adjacent roots of $P_N(\mu)$. Let $\xi_k$ and $\xi_{k+1}$ be 
  adjacent roots of $P_{N}(\mu)$.  Then, we have $G_N(\xi_k) = -b(\xi_k)$ 
  and $G_N(\xi_{k+1}) = -b(\xi_{k+1})$. Since $b(\mu)$ changes sign between 
  these two roots, $G_N(\mu)$ must similarly change sign and thereby must 
  have at least one root between them. Thus, $G_N(\mu)$ must have at least one root in {\it every} 
  interval between roots of $P_N(\mu)$. There are $N-1$ such intervals.
  
  Since $N$ is odd, $P_N(\mu)$ has $(N+1)/2$ nonnegative roots,
  and $G_N(\mu)$ has $(N-1)/2$ positive zeros on the
  interval $(0,\xi_N)$, where $\xi_N$ is the largest root of
  $P_N$. Furthermore, $P_{N-1}(\xi_N)$ must be positive.  This follows from 
  the fact that $P_{N-1}(1)=1$, so if $P_{N-1}(\xi_N)$ were negative, it would have a root 
  on $(\xi_N,1)$, which would not be between two roots of $P_{N}(\mu)$. Thus,
  $G_N(\xi_N)< 0$. Furthermore, since
  \begin{gather}
    \label{eq:polynomials:4}
    P_{N-1}(1) = P_{N}(1) = 1 \quad\text{and}\quad
    \frac{2N+1}{N+1} = \frac{N}{N+1} + 1 \ge \frac{N}{N+1}+\alpha,
  \end{gather}
  we find that $G_N(1) \ge 0$, so it has at least one additional root in
  $(\xi_N,1]$.  This root is equal to $1$ if and only if $\alpha = 1$.  Thus, $G_N(\mu)$ has at least
  $(N+1)/2$ positive roots. Since $G_N(\mu)$ is even, the negative of each positive root must also be a root. 
  Thus, $G_N(\mu)$ has at least $N+1$ roots symmetrically arranged about $\mu=0$.  
  Since $G_N(\mu)$ is a polynomial of degree $N+1$, it can have no more than $N+1$ roots.  Hence, we conclude 
  that $G_N(\mu)$ has exactly one root in each interval between roots of $P_N(\mu)$, exactly one root on 
  $(\xi_N,1)$ and $(-1,-\xi_N)$ if $\alpha < 1$, and roots at $1$ and $-1$ if $\alpha=1$.  This concludes the 
  demonstration of the desired properties. 
  

%%% Local Variables: 
%%% mode: latex
%%% TeX-master: "main"
%%% End: 

Once the quadrature points have been obtained by solving for the roots of $G_N(\mu)$, 
the quadrature weights are obtained simply by requiring that 
the quadrature set be exact for all polynomials of degree $N$.  In particular, the equations for the weights 
can be expressed in the following well-conditioned form:
\be
\sum_{m=1}^{N+1} P_k(\mu_m) w_m = 2 \delta_{k,0} \pec \quad k=0,N \pep
\LEQ{33}
\ee

There are two closure cases from which we obtain standard quadrature sets. The first 
corresponds to $\alpha=0$, which yields the standard P$_N$ closure.  In this case, $G(\mu)$ is just $P_{N+1}(\mu)$.  
It is well known that the quadrature points for an $(N+1)$-point Gauss set are the roots of $P_{N+1}(\mu)$. Furthermore, 
it is well-known that the standard P$_N$ equations are equivalent to the S$_{N+1}$ equations with Gauss quadrature. Thus 
our prescription clearly works for this case.  The other case is Pomraning's ``maximum anisotropy'' case which corresponds 
to $\alpha=1$.  Interestingly, we have numerically determined that our prescription yields Lobatto quadrature sets, which 
always include points at $\mu=\pm 1$.  As previously noted, $\alpha=1$ corresponds to $c=0$, which is a purely absorbing medium. 
In this case, the asymptotic value of $\nu$ is one.  The Lobatto set clearly preserves this asymptotic decay rate. 
For instance, let us assume a purely absorbing right half-space problem with the incoming fluxes defined at $x=0$ in the 
positive Lobatto directions by $\{\psi_m=f_m\}_{m=1}^{(N+1)/2}$, where $\mu_1 = 1$. The analytic solution for this problem 
is given by 
\be
\psi(x)_m = f_m \exp(-x/\mu_m) \pec \quad m=1,(N+1)/2 \pep
\LEQ{34}
\ee
Since $\mu_m < \mu_1$ for all $m > 1$, it follows that the flux decays most slowly in direction 1 and thus that the decay rate 
for this mode, which clearly corresponds to $\nu=1$, is the asymptotic rate.  We have numerically confirmed that the discrete-to-moment matrices 
associated with Lobatto quadrature sets are invertible.  Thus our prescription clearly works for this case as well. It is 
interesting to note that the usual definition for the quadrature points of a $(N+1)$-point Lobatto set is that all directions 
other than $\pm 1$ are the roots of $P^{\prime}_{N}(\mu)$.  Our definition states that {\it all} of the Lobatto points including $\mu=\pm 1$ are 
roots of $P_{N+1}(\mu)-P_{N-1}(\mu)$, which is apparently not well-known.



%\bc
%{ \bf IV.a Various Properties of the Sets}
%\ec
\subsection{Various Properties of the Sets}

In this section we demonstrate various properties of the asymptotic quadrature sets.  The first property we demonstrate is that the asymptotic quadrature set with $N+1$ points exactly integrates polynomials through degree $2N-1$. 
Note from \EQ{32} that $G_{N}$ is a polynomial of degree $N+1$.  Thus we can express any polynomial $h(\mu)$ of degree $2N-1$ or less, as follows:
\be
h(\mu)=q(\mu)G_{N}(\mu) + r(\mu)   \pec
\LEQ{35}
\ee
where $q$ is a polynomial of degree $N-2$ and $r$ is a polynomial of degree $N$ or less. All polynomials of degree $N-2$ are orthogonal to $G_N$, so 
\be
\int_{-1}^{+1} h(\mu) \, d\mu = \int_{-1}^{+1} \bracket{q(\mu)G_{N}(\mu) + r(\mu)} \, d\mu = \int_{-1}^{+1} r(\mu) \, d\mu \pep
\LEQ{36}
\ee
Using the asymptotic quadrature formula to evaluate the integral of $h$, and recognizing that the quadrature points are the roots of $G$, we get
\be
\sum_{m=1}^{N+1} h(\mu_m) w_m = \sum_{m=1}^{N+1} \bracket{q(\mu_m)G_{N}(\mu_m) + r(\mu_m)} w_m = \sum_{m=1}^{N+1} r(\mu_m) w_m \pep
\LEQ{37}
\ee
If follows from \EQ{33} that the quadrature exactly integrates $r$.  Thus it follows from \EQ{37} that the quadrature exactly integrates polynomials 
of degree $2N-1$ or less.  If $\alpha_N=0$, $P_{N-1}$ drops out of the expression for $G_N$ and we can raise the degree of $h$ by 2 while maintaining 
the orthogonality of $q$ and $G_N$.  Of course, when $\alpha_N=0$, the asymptotic quadrature set is the standard Gauss set.  Otherwise, all of the 
asymptotic sets have the same polynomial integration accuracy as the Lobatto sets ($\alpha_N=1$).

The second property we demonstrate is that the asymptotic quadrature sets exactly integrate the transport asymptotic modes. 
The analytic dispersion relationship satisfied by the asymptotic decay constant is 
\be
\int_{-1}^{+1} \frac{d\mu}{1 + \nu \mu }  = \frac{2}{c} \pep
\LEQ{38}
\ee
It is easy to show that \EQ{38} is satisfied when $-\nu$ is substituted for $\nu$. Thus we can rewrite \EQ{38} as follows:
\be
\int_{-1}^{+1} \frac{d\mu}{1 \pm \nu \mu }  = \frac{2}{c} \pep
\LEQ{39}
\ee
An analogous discrete expression is satisfied by the S$_{N+1}$ asymptotic decay constant, $\nu_s$: 
\be
\sum_{m=1}^{N+1} \frac{w_m}{1 \pm \nu_{s} \mu_m } = \frac{2}{c} \pec
\LEQ{40}
\ee
Since the asymptotic quadrature sets ensure that $\nu_{s}=\nu$, it follows that 
\be
\sum_{m=1}^{N+1} \frac{w_m}{1 \pm \nu \mu_m } = \frac{2}{c} \pec
\LEQ{41}
\ee
Recognizing that $f^{\pm}_{\nu}(\mu) \equiv (1 \pm \nu \mu_m)^{-1}$ has the angular shape of $\psi^{\pm}_{\nu}(x,\mu)$, 
and comparing Eqs.~(\REQ{39}) and(\REQ {41}), we find that each asymptotic quadrature set exactly integrates its corresponding 
asymptotic modes. 

The third and final property we demonstrate is that the asymptotic quadrature set with $N+1$ points exactly integrates the Legendre 
moments of the asymptotic modes through degree $2N$.  We begin by considering $f^{+}_{\nu}(\mu)$, but all of the results 
we obtain apply if we substitute $-\nu$ for $\nu$.  Thus our results also apply to $f^{-}_{\nu}(\mu)$.
Note that  
\be
(1 + \nu \mu)(1 + \nu \mu)^{-1} = 1 \pep
\LEQ{42}
\ee
We can re-express \EQ{42} as follows:
\be
(P_0(\mu) + \nu P_1(\mu))(1 + \nu \mu)^{-1} = 1 \pep
\LEQ{43}
\ee
If we analytically integrate \EQ{43} over all directions, we get 
\be
\Phi^{+}_0 + \nu \Phi^{+}_1  = 2   \pec 
\LEQ{44}
\ee
where $\Phi^{+}_0$ and $\Phi^{+}_1$ denote the zeroth and first Legendre moments of $f^{+}(\mu)$.  
If we integrate \EQ{43} using the asymptotic quadrature set with $N+1$ points, we get 
\be
\tilde{\Phi}^{+}_0 + \nu \tilde{\Phi}^{+}_1  = 2   \pec 
\LEQ{45}
\ee
where $\tilde{\Phi}$ denotes a quadrature-generated moment.  Note that since the quadrature set is exact for 
polynomials of degree $2N-1$, the right side of \EQ{43} is exactly integrated.
From previous results, we know that $\tilde{\Phi}^{+}_0=\Phi^{+}_0$, so substituting this result into \EQ{45}, 
we get 
\be
\Phi^{+}_0 + \nu \tilde{\Phi}^{+}_1  = 2   \pec 
\LEQ{46}
\ee
Comparing Eqs.~(\REQ{44}) and (\REQ{46}), we find that $\tilde{\Phi}^{+}_1$ and $\Phi^{+}_1$ satisfy the same equation and 
therefore are equal.  

We next multiply \EQ{43} by $\mu$ and substitute from \EQ{5} into the resultant 
equation to obtain 
\be
\bracet{P_1(\mu) + \nu \bracket{\frac{2}{3}P_2(\mu) + \frac{1}{3} P_0(\mu)}}(1 + \nu \mu)^{-1} = \mu  \pec 
\LEQ{47}
\ee
If we analytically integrate \EQ{47} over all directions, we obtain 
\be
\Phi^{+}_1 + \nu \bracket{\frac{2}{3}\Phi^{+}_2 + \frac{1}{3} \Phi^{+}_0} = 0 \pep
\LEQ{48}
\ee
If we integrate \EQ{47} using the asymptotic quadrature set with $N+1$ points, we get
\be
\tilde{\Phi}^{+}_1 + \nu \bracket{\frac{2}{3}\tilde{\Phi}^{+}_2 + \frac{1}{3} \tilde{\Phi}^{+}_0} = 0 \pec
\LEQ{49}
\ee
assuming that $\mu$ is exactly integrated. This will be so if $N$ is greater than 1.
Taking the exactness of $\tilde{\Phi^{+}_0}$ and $\tilde{\Phi^{+}_1}$ into account, \EQ{50} becomes
\be
\Phi^{+}_1 + \nu \bracket{\frac{2}{3}\tilde{\Phi}^{+}_2 + \frac{1}{3} \Phi^{+}_0} = 0 \pep
\LEQ{50}
\ee
Comparing Eqs.~(\REQ{48}) and (\REQ{50}), we find that $\Phi^{+}_2$ and $\tilde{\Phi}^{+}_2$ satisfy the same equation, and thus are equal. 

By generalizing this process, we can make an inductive argument to show that the quadrature set with $N+1$ points exactly integrates the Legendre 
moments of the asymptotic mode through degree $2N$.  In particular, we proceed as follows.
\begin{itemize}  
\item Multiply \EQ{43} by $\mu^{K-1}$, and apply \EQ{4a} to the left side of that equation $K-1$ times, after which it will contain  
$P_{0}(\mu)$ through $P_{K}(\mu)$ with no remaining products of $\mu$ and a Legendre polynomial.  Denote this equation as the base equation. 
\item Analytically integrate the base equation to obtain an equation containing $\Phi^{+}_0$ through $\Phi^{+}_K $. 
\item Integrate the base equation using the asymptotic quadrature set with $N+1$ points to obtain an equation containing $\tilde{\Phi}^{+}_0$ through 
$\tilde{\Phi}^{+}_K$.  
\item If one assumes that the quadrature set is exact for all moments of the asymptotic mode through degree $K-1$, and that the quadrature set 
exactly integrates $\mu^{K-1}$ on the right side of the base equation, one finds that $\tilde{\Phi}^{+}_K$ and $\Phi^{+}_K$ are equal since their respective equations 
are identical.
\end{itemize}
This equivalence will be lost if the quadrature set does not exactly integrate $\mu^{K-1}$.  Since the set with $N+1$ points is exact for polynomials 
of degree $2N-1$, it follows that this set will be exact for evaluating the Legendre moments of the asymptotic modes through degree $2N$.


%\bc
%{ \bf IV.b Boundary Conditions }
%\ec
\subsection{Boundary Conditions}


We next define boundary conditions for our asymptotic S$_{N+1}$ equations.  From our viewpoint there is some flexibility in these 
conditions since they do not affect the preservation of the asymptotic decay length.  Pomraning \cite{pomraning} defines Marshak-type 
boundary conditions for his asymptotic P$_N$ equations, but one needs not necessarily use these conditions to achieve convergence to the 
transport solution as $N \rightarrow \infty$.  We prefer to use traditional Mark-type boundary conditions for our asymptotic 
S$_{N+1}$ equations since such conditions decouple the directions at boundaries for the source (incident radiation) and vacuum conditions. For instance, 
let us assume that the analytic source condition at the left boundary of the system is given by 
\be
\psi(\mu) =  h(\mu) \pec \quad \mu > 0.
\LEQ{55}
\ee
Then the Mark condition is given by 
\be
\psi_m = \gamma h(\mu_m) \pec \quad \mu_m > 0.
\LEQ{56}
\ee
where the normalization constant $\gamma$ is chosen to preserve the exact half-range current:
\be
\gamma \sum_{\mu_m > 0} h(\mu_m) \mu_m w_m =  \int_{-1}^{+1} h(\mu) \mu \, d\mu \pep
\LEQ{57}
\ee
The vacuum condition simply corresponds to $h(\mu) = 0$. The reflective condition requires each 
incoming discrete flux value with cosine $\mu_m$ to equal the outgoing discrete flux value with cosine $-\mu_m$. 
Note that our asymptotic quadrature sets are symmetric about $\mu=0$, so the reflective condition can 
be met.

%\bc
%{ \bf IV.c Interface Conditions}
%\ec
\subsection{Interface Conditions}

We next define interface conditions for our asymptotic S$_{N+1}$ equations.  Pomraning derives interface conditions  
by integrating the asymptotic P$_N$ equations across an interface and taking the limit as the width of the domain of 
integration approaches zero.  This approach yields continuity of the odd moments and discontinuity of the even moments across the 
interface.  However, one need not necessarily use these interface conditions to achieve convergence to the transport solution as 
$N \rightarrow \infty$.  Pomraning's conditions couple all of the discrete angular fluxes at an interface.  We prefer to use 
Marshak-like interface conditions because only angular fluxes having direction cosines of the same sign couple to each other, which makes 
it possible to solve the source-iteration equations via a sweeping approach. Let us consider an interface at $x=x_0$.  We denote a 
quantity to the immediate left of the interface with a subscript $L$ and a quantity to the immediate right of the interface with a 
subscript $R$. Our interface conditions preserve half-range odd moments of the angular flux:
\be
\bracet{\sum_{\mu_{m}>0} \psi_m \mu_m P_{k}(\mu_{m}) w_{m}}_L = 
\bracet{\sum_{\mu_{m}>0} \psi_m \mu_m P_{k}(\mu_{m}) w_{m}}_R \pec \quad k=0,2, \ldots, (N-1),
\LEQ{58}
\ee
\be
\bracet{ \sum_{\mu_{m}<0} \psi_m \mu_m P_{k}(\mu_{m}) w_{m}}_R = 
\bracet{ \sum_{\mu_{m}<0} \psi_m \mu_m P_{k}(\mu_{m}) w_{m}}_L
\pec \quad k=0,2, \ldots, (N-1).
\LEQ{59}
\ee
These conditions ensure that all of the odd moments are continuous across the interface, while the even moments will generally be 
discontinuous across the interface.  These are also properties of Pomraning's conditions.  The above conditions are easily implemented with 
a discontinous spatial discretization.  For the case of a continuous spatial discretization, additional unknowns will be required 
at each interface where the $\alpha_N$-parameter jumps.  

%\bc
%{ \bf IV.d The Equations of Ganguly, et al.}
%\ec
\subsection{The Equations of Ganguly, et al.}

Ganguley, et al., previously generated our asymptotic quadrature sets using a principle based directly 
upon preservation of the transport asymptotic decay length, preservation of the exact leakage for the half-space 
constant source problem, and exact integration of polynomials through degree $2N-1$ for a set with $N+1$ 
points. We have shown that our sets preserve the transport asymptotic decay length and possess the same integration 
accuracy as the sets of Ganguley, et al., but our derivation does not explicitly relate to the preservation of 
the exact leakage for the half-space constant source problem. Thus we presently have no proof that our sets are identical 
to theirs.  We have simply observed it to be so by comparing numerically-generated sets. Ganguley, et al., solved a system 
of constrained nonlinear equations for both the directions and weights \cite{ganguley}.  Our equations 
are much simpler. The directions are obtained by computing the roots of a polynomial, after which the weights are 
obtained by solving a linear system.


%\bc
%{\large \bf V. Computational Results}
%\ec
\section{Computational Results}

In this section we compute various quantities to demonstrate the validity of our formalism.  A study of the accuracy of the these sets for various types 
of transport problems is beyond the scope of this paper.  Ganguley, et al., \cite{ganguley} present results that largely address this point.  All of the data presented in this section was generated with MATLAB using default tolerance and convergence parameters.  We first tabulate the asymptotic S$_4$ quadrature sets for $c=0.25$, $c=0.5$, and $c=0.75$ in \TA{1}.  
\begin{table}[!h]
\begin{center} 
\begin{tabular}{|c|c|c|c|c|c|}
\hline
\hline
\multicolumn{2}{|c|}{$c=0.25$}&\multicolumn{2}{|c|}{$c=0.5$}&\multicolumn{2}{|c|}{$c=0.75$}\\
\hline
 cosine & weight & cosine & weight & cosine & weight\\
 \hline
0.431098  & 0.803396  & 0.399586  &  0.746664 &  0.367456    & 0.693277\\
0.967482  & 0.196604  & 0.919335  &  0.253336 &  0.884062    & 0.306723\\
\hline
\hline
\end{tabular}
\caption{Asymptotic S$_4$ quadrature for various values of $c$.}
\LTA{1}
\end{center}
\end{table}
\afterpage{\clearpage}

The corresponding Lobatto and Gauss sets are tabulated in \TA{2}. 
It can be seen by comparing Tables (\RTA{1}) and (\RTA{2}) that the $c=0.25$ set is closer to the Lobatto set and the $c=0.75$ set is closer to the 
Gauss set, as expected.  Also note that the quadrature dependence upon $c$ is clearly nonlinear as the $c=0.25$ set is much closer to the Lobatto set 
than the $c=0.75$ set is to the Gauss set.  Ganguley, et al., presented a S$_4$ set for ``Case (a)'' and $c=0.5$ in Table I of their paper \cite{ganguley}.
It can be seen by comparison that their set is identical (at least to six digits) with our corresponding asymptotic set.
\begin{table}[!h]
\begin{center} 
\begin{tabular}{|c|c|c|c|}
\hline
\hline
\multicolumn{2}{|c|}{Lobatto} &\multicolumn{2}{|c|}{Gauss}\\
\hline
 cosine & weight & cosine & weight\\
 \hline
0.447214 & 0.833333 & 0.339981 & 0.652145  \\
1.000000 & 0.166667 & 0.861136 & 0.347855  \\
\hline
\hline
\end{tabular}
\caption{S$_4$ Lobatto and Gauss quadrature.}
\LTA{2}
\end{center}
\end{table}
\afterpage{\clearpage}

We have computed the $\mathbf{D}$ and $\mathbf{M}$ matrices, which map the discrete 
fluxes and moments to each other as explained in Section~(\ref{sec:genequiv}).  These two matrices are given below 
for the $c=0.5$ set. 
\be
\mathbf{D} = \bracket{
\begin{tabular}{cccc}
   0.253336 &  0.746663   &  0.746664 &   0.253336 \\
  -0.232901 &  -0.298357  &  0.298357 &   0.232901 \\
   0.194503 &  -0.194503  & -0.194503 &   0.194503 \\
  -0.142755 &   0.328439  & -0.328439 &   0.142755 
\end{tabular} 
} \pec
\LEQ{dmat}
\ee

\be
\mathbf{M} = \bracket{
\begin{tabular}{cccc}
   0.500000  & -1.37900  & 1.91941 &  -1.25270  \\
   0.500000  & -0.599380 &  -0.651240  &  0.977872  \\
   0.500000  &  0.599380 &  -0.651240  & -0.977872  \\
   0.500000  &  1.37900  &  1.91941   & 1.25270  
\end{tabular} 
} \pec
\LEQ{mmat}
\ee
where the vectors corresponding to these matrices are 
\be
\vec{\psi} = \bracket{
\begin{tabular}{c}
$\psi(-0.919335)$ \\
$\psi(-0.399586)$ \\
$\psi(+0.399586)$ \\
$\psi(+0.919335)$
\end{tabular}
} \pec
\LEQ{psivec}
\ee
and 
\be
\vec{\phi} = 
\bracket{
\begin{tabular}{c}
$\phi_0$ \\
$\phi_1$ \\
$\phi_2$ \\
$\phi_3$ 
\end{tabular}
} \pep
\LEQ{phivec}
\ee
The condition number of the $\mathbf{D}$-matrix is 3.28, which indicates a very well-conditioned system. The $\mathbf{M}$-matrix is the inverse of $\mathbf{D}$ with the latter being formed assuming generation of the 
Legendre angular flux moments via quadrature integration. 

The asymptotic decay length associated with the transport equation satisfies \EQ{12} while the asymptotic decay length associated with the S$_{N+1}$ equations satisfies \EQ{40}. 
The exact transport values and the asymptotic S$_4$ values are given in \TA{3} for $c=0.25$, $c=0.5$, and $c=0.75$.
The transport and S$_{N+1}$ values are in complete agreement.  
\begin{table}[!htbp]
\begin{center} 
\begin{tabular}{|c|c|c|c|c|c|}
\hline
\hline
\multicolumn{2}{|c|}{$c=0.25$}&\multicolumn{2}{|c|}{$c=0.5$}&\multicolumn{2}{|c|}{$c=0.75$}\\
\hline
 exact & S$_4$ & exact & S$_4$ & exact & S$_4$ \\
 \hline
0.999326  & 0.999326  & 0.957504  &  0.957504 &  0.775516  & 0.775516 \\
\hline
\hline
\end{tabular}
\caption{Exact and asymptotic S$_4$ quadrature values for the decay constant.}
\LTA{3}
\end{center}
\end{table} 
\afterpage{\clearpage}

We have computationally confirmed the ability of the S$_4$ quadrature sets to exactly integrate polynomials of degree 5 or less.  
For instance, consider the following matrix, $\mathbf{D}_e$, which exactly computes the P$_0$ through P$_3$ Legendre moments of the  
polynomial interpolating the discrete angular flux values defined by the vector $\psi$.    
\be
\mathbf{D} = \bracket{
\begin{tabular}{cccc}
   0.253336 &  0.746663   &  0.746664 &   0.253336 \\
  -0.232901 &  -0.298357  &  0.298357 &   0.232901 \\
   0.194503 &  -0.194503  & -0.194503 &   0.194503 \\
  -0.090673 &   0.208612  & -0.208612 &   0.090673 \\
\end{tabular} 
} \pec
\LEQ{dmate}
\ee
Comparing this matrix 
with the asymptotic $\mathbf{D}$-matrix given in \EQ{dmat}, we find that they differ only in the fourth row.  This implies that 
the S$_4$ quadrature integrates polynomials of through degree 5, but fails to integrate polynomials of degree 6.
Finally, we have computed the Legendre moments through degree 6 of the asymptotic mode, $(1 + \nu\mu)^{-1}$, both exactly (MATLAB) and 
with the S$_4$ quadrature set. These moments are compared in \TA{4}.  It can be seen from this table that the moments are identical 
though P$_6$ but differ at P$_7$.
\begin{table}[!htbp]
\begin{center} 
\begin{tabular}{|c|r|r|}
\hline
\hline
Moment & Exact & S$_4$ \\
\hline
0 &   4.00000 &  4.00000  \\
1 & -2.08876 & -2.08876  \\
2 &  1.27220 &  1.27220  \\
3 &  -0.82193 & -0.82193  \\
4 &   0.54807 &  0.54807  \\
5 &  -0.37276 & -0.37276  \\
6 &   0.25700 &  0.25700  \\
7 &   0.55634 & -0.17896  \\
\hline
\hline
\end{tabular}
\caption{Exact and S$_4$ Legendre moments of the asymptotic mode.}
\LTA{4}
\end{center}
\end{table} 
\afterpage{\clearpage}

Finally, we coded the method of 
Ganguley, et al., for generating the asymptotic quadrature sets and found our method simpler, better conditioned, 
and more efficient.



%\bc
%{\large \bf  VI. Summary and Conclusions}
%\ec
\section{Summary and Conclusions}


We have shown that a general equivalance exists between the S$_{N+1}$ equations and the 
P$_N$ equations with a quadrature-dependent closure if the discrete-to-moment matrix defined in Section~(\ref{sec:genequiv}) is invertible.  In addition, we have identified a family of quadrature sets that yield S$_{N+1}$ equations 
equivalent to Pomraning's asymptotic P$_N$ equations.  This equivalence implies that the corresponding S$_{N+1}$ solutions preserve the exact transport asymptotic decay length.
Ganguley, et al., previously generated these quadrature sets using a formalism based directly 
upon preservation of the asymptotic decay length, preservation of the exact leakage for the half-space 
constant source problem, and exact integration of polynomials. They solved a system 
of constrained nonlinear equations \cite{ganguley}. In contrast, we have shown that the quadrature points 
are roots of a polynomial. Once the quadrature points are known, a linear system can be solved for the 
weights.  Thus our equations for the quadrature sets are simpler and easier to solve than those of 
Ganguley, et al. We have also theoretically demonstrated that the asymptotic sets with $N+1$ points exactly integrate 
polynomials through degree $2N-1$ and exactly integrate polynomial moments of the asymptotic modes through 
degree $2N$.  This integration accuracy for polynomials is consistent with the requirements imposed by 
Ganguley, et al.  The integration accuracy for polynomial moments of the asymptotic modes appears to have 
been previously unknown.


\begin{thebibliography}{99}
\bibitem{pomraning} Pomraning, G. C. (1964). A generalized P$_N$ approximation for neutron transport
problems. \emph{Nukleonik} 6:348.
\bibitem{ganguley} Ganguley, K., Allen, E. J., Coskun, E., and Nielsen, S. (1993). 
On the discrete-ordinates method via Case's solution, \emph{J. Comp. Phys.}, 107:66.
\bibitem{morel} Morel, J. E. (1989). A hybrid collocation-Galerkin-$S_N$  method for solving the
Boltzmann transport equation, {\emph Nucl. Sci. and Eng.}, 101:72.
\bibitem{DLMF}  Digital Library of Mathematical Functions.2011-08-29.National Institute of Standards and Technology from URL http://dlmf.nist.gov/14.7E11.
\bibitem{larsen1} Larsen, E. W., McGhee, J. M., Morel, J. E. (1992). The simplified P$_N$ equations
as an asymptotic limit of the transport equation. \emph{Trans. Am. Nucl. Soc.} 66:231.
\bibitem{larsen2} Larsen, E. W., Morel, J. E., McGhee, J. M. (1996). Asymptotic derivation of the
multigroup P$_1$ and simplified $P_N$ equations with anisotropic scattering. \emph{Nucl. Sci. Eng.}
123:328.
\bibitem{larsen3} Larsen, E. W., Asymptotic diffusion and simplified Pn
approximations for diffusive and deep
penetration problems. part 1: theory. (2011) \emph{TTSP} 39:110.
\end{thebibliography} 
\end{document}