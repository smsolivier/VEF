% \section{Introduction}
The Variable Eddington Factor (VEF) method, also known as Quasi-Diffusion (QD), was one of the first nonlinear methods 
for accelerating source iterations in S$_n$ calculations \cite{AL}.  It is comparable in effectiveness to both linear 
and nonlinear forms of Diffusion-Synthetic Acceleration (DSA), but it offers much more flexibility than the DSA.   
Stability can only be guaranteed with DSA if the diffusion equation is differenced in a manner consistent with that 
of the S$_n$ equations \cite{A}. Modern S$_n$ codes often use advanced discretization schemes such as discontinuous 
Galerkin (DG) since classic discretization schemes such as step and diamond are not suitable for radiative transfer 
calculations in the high-energy density physics regime or coupled electron-photon calculations.  Diffusion discretizations consistent 
with the DG S$_n$ discretizations cannot actually be expressed in diffusion form, but rather must be expressed in 
first-order or P$_1$ form, and are much more difficult to solve than standard diffusion discretizations \cite{WWM}.  Considerable 
effort has gone into the development of ``partially consistent�� diffusion discretizations that yield a stable DSA 
algorithm with some degree of degraded effectiveness, but such discretizations are also generally difficult to develop \cite{ML,AM,WR}. 
A great advantage of the VEF method is that the drift-diffusion equation that accelerates the S$_n$ source iterations can be 
discretized in any valid manner without concern for consistency with the S$_n$ discretization.  When the VEF 
drift-diffusion equation is discretized in a way that is ``non-consistent,�� the S$_n$ and VEF drift-diffusion solutions 
for the scalar flux do not necessarily become identical when the iterative process converges.  However, they do become 
identical in the limit as the spatial mesh is refined, and the difference between the two solutions is proportional to 
the spatial truncation errors associated the S$_n$ and drift-diffusion discretizations.  In general the order accuracy 
of the S$_n$ and VEF drift-diffusion solutions will be the lowest order accuracy of their respective independent 
discretizations.  Although the S$_n$ solution obtained with such a ``non-consistent�� VEF method is not conservative, 
the VEF drift-diffusion solution is in fact conservative.  This is particularly useful in multiphysics calculations 
where the low-order VEF equation can be coupled to the other physics components rather than the high-order S$_n$ 
equations.  Another advantage of the non-consistent approach is that even if the S$_n$ spatial discretization 
scheme does not preserve the thick diffusion limit \cite{LMM}, that limit will generally be preserved using the VEF method. 
 
The purpose of this paper is to investigate the application of the VEF method with the 1-D S$_n$ equations 
discretized with the lumped linear-discontinuous method (LDG) and the drift-diffusion equation discretized using the 
constant-linear mixed finite-element method (MFEM).  To our knowledge, this combination has not been previously 
investigated.  Our motivation for this investigation is that MFEM methods are now being used for high-order hydrodynamics 
calculations \cite{blast}.  A radiation transport method compatible with MFEM 
methods is clearly desirable for developing a MFEM radiation-hydrodynamics code.  Such a code would combine thermal 
radiation transport with hydrodynamics.  However, MFEM methods are inappropriate for the standard first-order form of the 
transport equation.    
Thus the use of the VEF method with a DG S$_n$ discretization and a MFEM drift-diffusion discretization suggests itself.
Here we define a VEF method that should exhibit second-order accuracy since both the transport and drift-diffusion 
discretizations are second-order accurate in isolation.  In addition, our VEF method should preserve the thick diffusion 
limit, which is essential for radiative transfer calculations in the High-Energy Density Laboratory Physics 
(HEDLP) regime. We use the lumped 
rather than the standard LDG discretization because lumping yields a much more robust scheme, and robustness is essential 
for radiative transfer calculations in the HEDLP regime.   Because this is an initial study, we simplify the investigation 
by considering only the one-group neutron transport equation rather than the full radiative transfer equations, which include a 
material temperature equation as well as the radiation transport equation.  The vast majority of relevant properties of  
a VEF method for radiative transfer can be tested with an analogous method for one-group neutron transport.  Furthermore, 
a high-order DG-MFEM VEF method could be of interest for neutronics in addition to radiative transfer calculations. 
A full investigation for radiative transfer calculations will be carried out in a future study. 

The remainder of this paper is organized as follows.  First, we describe the VEF method analytically. Then we describe 
our discretized Sn equations, followed by a description of the discretized VEF drift-diffusion equation.  We next give 
computational results.  More specifically, we describe 
two ways to represent the S$_n$ variable Eddington factor in the MHEM drift-diffusion equation and several ways to 
construct the S$_n$ scattering source from the drift-diffusion solution for the scalar flux. Each of these options 
yields a different VEF method.  The accuracy of these methods is then compared to that of the standard lumped LDG 
S$_n$ solution for several test problems, and the iterative convergence rate of these methods is compared to that of the 
lumped LDG S$_n$ equations with fully-consistent DSA acceleration. Finally, we give conclusions and 
recommendations for future work.

% \begin{thebibliography}{99}
% \bibitem{AL} M.L.  Adams  and  E.W.  Larsen,  ``Fast  Iterative  Methods  for  Discrete-Ordinates  Particle  Transport 
% Calculations,''  {\it Progress in Nuclear Energy}, {\bf 40(1)}, 3--159 (2002).
% \bibitem{A} R.E. Alcouffe, ``Diffusion Synthetic Acceleration Methods for the Diamond-Differenced Discrete-Ordinates 
% Equations,'' {\it Nuclear Science and Engineering}, {\bf 64}, 344--355 (1977).
% \bibitem{WWM} J.S. Warsa, T.A. Wareing, and J.E. Morel, ``Fully-Consistent Diffusion-Synthetic Acceleration 
% of Linear Discontinuous $S_n$ Transport Discretizations on Unstructured Tetrahedral Meshes,'' 
% {\em Nuclear Science and Engineering}, {\bf 141}, 235-251 (2002).
% \bibitem{ML} J.E. Morel and E.W. Larsen, ``A Multiple Balance Approach for Differencing the $S_n$ Equations,'' 
% {\em Nuclear Science and Engineering}, {\bf 105}, 1-15 (1990).
% \bibitem{AM} Marvin L. Adams and William R. Martin, ``Diffusion Synthetic Acceleration of Discontinuous Finite Element 
% Transport Iterations,'' {\it Nuclear Science and Engineering}, {\bf 111}, 145--167 (1992).
% \bibitem{WR} Yaqi Wang, Jean C. Ragusa, ``Diffusion Synthetic Acceleration for High-Order Discontinuous Finite Element S$_n$ 
% Transport Schemes and Application to Locally Refined Unstructured Meshes,'' {\it Nuclear Science and Engineering}, {\bf 166(2)}, 
% 145--166 (2010).
% \bibitem{LMM} Edward W. Larsen, J.E. Morel, Warren F. Miller, Jr., ``Asymptotic Solutions of Numerical Transport 
% Problems in Optically Thick, Diffusive Regimes,'' {\em Journal of Computational Physics}, {\bf 69},  283-324 (1987).
% \bibitem{blast} V. Dobrev, Tz. Kolev and R. Rieben, ``High-Order Curvilinear Finite Element Methods for Lagrangian Hydrodynamics,'' 
% {\it SIAM Journal on Scientific Computing, {\bf 34), B606�-B641 (2012).
% \end{thebibliography}