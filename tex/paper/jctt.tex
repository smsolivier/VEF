\documentclass[12pt]{article}
\usepackage{afterpage}
\usepackage{amsmath}
\usepackage{amsfonts}
\usepackage{amssymb}
\usepackage{amsbsy}
\usepackage{bm}
\usepackage{epsfig}
\usepackage{rotating}
\usepackage{setspace}
\usepackage{tabls}
\usepackage{hhline}
\usepackage{float}
\usepackage{subfigure}
\usepackage{float}
\usepackage{xspace}
%\usepackage{subfigmat}
%\usepackage{citesort}
%\usepackage{cites}
%\usepackage{overcite}
%\usepackage[section]{placeins}

% uncomment for submission of manuscript to NSE
\usepackage[nolists, nomarkers]{endfloat}
%
% use to include postscript figures
\usepackage{graphicx,color}
%
%\usepackage[light,firsttwo]{draftcopy}
%\draftcopySetGrey{0.90}

%\usepackage{dbl}

% -----------------------------------------------------------------------------
% define newcommands
% -----------------------------------------------------------------------------

%\setlength{\floatsep}{4pt plus 1pt minus 1pt}
%\setlength{\textfloatsep}{8pt plus 1pt minus 1pt}
%\setlength{\intextsep}{4pt plus 1pt minus 1pt}
%\setlength{\abovedisplayskip}{4pt plus 1pt minus 1pt}
%\setlength{\belowdisplayskip}{4pt plus 1pt minus 1pt}

\makeatletter
\renewcommand{\@thesubfigure}{\thefigure\thesubfigure\space}
\makeatother
% =================================================================================================
% more new commands
% +++++++++++++++++++++++++++++++++++++++++++++++++++++++++++++++++++++++++++++++++++++++++++++++++
\setlength{\textwidth}{6.5in}
\setlength{\textheight}{8.in}
\setlength{\oddsidemargin}{0in}
%\setlength{\topmargin}{0.75in}
\setlength{\headsep}{12pt}
%\addtolength{\oddsidemargin}{-0.5in}
%\addtolength{\textwidth}{1.0in}
%\addtolength{\textheight}{1.0in}
\renewcommand{\thefootnote}{\fnsymbol{footnote}}
%
% -----------------------------------------------------------------------------
% define newcommands
% -----------------------------------------------------------------------------

%\setlength{\floatsep}{4pt plus 1pt minus 1pt}
%\setlength{\textfloatsep}{8pt plus 1pt minus 1pt}
%\setlength{\intextsep}{4pt plus 1pt minus 1pt}
%\setlength{\abovedisplayskip}{4pt plus 1pt minus 1pt}
%\setlength{\belowdisplayskip}{4pt plus 1pt minus 1pt}

% =================================================================================================
% more new commands
% +++++++++++++++++++++++++++++++++++++++++++++++++++++++++++++++++++++++++++++++++++++++++++++++++

% Ways of grouping things
%
\newcommand{\bracket}[1]{\left[ #1 \right]}
\newcommand{\bracet}[1]{\left\{ #1 \right\}}
\newcommand{\fn}[1]{\left( #1 \right)}
\newcommand{\ave}[1]{\left\langle #1 \right\rangle}
%
% Derivative forms
%
\newcommand{\dx}[1]{\,d#1}
\newcommand{\dxdy}[2]{\frac{\partial #1}{\partial #2}}
\newcommand{\dxdt}[1]{\frac{\partial #1}{\partial t}}
\newcommand{\dxdz}[1]{\frac{\partial #1}{\partial z}}
\newcommand{\dddx}[1]{\frac{d #1}{d x}}
\newcommand{\dfdt}[1]{\frac{\partial}{\partial t} \fn{#1}}
\newcommand{\dfdz}[1]{\frac{\partial}{\partial z} \fn{#1}}
\newcommand{\ddt}[1]{\frac{\partial}{\partial t} #1}
\newcommand{\ddz}[1]{\frac{\partial}{\partial z} #1}
\newcommand{\dd}[2]{\frac{\partial}{\partial #1} #2}
\newcommand{\ddx}[1]{\frac{\partial}{\partial x} #1}
\newcommand{\ddy}[1]{\frac{\partial}{\partial y} #1}
%
% Vector forms
%
%\renewcommand{\vec}[1]{\ensuremath{\stackrel{\rightarrow}{#1}}}
%\renewcommand{\div}{\ensuremath{\vec{\nabla} \cdot}}
%\newcommand{\grad}{\ensuremath{\vec{\nabla}}}
\renewcommand{\vec}[1]{\overrightarrow{#1}}
\renewcommand{\div}{\vec{\nabla}\! \cdot \!}
\newcommand{\grad}{\vec{\nabla}}
\newcommand{\oa}[1]{\fn{\frac{1}{3}\hat{\Omega}\!\cdot\!\overrightarrow{A_{#1}}}}

%
% Equation beginnings and endings
%
\newcommand{\bea}{\begin{eqnarray}}
\newcommand{\eea}{\end{eqnarray}}
\newcommand{\be}{\begin{equation}}
\newcommand{\ee}{\end{equation}}
\newcommand{\beas}{\begin{eqnarray*}}
\newcommand{\eeas}{\end{eqnarray*}}
\newcommand{\bdm}{\begin{displaymath}}
\newcommand{\edm}{\end{displaymath}}


%
% Equation punctuation
%
\newcommand{\pec}{\; ,}
\newcommand{\pep}{\; .}
%
% Equation labels and references, figure references, table references
%
\newcommand{\LEQ}[1]{\label{eq:#1}}
\newcommand{\EQ}[1]{Eq.~(\ref{eq:#1})}
\newcommand{\EQS}[1]{Eqs.~(\ref{eq:#1})}
\newcommand{\REQ}[1]{\ref{eq:#1}}
\newcommand{\LFI}[1]{\label{fi:#1}}
\newcommand{\FI}[1]{Fig.~\ref{fi:#1}}
\newcommand{\RFI}[1]{\ref{fi:#1}}
\newcommand{\LTA}[1]{\label{ta:#1}}
\newcommand{\TA}[1]{Table~\ref{ta:#1}}
\newcommand{\RTA}[1]{\ref{ta:#1}}

%
% List beginnings and endings
%
\newcommand{\bl}{\bss\begin{itemize}}
\newcommand{\el}{\vspace{-.5\baselineskip}\end{itemize}\ess}
\newcommand{\ben}{\bss\begin{enumerate}}
\newcommand{\een}{\vspace{-.5\baselineskip}\end{enumerate}\ess}
%
% Figure and table beginnings and endings
%
\newcommand{\bfg}{\begin{figure}}
\newcommand{\efg}{\end{figure}}
\newcommand{\bt}{\begin{table}}
\newcommand{\et}{\end{table}}
%
% Tabular and center beginnings and endings
%
\newcommand{\bc}{\begin{center}}
\newcommand{\ec}{\end{center}}
\newcommand{\btb}{\begin{center}\begin{tabular}}
\newcommand{\etb}{\end{tabular}\end{center}}
%
% Single space command
%
%\newcommand{\bss}{\begin{singlespace}}
%\newcommand{\ess}{\end{singlespace}}
\newcommand{\bss}{\singlespacing}
\newcommand{\ess}{\doublespacing}
%
%---New environment "arbspace". (modeled after singlespace environment
%                                in Doublespace.sty)
%   The baselinestretch only takes effect at a size change, so do one.
%
\def\arbspace#1{\def\baselinestretch{#1}\@normalsize}
\def\endarbspace{}
\newcommand{\bas}{\begin{arbspace}}
\newcommand{\eas}{\end{arbspace}}
%
% An explanation for a function
%
\newcommand{\explain}[1]{\mbox{\hspace{2em} #1}}
%
% Quick commands for symbols
%
\newcommand{\half}{\frac{1}{2}}
\newcommand{\halff}{1/2}
\newcommand{\third}{\frac{1}{3}}
\newcommand{\twothird}{\frac{2}{3}}
\newcommand{\mdot}{\dot{m}}
\newcommand{\ten}[1]{\times 10^{#1}\,}
\newcommand{\cL}{{\cal L}}
\newcommand{\cD}{{\cal D}}
\newcommand{\cF}{{\cal F}}
\newcommand{\cE}{{\cal E}}
\renewcommand{\Re}{\mbox{Re}}
\newcommand{\Ma}{\mbox{Ma}}
%
% Inclusion of Graphics Data
%
%\input{psfig}
%\psfiginit
%
% More Quick Commands
%
\newcommand{\bi}{\begin{itemize}}
\newcommand{\ei}{\end{itemize}}
\newcommand{\dxi}{\Delta x_i}
\newcommand{\dyj}{\Delta y_j}
\newcommand{\ts}[1]{\textstyle #1}
% Mark URL's
\newcommand{\URL}[1]{{\textcolor{blue}{#1}}}



% Alter some LaTeX defaults for better treatment of figures:
    % See p.105 of "TeX Unbound" for suggested values.
    % See pp. 199-200 of Lamport's "LaTeX" book for details.
    %   General parameters, for ALL pages:
    \renewcommand{\topfraction}{0.9}	% max fraction of floats at top
    \renewcommand{\bottomfraction}{0.8}	% max fraction of floats at bottom
    %   Parameters for TEXT pages (not float pages):
    \setcounter{topnumber}{2}
    \setcounter{bottomnumber}{2}
    \setcounter{totalnumber}{2}     % 2 may work better
    \setcounter{dbltopnumber}{2}    % for 2-column pages
    \renewcommand{\dbltopfraction}{0.9}	% fit big float above 2-col. text
    \renewcommand{\textfraction}{0.07}	% allow minimal text w. figs
    %   Parameters for FLOAT pages (not text pages):
    \renewcommand{\floatpagefraction}{0.7}	% require fuller float pages
	% N.B.: floatpagefraction MUST be less than topfraction !!
    \renewcommand{\dblfloatpagefraction}{0.7}	% require fuller float pages

	% remember to use [htp] or [htpb] for placement

\newcommand{\SN}{S$_n$\xspace}
\renewcommand{\vec}[1]{\bm{#1}} %vector is bold italic
\newcommand{\vd}{\bm{\cdot}} % slightly bold vector dot
\newcommand{\ud}{\mathop{}\!\mathrm{d}} % upright derivative symbol
\newcommand{\pderiv}[2]{\frac{\partial #1}{\partial #2}}
\newcommand{\dderiv}[2]{\frac{\ud #1}{\ud #2}}
\newcommand{\edd}{\langle \mu^2 \rangle} 
	
	
% =================================================================================================
\date{}

\begin{document}

%\bibliographystyle{nse}
%\bibnum{p}

\thispagestyle{empty}
\bc
{\Large \bf Variable Eddington Factor for Mixed Hybrid Finite Element/Linear Discontinuous Galerkin Source Iteration}\\
\vspace{0.5in}
{\large {\bf S. Olivier, J.E. Morel}\\
$ $\\
Department of Nuclear Engineering\\
Texas A\&M University\\
College Station, TX 77843}\\
\ec
$ $\\
\bc
{\large \bf Abstract}\\
\ec
\noindent
\emph{
Abstract goes here
}\\
$ $\\
\noindent
{\bf Keywords}\\
\noindent P$_N$ equations, S$_N$ equations, asymptotic decay lengths.\\
$ $\\
\noindent {\bf Running Head}\\
\noindent Asymptotic S$_{N+1}$ Equations\\
$ $\\
\noindent{\bf Corresponding Author}\\
\noindent Jim E. Morel, Phone: (979)845-6072, FAX: (979)845-6075, E-mail: \emph{morel@tamu.edu}.
$ $\\
\newpage
\ess

\section{Introduction}
%!TEX root = ./jctt.tex
The Variable Eddington Factor (VEF) method, also known as Quasi-Diffusion (QD), was one of the first nonlinear methods 
for accelerating source iterations in S$_n$ calculations \cite{*}.  It is comparable in effectiveness to both linear 
and nonlinear forms of Diffusion-Synthetic Acceleration (DSA), but it offers much more flexibility than the DSA.   
Stability can only be guaranteed with DSA if the diffusion equation is differenced in a manner consistent with that 
of the S$_n$ equations \cite{*}. Modern S$_n$ codes often use advanced discretization schemes such as discontinuous 
Galerkin (DG) since classic discretization schemes such as step and diamond are not suitable for radiative transfer 
calculations in the high-energy density laboratory physics (HEDLP) regime or coupled electron-photon calculations.  Diffusion discretizations consistent 
with the DG S$_n$ discretizations cannot actually be expressed in diffusion form, but rather must be expressed in 
first-order or P$_1$ form, and are much more difficult to solve than standard diffusion discretizations.   Considerable 
effort has gone into the development of ``partially consistent'' diffusion discretizations that yield a stable DSA 
algorithm with some degree of degraded effectiveness, but such discretizations are also generally difficult to develop. A 
great advantage of the VEF method is that the drift-diffusion equation that accelerates the S$_n$ source iterations can be 
discretized in any valid manner without concern for consistency with the S$_n$ discretization.  When the VEF 
drift-diffusion equation is discretized in a way that is ``non-consistent,'' the S$_n$ and VEF drift-diffusion solutions 
for the scalar flux do not necessarily become identical when the iterative process converges.  However, they do become 
identical in the limit as the spatial mesh is refined, and the difference between the two solutions is proportional to 
the spatial truncation errors associated with the S$_n$ and drift-diffusion discretizations.  In general the order of accuracy 
of the S$_n$ and VEF drift-diffusion solutions will be the lowest order accuracy of their respective independent
discretizations.  Although the S$_n$ solution obtained with such a ``non-consistent'' VEF method is not conservative, 
the VEF drift-diffusion solution is in fact conservative.  This is particularly useful in multiphysics calculations 
where the low-order VEF equation can be coupled to the other physics components rather than the high-order S$_n$ 
equations.  Another advantage of the non-consistent approach is that even if the S$_n$ spatial discretization 
scheme does not preserve the thick diffusion limit \cite{*}, that limit will generally be preserved using the VEF method. 
 
The purpose of this paper is to investigate the application of the VEF method with the 1-D S$_n$ equations 
discretized with the lumped linear-discontinuous method (LLDG) and the drift-diffusion equation discretized using the 
constant-linear mixed finite-element method (MFEM).  To our knowledge, this combination has not been previously 
investigated.  Our motivation for this investigation is that MFEM methods are now being used for high-order hydrodynamics 
calculations at Lawrence Livermore National Laboratory \cite{*}.  A radiation transport method compatible with MFEM 
methods is clearly desirable for developing a MFEM radiation-hydrodynamics code.  Such a code would combine thermal 
radiation transport with hydrodynamics.  However, MFEM methods are inappropriate for the first-order form of the 
transport equation, and are problematic even for the even-parity form. \cite{*}.  
Thus the use of the VEF method with a DG S$_n$ discretization and a MFEM drift-diffusion discretization suggests itself.
Here we define a VEF method that should exhibit second-order accuracy since both the transport and drift-diffusion 
discretizations are second-order accurate in isolation.  In addition, our VEF method should preserve the thick diffusion 
limit \cite{*}, which is essential for radiative transfer calculations in the HEDLP regime. We use the lumped 
rather than the standard LDG discretization because lumping yields a much more robust scheme, and robustness is essential 
for radiative transfer calculations in the HEDLP regime.   Because this is an initial study, we simplify the investigation by considering 
only one-group neutron transport rather than the full radiative transfer equations, which include a 
material temperature equation as well as the radiation transport equation.  The vast majority of relevant properties of  
a VEF method for radiative transfer can be tested with an analogous method for one-group neutron transport.  Furthermore, 
a high-order DG-MFEM VEF method could be of interest for neutronics in addition to radiative transfer calculations. 
A full investigation for radiative transfer calculations will be carried out in a future study. 

The remainder of this paper is organized as follows.  First, we describe the VEF method analytically. Then we describe 
our discretized S$_n$ equations, followed by a description of the discretized VEF drift-diffusion equation.  We next give 
computational results.  More specifically, we describe 
two ways to represent the S$_n$ variable Eddington factor in the MFEM drift-diffusion equation and several ways to 
construct the S$_n$ scattering source from the drift-diffusion solution for the scalar flux. Each of these options 
yields a different VEF method.  The accuracy of these methods is then compared to that of the standard lumped LDG 
S$_n$ solution for several test problems, and the iterative convergence rate of these methods is compared to that of the 
lumped LDG S$_n$ equations with fully-consistent S$_2$ Synthetic Acceleration (S$_2$SA). Finally, we give conclusions and 
recommendations for future work.

\section{Variable Eddington Factor Method}
% %!TEX root = ./jctt.tex

\subsection{Source Iteration}
The steady-state, mono-energetic, isotropically-scattering, fixed-source Linear Boltzmann Equation in slab geometry is: 
	\begin{equation} \label{eq:bte}
		\mu \pderiv{\psi}{x}(x, \mu) + \Sigma_t(x) \psi(x,\mu) = 
		\frac{\Sigma_s(x)}{2} \int_{-1}^{1} \psi(x, \mu') d\mu' + \frac{Q(x)}{2} \,,
	\end{equation}
where $\mu = \cos\theta$ is the cosine of the angle of flight $\theta$ relative to the $x$--axis, $\Sigma_t(x)$ and $\Sigma_s(x)$ the total and scattering macroscopic cross sections, $Q(x)$ the isotropic fixed-source and $\psi(x, \mu)$ the angular flux \cite{adams}. Applying the Discrete Ordinates (\SN) angular discretization yields the following system of $N$ coupled, ordinary differential equations: 
	\begin{equation} \label{eq:sn}
		\mu_n \dderiv{\psi_n}{x}(x) + \Sigma_t(x) \psi_n(x) = 
		\frac{\Sigma_s(x)}{2} \phi(x) + \frac{Q(x)}{2} \,, 1 \leq n \leq N \,,
	\end{equation}
where $\psi_n(x) = \psi(x, \mu_n)$ is the discrete angular flux. The scalar flux is computed using an $N$-point quadrature rule where the quadrature weights, $w_n$, sum to two. In other words, 
	\begin{equation} 
		\phi(x) = \sum_{n=1}^N w_n \psi_n(x)
	\end{equation}
The Source Iteration (SI) scheme lags the flux in the scattering term resulting in a system of $N$ independent, first-order, ordinary differential equations: 
	\begin{equation} \label{eq:si}
			\mu_n \dderiv{\psi_n^{\ell+1}}{x}(x) + \Sigma_t(x) \psi_n^{\ell+1}(x) = 
			\frac{\Sigma_s(x)}{2} \phi^{\ell}(x) + \frac{Q(x)}{2} \,, 1 \leq n \leq N \,,
		\end{equation}
where the superscripts indicate the iteration index. 

\subsection{Variable Eddington Factor Acceleration}
The Eddington equations are found by taking the first two angular moments of Eq. \ref{eq:bte}: 
	\begin{subequations} 
	\begin{equation} \label{eq:zero}
		\dderiv{}{x} J(x) + \Sigma_a(x) \phi(x) = Q(x) \,,
	\end{equation} 
	\begin{equation} \label{eq:first}
		\frac{\ud}{\ud x} \edd(x) \phi(x) + \Sigma_t(x) J(x) = 0 \,,
	\end{equation}
	\end{subequations}
where $\phi(x) = \int_{-1}^1 \psi(x,\mu) \ud \mu$ is the scalar flux, $J(x) = \int_{-1}^{1} \mu \psi(x, \mu) \ud \mu$ the current and 
	\begin{equation} \label{eq:eddington} 
		\edd(x) = \frac{\int_{-1}^1 \mu^2 \psi(x, \mu) \ud \mu}{\int_{-1}^1 \psi(x, \mu) \ud \mu}
		% \xrightarrow{\text{S}_N} \frac{\sum_{n=1}^N \mu_n^2 w_n\psi_n(x)}{\sum_{n=1}^N w_n \psi_n(x)} 
	\end{equation}
the Eddington factor. 

This formulation is beneficial because Eq. \ref{eq:zero} is a conservative balance equation and---if $\edd(x)$ is known---the Eddington equations' system of two first-order, ordinary differential equations can be solved directly with well-established methods. However, computing $\edd(x)$ requires knowledge of the angular flux. 

In VEF, \SN is used to compute the Eddington factor needed to solve the Eddington equations. Source iteration is then:  
	\begin{enumerate}
		\item Given the previous estimate for the scalar flux, $\phi^{\ell}(x)$, solve Eq. \ref{eq:si} for $\psi_n^{\ell+1/2}(x)$. 
		\item Compute $\edd^{\ell+1/2}(x)$ with Eq. \ref{eq:edd_sn}. 
		% \item Interpolate $\edd(x)$ onto the MFEM grid 
		\item Solve the Eddington equations for $\phi^{\ell+1}(x)$ using $\edd^{\ell+1/2}(x)$. 
		% \item Use the Eddington equations' $\phi(x)$ on the right hand side of Eq. \ref{eq:si}.  
		\item Update the scalar flux estimate on the right side of Eq. \ref{eq:si} with $\phi^{\ell+1}(x)$ and repeat the iteration process until the scalar flux converges. 
	\end{enumerate}
% This process is one source iteration consisting of an \SN transport step to compute the Eddington factor and an MFEM acceleration step to compute $\phi(x)$. The scalar flux from the acceleration step is used in the right hand side of Eq. \ref{eq:si} and steps 1--4 are repeated until the acceleration step's $\phi(x)$ converges according to Eq. \ref{eq:converg}.  

Acceleration occurs because the angular shape of the angular flux, and thus the Eddington factor, converges much faster than the scalar flux. In addition, the Eddington equations model the contributions of all scattering events at once, reducing the dependence on source iterations to introduce scattering information. The solution from the Eddington equations is then an approximation for the full flux and not the $\ell - 1$ collided flux as it was without acceleration. 

In addition to acceleration, this scheme allows the \SN equations and Eddington equations to be solved with different spatial discretization methods. \SN can be spatially discretized using normal methods, such as Linear Discontinuous Galerkin, while the Eddington equations can be solved with MFEM. 

\subsection{Lumped Linear Discontinuous Galerkin \SN}
The LLDG discretization of Eq. \ref{eq:sn} is:
	\begin{subequations} \label{eq:lldg}
	\begin{equation} 
		\mu_n \left(\psi_{n,i} - \psi_{n, i-1/2}\right) + \frac{\Sigma_{t,i} h_i}{2} \psi_{n,i}^L
		= \frac{\Sigma_{s,i} h_i}{4} \phi_i^L + \frac{h_i}{4} Q_i^L \,, 1 \leq n \leq N \,, 1 \leq i \leq I\,, 
	\end{equation}
	\begin{equation}
		\mu_n \left(\psi_{n,i+1/2} - \psi_{n,i}\right) + \frac{\Sigma_{t,i} h_i}{2} \psi_{n,i}^R
		= \frac{\Sigma_{s,i} h_i}{4} \phi_i^R + \frac{h_i}{4} Q_i^R \,, 1 \leq n \leq N \,, 1 \leq i \leq I\,,
	\end{equation}
	\end{subequations}
where $h_i$, $\Sigma_{t,i}$, and $\Sigma_{s,i}$ are the cell width, total cross section, and scattering cross section in cell $i$. The cell-edged angular fluxes are found through upwinding: 
	\begin{subequations}
	\begin{equation}
		\psi_{n,i-1/2} = \begin{cases}
			\psi_{n,i-1}^R & \mu_n > 0 \\
			\psi_{n,i}^L & \mu_n < 0
		\end{cases}
	\end{equation}
	\begin{equation}
		\psi_{n,i+1/2} = \begin{cases}
			\psi_{n,i}^R & \mu_n > 0 \\
			\psi_{n,i+1}^L & \mu_n < 0
		\end{cases}
	\end{equation}
	\end{subequations}
and the cell-centered angular flux is computed with:
	\begin{equation} \label{eq:psicent}
		\psi_{n,i} = \half \left(\psi_{n,i}^L + \psi_{n,i}^R\right) \,. 
	\end{equation}

In \SN, the Eddington factor is 
	\begin{equation} \label{eq:edd_sn}
		\edd(x) = \frac{\sum_{n=1}^N \mu_n^2 w_n\psi_n(x)}{\sum_{n=1}^N w_n \psi_n(x)} \,.
	\end{equation}

\subsection{Mixed Finite Element Method Eddington}


\subsection{Data Reconstruction}
	Data reconstruction methods are required to recover the linear discontinuous representation in LDG from the piecewise constant with discontinuous jumps at the cell edges representation in MFEM. Data reconstruction methods use neighboring cell centers or edges to extrapolate the MFEM flux to determine the discontinuous cell edge values in LDG that maintain the average and slope. Two data reconstruction methods have been implemented: one that uses the slope between the MFEM edge values and one that only uses cell centered values. 

	The first method sets the LDG discontinuous left and right fluxes in cell $i$, $\phi_{i,L}$ and $\phi_{i,R}$ to 
		\begin{equation}
			\phi_{i,L/R} = \phi_i \mp \frac{1}{2} \left(\phi_{i+1/2} - \phi_{i-1/2}\right)
		\end{equation}

	The second method uses a van Leer slope limiter on the MFEM cell centers. 
% %!TEX root = ./jctt.tex

\newcommand{\rell}{^\ell}
\newcommand{\elll}{^{\ell+1}}

The steady-state, mono-energetic, isotropically-scattering, fixed-source Linear Boltzmann Equation in slab geometry is: 
	\begin{equation} \label{eq:bte}
		\mu \pderiv{\psi}{x}(x, \mu) + \Sigma_t(x) \psi(x,\mu) = 
		\frac{\Sigma_s(x)}{2} \int_{-1}^{1} \psi(x, \mu') d\mu' + \frac{Q(x)}{2} \,,
	\end{equation}
where $\mu = \cos\theta$ is the cosine of the angle of flight $\theta$ relative to the $x$--axis, $\Sigma_t(x)$ and $\Sigma_s(x)$ the total and scattering macroscopic cross sections, $Q(x)$ the isotropic fixed-source and $\psi(x, \mu)$ the angular flux. Applying the Discrete Ordinates angular discretization yields the following set of $N$ coupled, ordinary differential equations: 
	\begin{equation} \label{eq:sn}
		\mu_n \dderiv{\psi_n}{x}(x) + \Sigma_t(x) \psi_n(x) = 
		\frac{\Sigma_s(x)}{2} \phi(x) + \frac{Q(x)}{2} \,, 1 \leq n \leq N \,,
	\end{equation}
where $\psi_n(x) = \psi(x, \mu_n)$ is the angular flux in direction $\mu_n$. The scalar flux, $\phi(x)$, is computed using an $N$-point Gauss quadrature rule such that 
	\begin{equation} \label{eq:phiquad}
		\phi(x) = \sum_{n=1}^N w_n \psi_n(x) \,.
	\end{equation}
The Source Iteration (SI) scheme decouples the system of equations defined by Eq. \ref{eq:sn} by lagging the scattering term. In other words, 
	\begin{equation} \label{eq:si}
		\mu_n \dderiv{\psi_n\elll}{x}(x) + \Sigma_t(x) \psi_n\elll(x) = 
		\frac{\Sigma_s(x)}{2} \phi^\ell(x) + \frac{Q(x)}{2} \,, 1 \leq n \leq N \,,
	\end{equation}
where the superscripts indicate the iteration index. The Lumped Linear Discontinuous Galerkin (LLDG) spatial discretization can now be applied: 
	\begin{subequations} \label{eq:lldg}
	\begin{equation} 
		\mu_n \left(\psi_{n,i}^{\ell+1} - \psi_{n, i-1/2}^{\ell+1}\right) 
		+ \frac{\Sigma_{t,i} h_i}{2} \psi_{n,i,L}^{\ell+1}
		= \frac{\Sigma_{s,i} h_i}{4} \phi_{i,L}^\ell + \frac{h_i}{4} Q_{i,L} \,, 1 \leq n \leq N \,, 1 \leq i \leq I\,, 
	\end{equation}
	\begin{equation}
		\mu_n \left(\psi_{n,i+1/2}^{\ell+1} - \psi_{n,i}^{\ell+1}\right) 
		+ \frac{\Sigma_{t,i} h_i}{2} \psi_{n,i,R}^{\ell+1}
		= \frac{\Sigma_{s,i} h_i}{4} \phi_{i,R}^\ell + \frac{h_i}{4} Q_{i,R} \,, 1 \leq n \leq N \,, 1 \leq i \leq I\,,
	\end{equation}
	\begin{equation}
		\psi_{n,i}^{\ell+1} = \half\left(\psi_{n,i,L}^{\ell+1} + \psi_{n,i,R}^{\ell+1}\right) \,,
	\end{equation}
	\begin{equation}
		\psi_{n,i-1/2}\elll = \begin{cases}
			\psi_{n,i-1,R}\elll \,, & \mu_n > 0 \\ 
			\psi_{n,i,L}\elll \,, & \mu_n < 0 
		\end{cases} \,,
	\end{equation}
	\begin{equation}
		\psi_{n,i+1/2}\elll = \begin{cases}
			\psi_{n,i,R}\elll \,, & \mu_n > 0 \\
			\psi_{n,i+1,L}\elll \,, & \mu_n < 0 
		\end{cases} \,,
	\end{equation}
	\end{subequations}
where $h_i$, $\Sigma_{t,i}$, and $\Sigma_{s,i}$ are the cell width, total cross section and scattering cross section in cell $i$. The $i,L$ and $i,R$ subscripts indicate the the subscripted value is the left and right discontinuous edge value. Equation \ref{eq:lldg} can be rewritten as 
	\begin{equation} \label{eq:sweepLR}
		\left[\begin{matrix}
			\mu_n + \Sigma_{t,i} h_i & \mu_n  \\ 
			\mu_n & \Sigma_{t,i} + \mu_n \\ 
		\end{matrix}\right]
		\left[\begin{matrix}
			\psi_{n,i,L}\elll \\ \psi_{n,i,R}\elll
		\end{matrix}\right]
		= \left[\begin{matrix}
			\frac{\Sigma_{s,i}h_i}{2} \phi_{i,L}\rell + \frac{h_i}{2} Q_{i,L} + 2\mu_n \psi_{n,i-1,R}\elll \\
			\frac{\Sigma_{s,i}h_i}{2} \phi_{i,R}\rell + \frac{h_i}{2} Q_{i,R} 
		\end{matrix}\right] \,, 
	\end{equation}
for sweeping from left to right ($\mu_n > 0$) and 
	\begin{equation} \label{eq:sweepRL}
		\left[\begin{matrix} 
			|\mu_n| + \Sigma_{t,i}h_i & \mu_n \\ 
			|\mu_n| & |\mu_n| + \Sigma_{t,i}h_i \\ 
		\end{matrix} \right]
		\left[\begin{matrix}
			\psi_{n,i,L}\elll \\ \psi_{n,i,R}\elll
		\end{matrix} \right]
		= \left[\begin{matrix}
			\frac{\Sigma_{s,i}h_i}{2} \phi_{i,L}\rell + \frac{h_i}{2} Q_{i,L} \\ 
			\frac{\Sigma_{s,i}h_i}{2} \phi_{i,R}\rell + \frac{h_i}{2} Q_{i,R} - 2\mu_n \psi_{n,i+1,L}\elll
		\end{matrix} \right]
		\,, 
	\end{equation}
for sweeping from right to left ($\mu_n < 0$). The right hand sides of Eqs. \ref{eq:sweepLR} and \ref{eq:sweepRL} are known as the scalar flux from the previous iteration, the fixed source, and the angular flux entering from the previous cell are all known values. 

SI is then: solve Eq. \ref{eq:si} for $\psi_n\elll(x)$, compute the scalar flux with Eq. \ref{eq:phiquad}, update the scattering term on the right hand side of Eq. \ref{eq:si}, and repeat until the scalar flux converges. 
%!TEX root = ./jctt.tex

\newcommand{\rell}{^\ell} % raise to ellth power 
\newcommand{\relll}{^{\ell+1}} % raise to ell + 1 th power 
\newcommand{\rellh}{^{\ell+1/2}} % raise to ell + 1/2 power 

\subsection{The Algorithm}
The steady state, one group, isotropically scattering, slope reconstruction fixed source Linear Boltzmann Equation in 1-D slab geometry is: 
	\begin{equation} \label{eq:bte}
		\mu \pderiv{\psi}{x}(x, \mu) + \sigma_t(x) \psi(x,\mu) = 
		\frac{\sigma_s(x)}{2} \int_{-1}^{1} \psi(x, \mu') d\mu' + \frac{Q(x)}{2} \,,
	\end{equation}
where $\mu = \cos\theta$ is the cosine of the angle of flight $\theta$ relative to the $x$--axis, $\sigma_t(x)$ and $\sigma_s(x)$ the total and scattering macroscopic cross sections, $Q(x)$ the isotropic fixed-source and $\psi(x, \mu)$ the angular flux. Applying the Discrete Ordinates angular discretization yields the following set of $N$ coupled, ordinary differential equations: 
	\begin{equation} \label{eq:sn}
		\mu_n \dderiv{\psi_n}{x}(x) + \sigma_t(x) \psi_n(x) = 
		\frac{\sigma_s(x)}{2} \phi(x) + \frac{Q(x)}{2} \,, 1 \leq n \leq N \,,
	\end{equation}
where $\psi_n(x) = \psi(x, \mu_n)$ is the angular flux in direction $\mu_n$. The scalar flux, $\phi(x)$, is computed using an $N$-point Gauss quadrature rule such that 
	\begin{equation} \label{eq:phiquad}
		\phi(x) = \sum_{n=1}^N w_n \psi_n(x) \,.
	\end{equation}
The Source Iteration (SI) scheme decouples the system of equations defined by Eq. \ref{eq:sn} by lagging the scattering term. In other words, 
	\begin{equation} \label{eq:si}
		\mu_n \dderiv{\psi_n\relll}{x}(x) + \sigma_t(x) \psi_n\relll(x) = 
		\frac{\sigma_s(x)}{2} \phi^\ell(x) + \frac{Q(x)}{2} \,, 1 \leq n \leq N \,,
	\end{equation}
where the superscripts indicate the iteration index. SI is then: solve Eq. \ref{eq:si} for the $\psi_n(x)$, compute the scalar flux using Eq. \ref{eq:phiquad}, update the scalar flux on the right side of Eq. \ref{eq:si}, and repeat until the scalar flux converges. 

The VEF method adds a drift diffusion acceleration step to increase the rate of convergence of SI. The VEF drift diffusion equations are found by taking the first two moments of Eq. \ref{eq:bte}: 
	\begin{subequations} 
	\begin{equation} \label{eq:zero}
		\dderiv{}{x} J(x) + \sigma_a(x) \phi(x) = Q(x) \,,
	\end{equation} 
	\begin{equation} \label{eq:first}
		\frac{\ud}{\ud x} \edd(x) \phi(x) + \sigma_t(x) J(x) = 0 \,,
	\end{equation}
	\end{subequations}
where $J(x) = \int_{-1}^1 \mu \psi_(x, \mu) \ud \mu$ is the current and 
	\begin{equation} \label{eq:eddington} 
		\edd(x) = \frac{\int_{-1}^1 \mu^2 \psi(x, \mu) \ud \mu}{\int_{-1}^1 \psi(x, \mu) \ud \mu}
	\end{equation}
the Eddington factor. By computing the Eddington factor with the \SN angular flux, Eqs. \ref{eq:zero} and \ref{eq:first} can be solved directly for the scalar flux. The drift diffusion scalar flux can then be used to update the scattering term on the right side of Eq. \ref{eq:si}. The VEF method is: 
	\begin{enumerate}
		\item Given the previous estimate for the scalar flux, $\phi^{\ell}(x)$, solve Eq. \ref{eq:si} for $\psi_n^{\ell+1/2}(x)$. 
		\item Compute $\edd^{\ell+1/2}(x)$ with: 
			\begin{equation*}
				\edd\rellh(x) = \frac{\sum_{n=1}^N \mu_n^2 \psi_n\rellh(x)}{\sum_{n=1}^N \psi_n\rellh(x)} \,.
			\end{equation*}
		% \item Interpolate $\edd(x)$ onto the MHFEM grid 
		\item Solve Eqs. \ref{eq:zero} and \ref{eq:first} for $\phi^{\ell+1}(x)$ using $\edd^{\ell+1/2}(x)$. 
		% \item Use the moment equations' $\phi(x)$ on the right hand side of Eq. \ref{eq:si}.  
		\item Update the scalar flux estimate on the right side of Eq. \ref{eq:si} with $\phi^{\ell+1}(x)$ and repeat the iteration process until the scalar flux converges. 
	\end{enumerate}
Acceleration occurs because the angular shape of the angular flux, and thus the Eddington factor, converges much faster than the scalar flux. In addition, the VEF equations model the contributions of all scattering events at once, reducing the dependence on source iterations to introduce scattering information. The solution from the VEF equations is then an approximation for the full flux and not the $\ell - 1$ collided flux as it was without acceleration. 

In addition to acceleration, this scheme allows the \SN equations and drift diffusion equations to be solved with arbitrarily different spatial discretization methods. The following sections  present the application of the Lumped Linear Discontinuous Galerkin (LLDG) spatial discretization to the \SN equations and the Mixed Finite Element Method (MFEM) to the VEF drift diffusion equations. 

\subsection{Lumped Linear Discontinuous Galerkin \SN}
The LLDG discretization of Eq. \ref{eq:si} is: 
	\begin{subequations} 
	\begin{equation} \label{eq:lldg_l}
		\mu_n \left(\psi_{n,i}\rellh - \psi_{n, i-1/2}\rellh\right) 
		+ \frac{\sigma_{t,i} h_i}{2} \psi_{n,i,L}\rellh
		= \frac{\sigma_{s,i} h_i}{4} \phi_{i,L}\rell + \frac{h_i}{4} Q_{i,L} \,, 
		% 1 \leq n \leq N \,, 
		% 1 \leq i \leq I\,, 
	\end{equation}
	\begin{equation} \label{eq:lldg_r}
		\mu_n \left(\psi_{n,i+1/2}\rellh - \psi_{n,i}\rellh\right) 
		+ \frac{\sigma_{t,i} h_i}{2} \psi_{n,i,R}\rellh
		= \frac{\sigma_{s,i} h_i}{4} \phi_{i,R}\rell + \frac{h_i}{4} Q_{i,R} \,, 
		% 1 \leq n \leq N \,, 
		% 1 \leq i \leq I\,,
	\end{equation}
	\end{subequations}
where $h_i$, $\sigma_{t,i}$, and $\sigma_{s,i}$ are the cell width, total cross section and scattering cross section in cell $i$. The $i,L$ and $i,R$ subscripts indicate the the subscripted value is the left or right discontinuous edge value. The cell centered angular flux is the average of the left and right discontinuous edge fluxes:
	\begin{equation} \label{eq:lldg_i}
		\psi_{n,i}\rellh = \half\left(\psi_{n,i,L}\rellh + \psi_{n,i,R}\rellh\right) \,,
	\end{equation}
and the cell edged angular fluxes are defined through upwinding: 
	\begin{subequations}
	\begin{equation} \label{eq:downwind}
		\psi_{n,i-1/2}\rellh = \begin{cases}
			\psi_{n,i-1,R}\rellh \,, & \mu_n > 0 \\ 
			\psi_{n,i,L}\rellh \,, & \mu_n < 0 
		\end{cases} \,,
	\end{equation}
	\begin{equation} \label{eq:upwind}
		\psi_{n,i+1/2}\rellh = \begin{cases}
			\psi_{n,i,R}\rellh \,, & \mu_n > 0 \\
			\psi_{n,i+1,L}\rellh \,, & \mu_n < 0 
		\end{cases} \,.
	\end{equation}
	\end{subequations}
Equations \ref{eq:lldg_l}, \ref{eq:lldg_r}, \ref{eq:lldg_i}, \ref{eq:downwind}, and \ref{eq:upwind} can be combined and rewritten as 
	\begin{equation} \label{eq:sweepLR}
		\left[\begin{matrix}
			\mu_n + \sigma_{t,i} h_i & \mu_n  \\ 
			-\mu_n & \sigma_{t,i} + \mu_n \\ 
		\end{matrix}\right]
		\left[\begin{matrix}
			\psi_{n,i,L}\rellh \\ \psi_{n,i,R}\rellh
		\end{matrix}\right]
		= \left[\begin{matrix}
			\frac{\sigma_{s,i}h_i}{2} \phi_{i,L}\rell + \frac{h_i}{2} Q_{i,L} + 2\mu_n \psi_{n,i-1,R}\rellh \\
			\frac{\sigma_{s,i}h_i}{2} \phi_{i,R}\rell + \frac{h_i}{2} Q_{i,R} 
		\end{matrix}\right] \,, 
	\end{equation}
for sweeping from left to right ($\mu_n > 0$) and 
	\begin{equation} \label{eq:sweepRL}
		\left[\begin{matrix} 
			-\mu_n + \sigma_{t,i}h_i & \mu_n \\ 
			-\mu_n & -\mu_n + \sigma_{t,i}h_i \\ 
		\end{matrix} \right]
		\left[\begin{matrix}
			\psi_{n,i,L}\rellh \\ \psi_{n,i,R}\rellh
		\end{matrix} \right]
		= \left[\begin{matrix}
			\frac{\sigma_{s,i}h_i}{2} \phi_{i,L}\rell + \frac{h_i}{2} Q_{i,L} \\ 
			\frac{\sigma_{s,i}h_i}{2} \phi_{i,R}\rell + \frac{h_i}{2} Q_{i,R} - 2\mu_n \psi_{n,i+1,L}\rellh
		\end{matrix} \right]
		\,, 
	\end{equation}
for sweeping from right to left ($\mu_n < 0$). The right hand sides of Eqs. \ref{eq:sweepLR} and \ref{eq:sweepRL} are known as the scalar flux from the previous iteration, the fixed source, and the angular flux entering from the previous cell are all known. By supplying the flux entering the left side of the first cell, the positive-angled solution can be propagated from left to right by solving Eq. \ref{eq:sweepLR}. Similarly, supplying the incident flux on the right boundary allows the negative-angled solution to be propagated from right to left with Eq. \ref{eq:sweepRL}. 

\subsection{Mixed Finite Element Method VEF Drift Diffusion}
Applying the MFEM to Eqs. \ref{eq:zero} and \ref{eq:first} and enforcing continuity of current yields: 
	\begin{subequations} \label{eq:mfem}
	\begin{equation}
		-\frac{6}{\sigma_{t,i}h_i} \edd_{i-1/2} \phi_{i-1/2}
		+ \left(\frac{12}{\sigma_{t,i}h_i} \edd_i + \sigma_{a,i} h_i\right) \phi_i 
		- \frac{6}{\sigma_{t,i} h_i} \edd_{i+1/2} \phi_{i+1/2} 
		= Q_i h_i \,,
	\end{equation}
	\begin{multline}
		-\frac{2}{\sigma_{t,i} h_i} \edd_{i-1/2}\phi_{i-1/2} + 
		\frac{6}{\sigma_{t,i} h_i} \edd_i \phi_i 
		- 4\left(\frac{1}{\sigma_{t,i} h_i} + \frac{1}{\sigma_{t,i+1} h_{i+1}}\right) 
			\edd_{i+1/2} \phi_{i+1/2}
		\\ + \frac{6}{\sigma_{t,i+1} h_{i+1}} \edd_{i+1} \phi_{i+1} 
		- \frac{2}{\sigma_{t,i+1} h_{i+1}} \edd_{i+3/2} \phi_{i+3/2} 
		= 0 \,,
	\end{multline}
	\end{subequations}
where the Eddington factor is evaluated at iteration $\ell+1/2$ and the scalar flux at $\ell+1$. 
Here, the Eddington factor has been assumed to be constant in each cell with discontinuous jumps at the edges. 
The simplest method of converting the Eddington factor from LLDG to MFEM is to compute the Eddington factor using the cell centered and cell edged angular fluxes using Eqs. \ref{eq:lldg_i}, \ref{eq:downwind}, and \ref{eq:upwind}. 

A more consistent way to transfer the Eddington factor is to represent it as a linear function using the MFEM basis functions:
	\begin{equation} \label{eq:eddquad}
		\edd_i(x) = \frac{
			\sum_{n=1}^N \mu_n^2 \left[\psi_{n,i,L}B_{i,L}(x) + \psi_{n,i,R} B_{i,R}(x)\right]
		}
		{
			B_{i,L}(x) \sum_{n=1}^N w_n \psi_{n,i,L} + B_{i,R}(x) \sum_{n=1}^N w_n \psi_{n,i,R} 
		}
	\end{equation}
where 
	\begin{equation}
		B_{i,L}(x) = \begin{cases}
			\frac{x_{i+1/2} - x}{h_i} \,, & x \in [x_{i-1/2}, x_{i+1/2}] \\ 
			0 \,, & \text{otherwise}
		\end{cases}
	\end{equation}
and 
	\begin{equation}
		B_{i,R}(x) = \begin{cases}
			\frac{x - x_{i-1/2}}{h_i} \,, & x \in [x_{i-1/2}, x_{i+1/2}] \\ 
			0 \,, & \text{otherwise}
		\end{cases} \,.
	\end{equation}
When MFEM is applied, the integral over cell $i$ of Eq. \ref{eq:eddquad} is approximated with 2 point Gauss quadrature. The cell centered Eddington factors used in Eq. \ref{eq:mfem} are then: 
	\begin{equation} 
		\edd_i = \half \left[ \edd_i(x_L) + \edd_i(x_R) \right]
	\end{equation}
where 
	\begin{equation}
		x_{L/R} = \frac{x_{i+1/2} - x_{i-1/2}}{2} \mp \frac{x_{i+1/2} + x_{i-1/2}}{2\sqrt{3}}
	\end{equation}
are the transformed quadrature points. 

Transport consistent boundary conditions are applied through a modified Marshak boundary condition: 
	\begin{equation} 
		J(x) = B(x) \phi(x) 
	\end{equation} 
where 
	\begin{equation} 
		B(x) = \frac{\int_{-1}^1 |\mu| \psi(x, \mu) \ud \mu}
		{\int_{-1}^1 \psi(x, \mu) \ud \mu} \,. 
	\end{equation}

Once the MFEM scalar flux has been found, the LLDG scattering term must be reconstructed. Two methods have been tested: no reconstruction and van Leer cell centered slope reconstruction. The no reconstruction method sets the LLDG discontinuous left and right scalar flux to the MFEM edge scalar flux: 
	\begin{equation} 
		\phi_{i,L/R} = \phi_{i\mp1/2} \,,
	\end{equation} 
where the left hand side is reconstructed LLDG flux used in the scattering term of Eq. \ref{eq:si} and the right hand side the MFEM drift diffusion flux. The van Leer cell centered reconstruction is: 
	\begin{equation} 
		\phi_{i,L/R} = \phi_i \mp \frac{1}{4} \xi_\text{van Leer} \left[\left(\phi_{i+1} - \phi_i\right) + \left(\phi_i - \phi_{i-1}\right) \right] \,,
	\end{equation}
where $\xi_\text{van Leer}$ is the van Leer is the slope limiter given in \cite{}. 

\section{Computational Results}
%!TEX root = ./jctt.tex

The order of accuracy, diffusion limit, and solution convergence were tested in 1D slab geometry with two Eddington factor representations and two scattering term reconstruction methods. The Eddington factor was represented as a piecewise constant with discontinuous jumps at the cell edges and as linear function using the MFEM basis functions. The scattering term reconstruction methods were no reconstruction and maintaining van Leer limited slopes with the MFEM cell centers. The no reconstruction method set the the left and right discontinuous scalar flux to the MFEM edge scalar flux:
	\begin{equation} 
		\phi_{i,L/R} = \phi_{i\mp1/2} \,,
	\end{equation}
where the left hand side is the flux used in the LLDG scattering term and the right hand side the MFEM drift diffusion scalar flux. 
% Maintaining slopes with the MFEM edge values was done according to: 
% 	\begin{equation} \label{eq:edgeReconstruction}
% 		\phi_{i,L/R} = \phi_i \mp \half\left(\phi_{i+1/2} - \phi_{i-1/2}\right) \,.
% 	\end{equation}
Finally, the van Leer cell centered reconstruction is: 
	\begin{equation} \label{eq:vanLeer}
		\phi_{i,L/R} = \phi_i \mp \frac{1}{4} \xi_\text{van Leer} \left[ \left(\phi_{i+1} - \phi_{i}\right) + 
			\left(\phi_{i} - \phi_{i-1}\right)\right] \,
	\end{equation}
where $\xi_\text{van Leer}$ is the van Leer slope limiter given in \cite{}. This reconstruction method is especially important for radiative transfer calculations because the MFEM discretized temperature equation will only have cell centered values. 

\subsection{Order of Accuracy}
The Method of Manufactured Solutions (MMS) was used to compare the accuracy of the VEF method as the cell width was decreased. The L2 norm of the difference between the numerical and MMS solution was compared at five logarithmically spaced cell widths between \SI{0.5}{mm} and \SI{0.01}{mm}. A line of best fit of the form 
	\begin{equation} 
		E = C h^n 
	\end{equation}
was used to find the order of accuracy, $n$, and the constant of proportionality, $C$, of the numerical error, $E$. These values along with the coefficient of correlation are provided in Table \ref{tab:mms} for all six permutations of the two Eddington representation methods and three slope reconstruction methods. All of the permutations are second order accurate and have similar overall accuracy. This suggests that Eddington representation and slope reconstruction do not affect numerical accuracy. It is also a testament to the robustness of the VEF method as the inconsistent, partially consistent, and fully consistent variations all performed similarly. 
	\begin{table}[!h] \centering
	\begin{tabular}{|c|c|c|c|c|}
	\hline
	\hline
	Reconstruction Method & Edd. Representation & Order & $C$ & $R^2$ \\ 
	\hline
		00 & \num{1.997} & \num{0.682} & \num{9.9999e-01} \\
01 & \num{1.998} & \num{0.687} & \num{1.0000e+00} \\
10 & \num{1.997} & \num{0.68} & \num{9.9999e-01} \\
11 & \num{1.998} & \num{0.685} & \num{9.9999e-01} \\
20 & \num{2.007} & \num{0.726} & \num{9.9992e-01} \\
21 & \num{2.009} & \num{0.732} & \num{9.9991e-01} \\

	\hline
	\hline
	\end{tabular}
	\caption{Asymptotic S$_4$ quadrature for various values of $c$.}
	\label{tab:mms}
	\end{table}
	\afterpage{\clearpage}

\subsection{Diffusion Limit}
To test the algorithm in the diffusion limit, the cross sections and fixed source were scaled according to: 
	\begin{subequations} \label{res:scaling}
	\begin{equation} 
		\Sigma_t(x) \rightarrow \Sigma_t(x)/\epsilon\,, 
	\end{equation}
	\begin{equation}
		\Sigma_s(x) \rightarrow \epsilon \Sigma_s(x) \,,
	\end{equation}
	\begin{equation}
		Q(x) \rightarrow \epsilon Q(x)\,. 
	\end{equation}
	\end{subequations}
In the limit as $\epsilon \rightarrow 0$, the system becomes diffusive. Figure \ref{fig:diffLim_iterations} shows the number of iterations needed for the VEF method to converge for all six permutations and Fig. \ref{fig:diffLim_error} shows the L2 norm of the error between the VEF method solution and the exact diffusion solution. These plots show that all of the VEF methods survive the diffusion limit. 

	\begin{figure} \centering
		\includegraphics[width=.75\textwidth]{figs/diffLimit.pdf}
		\caption{The number of iterations for the VEF method to converge in the limit as $\epsilon \rightarrow 0$.}
		\label{fig:diffLim_iterations}
	\end{figure}
	\begin{figure} \centering
		\includegraphics[width=.75\textwidth]{figs/diffError.pdf}
		\caption{A plot of the error between Diffusion Theory and the VEF solution as $\epsilon\rightarrow 0$. }
		\label{fig:diffLim_error}
	\end{figure}

\subsection{Solution Convergence}
The convergence between unaccelerated SI and the VEF method was compared as a function of cell width for a simple homogeneous slab and for a variant of Reed's problem. In both cases, the slab had a reflecting left boundary and vacuum right boundary. The homogeneous slab had a scattering ratio of 0.75. The cross sections and source for Reed's problem are provided in Table \ref{tab:reedXS}. Figure \ref{fig:reed} shows the L2 norm of the difference between unaccelerated and VEF accelerated S$_8$ for the two test problems. 

	\begin{table} \centering
		\begin{tabular}{|c|c|c|c|c|c|}
			\hline
			& Region 1 & Region 2 & Region 3 & Region 4 & Region 5 \\ 
			\hline 
			$q$ & 50 & 0 & 0 & 0 & 1 \\ 
			$\Sigma_t$ & 50 & 0.001 & 1 & 5 & 1 \\ 
			$\Sigma_a$ & 50 & 0 & 0.1 & 0 & 0.1 \\ 
			\hline 
			Domain & $0 \leq x < 2$ & $2 \leq x < 4$ & $4\leq x < 6$ &
				$6 \leq x < 7$ & $7 \leq x \leq 8$\\ 
			\hline 
		\end{tabular}
		\caption{The cross sections and source used for Reed's problem.}
		\label{tab:reedXS}
	\end{table}

	\begin{figure} \centering
		\includegraphics[width=.75\textwidth]{figs/convergence.pdf}
		\caption{The L2 norm of the difference between unaccelerated and VEF S$_8$ for a homogeneous slab and Reed's problem. }
		\label{fig:reed}
	\end{figure}

In both cases, the \SN and VEF solutions converge as the cell width is decreased. The convergence for Reed's problem is three orders of magnitudes worse than for the homogeneous slab case. This suggests that 

\subsection{Comparison to S$_2$SA}



\section{Conclusions and Future Work}


\begin{thebibliography}{99}
\bibitem{pomraning} Pomraning, G. C. (1964). A generalized P$_N$ approximation for neutron transport
problems. \emph{Nukleonik} 6:348.
\bibitem{ganguley} Ganguley, K., Allen, E. J., Coskun, E., and Nielsen, S. (1993). 
On the discrete-ordinates method via Case's solution, \emph{J. Comp. Phys.}, 107:66.
\bibitem{morel} Morel, J. E. (1989). A hybrid collocation-Galerkin-$S_N$  method for solving the
Boltzmann transport equation, {\emph Nucl. Sci. and Eng.}, 101:72.
\bibitem{DLMF}  Digital Library of Mathematical Functions.2011-08-29.National Institute of Standards and Technology from URL http://dlmf.nist.gov/14.7E11.
\bibitem{larsen1} Larsen, E. W., McGhee, J. M., Morel, J. E. (1992). The simplified P$_N$ equations
as an asymptotic limit of the transport equation. \emph{Trans. Am. Nucl. Soc.} 66:231.
\bibitem{larsen2} Larsen, E. W., Morel, J. E., McGhee, J. M. (1996). Asymptotic derivation of the
multigroup P$_1$ and simplified $P_N$ equations with anisotropic scattering. \emph{Nucl. Sci. Eng.}
123:328.
\bibitem{larsen3} Larsen, E. W., Asymptotic diffusion and simplified Pn
approximations for diffusive and deep
penetration problems. part 1: theory. (2011) \emph{TTSP} 39:110.
\end{thebibliography} 
\end{document}