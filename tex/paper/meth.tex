%!TEX root = ./jctt.tex

\newcommand{\rell}{^\ell}
\newcommand{\elll}{^{\ell+1}}

The steady-state, mono-energetic, isotropically-scattering, fixed-source Linear Boltzmann Equation in slab geometry is: 
	\begin{equation} \label{eq:bte}
		\mu \pderiv{\psi}{x}(x, \mu) + \Sigma_t(x) \psi(x,\mu) = 
		\frac{\Sigma_s(x)}{2} \int_{-1}^{1} \psi(x, \mu') d\mu' + \frac{Q(x)}{2} \,,
	\end{equation}
where $\mu = \cos\theta$ is the cosine of the angle of flight $\theta$ relative to the $x$--axis, $\Sigma_t(x)$ and $\Sigma_s(x)$ the total and scattering macroscopic cross sections, $Q(x)$ the isotropic fixed-source and $\psi(x, \mu)$ the angular flux. Applying the Discrete Ordinates angular discretization yields the following set of $N$ coupled, ordinary differential equations: 
	\begin{equation} \label{eq:sn}
		\mu_n \dderiv{\psi_n}{x}(x) + \Sigma_t(x) \psi_n(x) = 
		\frac{\Sigma_s(x)}{2} \phi(x) + \frac{Q(x)}{2} \,, 1 \leq n \leq N \,,
	\end{equation}
where $\psi_n(x) = \psi(x, \mu_n)$ is the angular flux in direction $\mu_n$. The scalar flux, $\phi(x)$, is computed using an $N$-point Gauss quadrature rule such that 
	\begin{equation} \label{eq:phiquad}
		\phi(x) = \sum_{n=1}^N w_n \psi_n(x) \,.
	\end{equation}
The Source Iteration (SI) scheme decouples the system of equations defined by Eq. \ref{eq:sn} by lagging the scattering term. In other words, 
	\begin{equation} \label{eq:si}
		\mu_n \dderiv{\psi_n\elll}{x}(x) + \Sigma_t(x) \psi_n\elll(x) = 
		\frac{\Sigma_s(x)}{2} \phi^\ell(x) + \frac{Q(x)}{2} \,, 1 \leq n \leq N \,,
	\end{equation}
where the superscripts indicate the iteration index. The Lumped Linear Discontinuous Galerkin (LLDG) spatial discretization can now be applied: 
	\begin{subequations} \label{eq:lldg}
	\begin{equation} 
		\mu_n \left(\psi_{n,i}^{\ell+1} - \psi_{n, i-1/2}^{\ell+1}\right) 
		+ \frac{\Sigma_{t,i} h_i}{2} \psi_{n,i,L}^{\ell+1}
		= \frac{\Sigma_{s,i} h_i}{4} \phi_{i,L}^\ell + \frac{h_i}{4} Q_{i,L} \,, 1 \leq n \leq N \,, 1 \leq i \leq I\,, 
	\end{equation}
	\begin{equation}
		\mu_n \left(\psi_{n,i+1/2}^{\ell+1} - \psi_{n,i}^{\ell+1}\right) 
		+ \frac{\Sigma_{t,i} h_i}{2} \psi_{n,i,R}^{\ell+1}
		= \frac{\Sigma_{s,i} h_i}{4} \phi_{i,R}^\ell + \frac{h_i}{4} Q_{i,R} \,, 1 \leq n \leq N \,, 1 \leq i \leq I\,,
	\end{equation}
	\begin{equation}
		\psi_{n,i}^{\ell+1} = \half\left(\psi_{n,i,L}^{\ell+1} + \psi_{n,i,R}^{\ell+1}\right) \,,
	\end{equation}
	\begin{equation} \label{eq:downwind}
		\psi_{n,i-1/2}\elll = \begin{cases}
			\psi_{n,i-1,R}\elll \,, & \mu_n > 0 \\ 
			\psi_{n,i,L}\elll \,, & \mu_n < 0 
		\end{cases} \,,
	\end{equation}
	\begin{equation} \label{eq:upwind}
		\psi_{n,i+1/2}\elll = \begin{cases}
			\psi_{n,i,R}\elll \,, & \mu_n > 0 \\
			\psi_{n,i+1,L}\elll \,, & \mu_n < 0 
		\end{cases} \,,
	\end{equation}
	\end{subequations}
where $h_i$, $\Sigma_{t,i}$, and $\Sigma_{s,i}$ are the cell width, total cross section and scattering cross section in cell $i$. The $i,L$ and $i,R$ subscripts indicate the the subscripted value is the left and right discontinuous edge value. Equation \ref{eq:lldg} can be rewritten as 
	\begin{equation} \label{eq:sweepLR}
		\left[\begin{matrix}
			\mu_n + \Sigma_{t,i} h_i & \mu_n  \\ 
			-\mu_n & \Sigma_{t,i} + \mu_n \\ 
		\end{matrix}\right]
		\left[\begin{matrix}
			\psi_{n,i,L}\elll \\ \psi_{n,i,R}\elll
		\end{matrix}\right]
		= \left[\begin{matrix}
			\frac{\Sigma_{s,i}h_i}{2} \phi_{i,L}\rell + \frac{h_i}{2} Q_{i,L} + 2\mu_n \psi_{n,i-1,R}\elll \\
			\frac{\Sigma_{s,i}h_i}{2} \phi_{i,R}\rell + \frac{h_i}{2} Q_{i,R} 
		\end{matrix}\right] \,, 
	\end{equation}
for sweeping from left to right ($\mu_n > 0$) and 
	\begin{equation} \label{eq:sweepRL}
		\left[\begin{matrix} 
			-\mu_n + \Sigma_{t,i}h_i & \mu_n \\ 
			-\mu_n & -\mu_n + \Sigma_{t,i}h_i \\ 
		\end{matrix} \right]
		\left[\begin{matrix}
			\psi_{n,i,L}\elll \\ \psi_{n,i,R}\elll
		\end{matrix} \right]
		= \left[\begin{matrix}
			\frac{\Sigma_{s,i}h_i}{2} \phi_{i,L}\rell + \frac{h_i}{2} Q_{i,L} \\ 
			\frac{\Sigma_{s,i}h_i}{2} \phi_{i,R}\rell + \frac{h_i}{2} Q_{i,R} - 2\mu_n \psi_{n,i+1,L}\elll
		\end{matrix} \right]
		\,, 
	\end{equation}
for sweeping from right to left ($\mu_n < 0$). The right hand sides of Eqs. \ref{eq:sweepLR} and \ref{eq:sweepRL} are known as the scalar flux from the previous iteration, the fixed source, and the angular flux entering from the previous cell are all known values. 

SI is then: conduct a transport sweep using Eqs. \ref{eq:sweepLR} and \ref{eq:sweepRL} to find $\{\psi_{n,i}\}$, compute $\{\phi_i\}$ using Gauss quadrature, update the scalar flux on the right hand side of Eqs. \ref{eq:sweepLR} and \ref{eq:sweepRL}, and repeat until both $\phi_{i,L}$ and $\phi_{i,R}$ converge. 

The VEF acceleration method alters the above SI scheme by adding a drift diffusion solve. The VEF drift diffusion equations are found by taking the first two moments of Eq. \ref{eq:bte}: 
	\begin{subequations} 
	\begin{equation} \label{eq:zero}
		\dderiv{}{x} J(x) + \Sigma_a(x) \phi(x) = Q(x) \,,
	\end{equation} 
	\begin{equation} \label{eq:first}
		\frac{\ud}{\ud x} \edd(x) \phi(x) + \Sigma_t(x) J(x) = 0 \,,
	\end{equation}
	\end{subequations}
where $J(x) = \int_{-1}^1 \mu \psi_(x, \mu) \ud \mu$ and 
	\begin{equation} \label{eq:eddington} 
		\edd(x) = \frac{\int_{-1}^1 \mu^2 \psi(x, \mu) \ud \mu}{\int_{-1}^1 \psi(x, \mu) \ud \mu}
	\end{equation}
the Eddington factor. In the context of VEF acceleration, the drift diffusion equations are 
	\begin{subequations} 
	\begin{equation} 
		\dderiv{}{x} J\elll(x) + \Sigma_a(x) \phi\elll(x) = Q(x) \,,
	\end{equation} 
	\begin{equation} 
		\frac{\ud}{\ud x} \edd^{\ell+1/2}(x) \phi\elll(x) + \Sigma_t(x) J\elll(x) = 0 \,,
	\end{equation}
	\end{subequations}
where $\edd^{\ell+1/2}(x)$ 

Applying the MFEM and enforcing continuity of current yields: 
	\begin{subequations} \label{eq:mfem}
	\begin{equation}
		-\frac{6}{\Sigma_{t,i}h_i} \edd_{i-1/2} \phi_{i-1/2} 
		+ \left(\frac{12}{\Sigma_{t,i}h_i} \edd_i + \Sigma_{a,i} h_i\right) \phi_i 
		- \frac{6}{\Sigma_{t,i} h_i} \edd_{i+1/2} \phi_{i+1/2} = Q_i h_i 
	\end{equation}
	\begin{multline}
		-\frac{2}{\Sigma_{t,i} h_i} \edd_{i-1/2}\phi_{i-1/2} + 
		\frac{6}{\Sigma_{t,i} h_i} \edd_i \phi_i 
		- 4\left(\frac{1}{\Sigma_{t,i} h_i} + \frac{1}{\Sigma_{t,i+1} h_{i+1}}\right) 
			\edd_{i+1/2} \phi_{i+1/2}
		\\ + \frac{6}{\Sigma_{t,i+1} h_{i+1}} \edd_{i+1} \phi_{i+1} 
		- \frac{2}{\Sigma_{t,i+1} h_{i+1}} \edd_{i+3/2} \phi_{i+3/2} 
		= 0 \,.
	\end{multline}
	\end{subequations}
Here, the Eddington factor has been assumed to be constant in each cell with discontinuous jumps at the edges. A more consistent representation will be presented later. 