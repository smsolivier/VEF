\documentclass[10pt]{beamer}

\usetheme[progressbar=frametitle]{metropolis}
\usepackage{appendixnumberbeamer}

\usepackage{booktabs}
\usepackage[scale=2]{ccicons}

\usepackage{pgfplots}
\usepgfplotslibrary{dateplot}

\usepackage{lmodern}

\usepackage{xspace}
\newcommand{\themename}{\textbf{\textsc{metropolis}}\xspace}

\title{Mixed Hybrid Finite Element Eddington Acceleration of Discrete Ordinates Source Iteration}
\subtitle{\normalsize American Nuclear Society Student Conference Presentation}
% \date{\today}
\date{\today}
\author{Samuel S. Olivier }
\institute{Department of Nuclear Engineering, Texas A\&M University}
% \titlegraphic{\hfill\includegraphics[height=1.5cm]{logo.pdf}}


\newcommand{\SN}{S$_N$\xspace}
\renewcommand{\vec}[1]{\bm{#1}} %vector is bold italic
\newcommand{\vd}{\bm{\cdot}} % slightly bold vector dot
\newcommand{\grad}{\vec{\nabla}} % gradient
\newcommand{\ud}{\mathop{}\!\mathrm{d}} % upright derivative symbol
\newcommand{\pderiv}[2]{\frac{\partial #1}{\partial #2}}
\newcommand{\dderiv}[2]{\frac{\ud #1}{\ud #2}}
\newcommand{\edd}{\langle \mu^2 \rangle} 
\begin{document}

% make blocks fill 
\metroset{block=fill}

\maketitle

\begin{frame}{Table of contents}
  \setbeamertemplate{section in toc}[sections numbered]
  \tableofcontents[hideallsubsections]
\end{frame}

\section{Introduction}

\begin{frame}{Motivation}
    
    Radiation transport simulations are expensive 

    Multiphysics Coupling 
        \begin{itemize}
            \item Operator split requires iteration
            \item Efficient solution methods are often incompatible 
        \end{itemize}

    Radiation Hydrodynamics 
        \begin{itemize}
            \item Mixed Hybrid Finite Element Method for hydro 
            \item Linear Discontinuous Galerkin for transport 
        \end{itemize}

    \begin{block}{Goal}
        Develop an acceleration scheme that 
            \begin{enumerate}
                \item Robustly reduces the number of source iterations in Discrete Ordinates calculations 

                \item Remains compatible with MHFEM multiphysics 
            \end{enumerate}
        Test in 1D slab geometry case
    \end{block}

\end{frame}

\section{Background}

\begin{frame}{Boltzmann Equation}
    Steady-state, mono-energetic, istropically-scattering, fixed-source Linear Boltzmann Equation in 1D slab geometry:
    \begin{equation*}
        \mu \pderiv{\psi}{x}(x, \mu) + \Sigma_t(x) \psi(x,\mu) = 
        \frac{\Sigma_s(x)}{2} \int_{-1}^{1} \psi(x, \mu') d\mu' + \frac{Q(x)}{2}
    \end{equation*}

    $\mu = \cos \theta$ the cosine of the angle of flight $\theta$ relative to the $x$-axis

    $\Sigma_t(x)$, $\Sigma_s(x)$ total and scattering macroscopic cross sections 

    $Q(x)$ the isotropic fixed-source

    $\psi(x,\mu)$ the angular flux 

    Factors of 1/2 come from 
        \begin{equation*}
            \phi(x) = \int_{-1}^{1} \psi(x,\mu) \ud \mu
        \end{equation*}

    \textbf{Integro-differential equation}

\end{frame}

\begin{frame}{Discrete Ordinates Angular Discretization}

    Compute angular flux on $N$ discrete angles 
        \begin{equation*}
            \psi(x,\mu) \xrightarrow{\text{S}_N} 
            \begin{cases}
                \psi_1(x), & \mu = \mu_1 \\ 
                \psi_2(x), & \mu = \mu_2 \\ 
                \vdots \\ 
                \psi_N, & \mu = \mu_N 
            \end{cases}
        \end{equation*}

    $\mu_1$, $\mu_2$, $\dots$, $\mu_N$ defined by N-point Gauss Quadrature Rule 

    \begin{equation*}
        \phi(x) = \int_{-1}^1 \psi(x, \mu) \ud\mu 
            \xrightarrow{\text{S}_N} \sum_{n=1}^N w_n \psi_n(x)
    \end{equation*}

\end{frame}

\begin{frame}{\SN Equations}

    \begin{equation*}
        \mu_n \dderiv{\psi_n}{x}(x) + \Sigma_t(x) \psi_n(x) = 
        \frac{\Sigma_s(x)}{2} \phi(x) + \frac{Q(x)}{2} \,, \, 1 \leq n \leq N
    \end{equation*}

    \begin{equation*}
        \phi(x) = \sum_{n=1}^N w_n \psi_n(x)
    \end{equation*}

    \textbf{$N$ coupled, ordinary differential equations}

\end{frame}

\begin{frame}{Source Iteration}

    Lag scattering term 

    \begin{equation*}
        \mu_n \dderiv{\psi_n^{\ell+1}}{x}(x) + \Sigma_t(x) \psi_n^{\ell+1}(x) = 
        \frac{\Sigma_s(x)}{2} \phi^{\ell}(x) + \frac{Q(x)}{2} \,, 1 \leq n \leq N        
    \end{equation*}

    \begin{exampleblock}{Source Iteration}
    \begin{enumerate}
        \item Given previous estimate for $\phi^{\ell}(x)$, solve for $\psi_n^{\ell+1}$

        \item Compute $\phi^{\ell+1}(x) = 
            \sum_{n=1}^N w_n \psi_n^{\ell+1}(x)$ 

        \item Update scattering term with $\phi^{\ell+1}(x)$ and repeat until: 
             \begin{equation*}
                \frac{\|\phi^{\ell+1}(x) - \phi^{\ell}(x)\|}{\|\phi^{\ell+1}(x)\|} < \epsilon 
             \end{equation*}

    \end{enumerate}
    \end{exampleblock}

    \textbf{$N$ independent, first-order, ordinary differential equations}

\end{frame}

\begin{frame}{Need For Acceleration in Source Iteration}

    If $\phi^0(x) = 0$
    % add arrow through phi_0
    \begin{equation*}
        \mu_n \dderiv{\psi_n^{1}}{x}(x) + \Sigma_t(x) \psi_n^{1}(x) =
        \frac{\Sigma_s(x)}{2} \phi^{0}(x) + \frac{Q(x)}{2} \,, 1 \leq n \leq N 
    \end{equation*}      

    $\Rightarrow \phi^1(x) $ is the uncollided flux 

    Each source iteration adds scattering information 

    $\phi^{\ell}(x)$ is the scalar flux of particles that have undergone at most $\ell - 1$ collisions 

    Number of iterations is linked to the number of collisions in a particle's lifetime 

    \textbf{Slow to converge in optically thick systems with minimal losses to absorption and leakage}

\end{frame}

\begin{frame}{Diffusion Synthetic Acceleration}

    Large, highly scattering systems $\Rightarrow$ Diffusion Theory is accurate! 

    % bold acceleration additions

    \begin{exampleblock}{Diffusion Synthetic Acceleration}
    \begin{enumerate}
        \item Given previous estimate for $\phi^{\ell}(x)$, solve for $\psi_n^{\ell+1/2}$

        \item Compute $\phi^{\ell+1/2}(x) = 
            \sum_{n=1}^N w_n \psi_n^{\ell+1/2}(x)$ 

        \item \textbf{Solve diffusion equation for a correction factor, $f^{\ell+1}(x)$}

        \item Update scattering term with 
            $\bf{\phi^{\ell+1}(x) = \phi^{\ell+1/2}(x) + f^{\ell+1}(x)}$ 
        and repeat until: 
             \begin{equation*}
                \frac{\|\phi^{\ell+1}(x) - \phi^{\ell}(x)\|}{\|\phi^{\ell+1}(x)\|} < \epsilon 
             \end{equation*}

    \end{enumerate}
    \end{exampleblock}

\end{frame}

\begin{frame}{DSA Problems}

    Transport and Diffusion steps must be consistenly differenced to prevent non-convergence 

    Consistently differenced diffusion is much more expensive to solve 

    Transport and MHFEM are not compatible

    \textbf{A new acceleration scheme is needed!}

\end{frame}

\section{Eddigton Acceleration}

\begin{frame}{Zeroth Angular Moment}

    % *** make replace at same spot *** 

    \onslide<1>
    \begin{equation*}
        \mu \pderiv{\psi}{x}(x, \mu) + \Sigma_t(x) \psi(x,\mu) = 
        \frac{\Sigma_s(x)}{2} \phi(x) + \frac{Q(x)}{2}
    \end{equation*}

    \onslide<2>
    \begin{equation*}
        \int_{-1}^{1} \mu \pderiv{\psi}{x}(x, \mu) + 
            \Sigma_t(x) \psi(x,\mu) \ud \mu = 
        \int_{-1}^{1} \frac{\Sigma_s(x)}{2} \phi(x) + \frac{Q(x)}{2} \ud \mu
    \end{equation*}

    \onslide<3>
    \begin{equation*}
        \dderiv{}{x} J(x) + \Sigma_a(x) \phi(x) = Q(x)
    \end{equation*}

\end{frame}

\begin{frame}{First Angular Moment}
    
    \begin{equation*}
        \frac{\ud}{\ud x} \edd(x) \phi(x) + \Sigma_t(x) J(x) = 0 
    \end{equation*}

\end{frame}

\begin{frame}{Eddington Acceleration}

    Use \SN to compute $\edd(x)$ and Moment Equations to find $\phi(x)$ 

    \begin{exampleblock}{Eddington Acceleration}
    \begin{enumerate}
        \item Given the previous estimate for the scalar flux, $\phi^{\ell}(x)$, solve for $\psi_n^{\ell+1/2}(x)$

        \item Compute $\edd^{\ell+1/2}(x)$ 

        \item Solve the moment equations for $\phi^{\ell+1}(x)$ using $\edd^{\ell+1/2}(x)$ 

        \item Update the scalar flux estimate with $\phi^{\ell+1}(x)$ and repeat the iteration process until the scalar flux converges
    \end{enumerate}
    \end{exampleblock}

\end{frame}

\begin{frame}{Eddington Acceleration Properties}
    
    Angular shape of the angular flux converges quickly $\Rightarrow$ Eddington factor quickly converges 

    Solution to moment equations models all scattering events at once 

    Reduces dependence on source iterations to introduce scattering information 

    Benefits 
    \begin{enumerate}
        \item Moment Equations are conservative 

        \item Transport and Acceleration steps can be differenced with arbitrarily different methods 

        \item Accelerates source iterations 

    \end{enumerate}

\end{frame}

\begin{frame}{Metropolis titleformats}
  \themename supports 4 different titleformats:
  \begin{itemize}
    \item Regular
    \item \textsc{Smallcaps}
    \item \textsc{allsmallcaps}
    \item ALLCAPS
  \end{itemize}
  They can either be set at once for every title type or individually.
\end{frame}

{
    \metroset{titleformat frame=smallcaps}
\begin{frame}{Small caps}
  This frame uses the \texttt{smallcaps} titleformat.

  \begin{alertblock}{Potential Problems}
    Be aware, that not every font supports small caps. If for example you typeset your presentation with pdfTeX and the Computer Modern Sans Serif font, every text in smallcaps will be typeset with the Computer Modern Serif font instead.
  \end{alertblock}
\end{frame}
}

{
\metroset{titleformat frame=allsmallcaps}
\begin{frame}{All small caps}
  This frame uses the \texttt{allsmallcaps} titleformat.

  \begin{alertblock}{Potential problems}
    As this titleformat also uses smallcaps you face the same problems as with the \texttt{smallcaps} titleformat. Additionally this format can cause some other problems. Please refer to the documentation if you consider using it.

    As a rule of thumb: Just use it for plaintext-only titles.
  \end{alertblock}
\end{frame}
}

{
\metroset{titleformat frame=allcaps}
\begin{frame}{All caps}
  This frame uses the \texttt{allcaps} titleformat.

  \begin{alertblock}{Potential Problems}
    This titleformat is not as problematic as the \texttt{allsmallcaps} format, but basically suffers from the same deficiencies. So please have a look at the documentation if you want to use it.
  \end{alertblock}
\end{frame}
}

\section{Results}

\begin{frame}[fragile]{Typography}
      \begin{verbatim}The theme provides sensible defaults to
\emph{emphasize} text, \alert{accent} parts
or show \textbf{bold} results.\end{verbatim}

  \begin{center}becomes\end{center}

  The theme provides sensible defaults to \emph{emphasize} text,
  \alert{accent} parts or show \textbf{bold} results.
\end{frame}

\begin{frame}{Font feature test}
  \begin{itemize}
    \item Regular
    \item \textit{Italic}
    \item \textsc{SmallCaps}
    \item \textbf{Bold}
    \item \textbf{\textit{Bold Italic}}
    \item \textbf{\textsc{Bold SmallCaps}}
    \item \texttt{Monospace}
    \item \texttt{\textit{Monospace Italic}}
    \item \texttt{\textbf{Monospace Bold}}
    \item \texttt{\textbf{\textit{Monospace Bold Italic}}}
  \end{itemize}
\end{frame}

\begin{frame}{Lists}
  \begin{columns}[T,onlytextwidth]
    \column{0.33\textwidth}
      Items
      \begin{itemize}
        \item Milk \item Eggs \item Potatos
      \end{itemize}

    \column{0.33\textwidth}
      Enumerations
      \begin{enumerate}
        \item First, \item Second and \item Last.
      \end{enumerate}

    \column{0.33\textwidth}
      Descriptions
      \begin{description}
        \item[PowerPoint] Meeh. \item[Beamer] Yeeeha.
      \end{description}
  \end{columns}
\end{frame}
\begin{frame}{Animation}
  \begin{itemize}[<+- | alert@+>]
    \item \alert<4>{This is\only<4>{ really} important}
    \item Now this
    \item And now this
  \end{itemize}
\end{frame}
\begin{frame}{Figures}
  \begin{figure}
    \newcounter{density}
    \setcounter{density}{20}
    \begin{tikzpicture}
      \def\couleur{alerted text.fg}
      \path[coordinate] (0,0)  coordinate(A)
                  ++( 90:5cm) coordinate(B)
                  ++(0:5cm) coordinate(C)
                  ++(-90:5cm) coordinate(D);
      \draw[fill=\couleur!\thedensity] (A) -- (B) -- (C) --(D) -- cycle;
      \foreach \x in {1,...,40}{%
          \pgfmathsetcounter{density}{\thedensity+20}
          \setcounter{density}{\thedensity}
          \path[coordinate] coordinate(X) at (A){};
          \path[coordinate] (A) -- (B) coordinate[pos=.10](A)
                              -- (C) coordinate[pos=.10](B)
                              -- (D) coordinate[pos=.10](C)
                              -- (X) coordinate[pos=.10](D);
          \draw[fill=\couleur!\thedensity] (A)--(B)--(C)-- (D) -- cycle;
      }
    \end{tikzpicture}
    \caption{Rotated square from
    \href{http://www.texample.net/tikz/examples/rotated-polygons/}{texample.net}.}
  \end{figure}
\end{frame}
\begin{frame}{Tables}
  \begin{table}
    \caption{Largest cities in the world (source: Wikipedia)}
    \begin{tabular}{lr}
      \toprule
      City & Population\\
      \midrule
      Mexico City & 20,116,842\\
      Shanghai & 19,210,000\\
      Peking & 15,796,450\\
      Istanbul & 14,160,467\\
      \bottomrule
    \end{tabular}
  \end{table}
\end{frame}
\begin{frame}{Blocks}
  Three different block environments are pre-defined and may be styled with an
  optional background color.

  \begin{columns}[T,onlytextwidth]
    \column{0.5\textwidth}
      \begin{block}{Default}
        Block content.
      \end{block}

      \begin{alertblock}{Alert}
        Block content.
      \end{alertblock}

      \begin{exampleblock}{Example}
        Block content.
      \end{exampleblock}

    \column{0.5\textwidth}

      \metroset{block=fill}

      \begin{block}{Default}
        Block content.
      \end{block}

      \begin{alertblock}{Alert}
        Block content.
      \end{alertblock}

      \begin{exampleblock}{Example}
        Block content.
      \end{exampleblock}

  \end{columns}
\end{frame}
\begin{frame}{Math}
  \begin{equation*}
    e = \lim_{n\to \infty} \left(1 + \frac{1}{n}\right)^n
  \end{equation*}
\end{frame}
\begin{frame}{Line plots}
  \begin{figure}
    \begin{tikzpicture}
      \begin{axis}[
        mlineplot,
        width=0.9\textwidth,
        height=6cm,
      ]

        \addplot {sin(deg(x))};
        \addplot+[samples=100] {sin(deg(2*x))};

      \end{axis}
    \end{tikzpicture}
  \end{figure}
\end{frame}
\begin{frame}{Bar charts}
  \begin{figure}
    \begin{tikzpicture}
      \begin{axis}[
        mbarplot,
        xlabel={Foo},
        ylabel={Bar},
        width=0.9\textwidth,
        height=6cm,
      ]

      \addplot plot coordinates {(1, 20) (2, 25) (3, 22.4) (4, 12.4)};
      \addplot plot coordinates {(1, 18) (2, 24) (3, 23.5) (4, 13.2)};
      \addplot plot coordinates {(1, 10) (2, 19) (3, 25) (4, 15.2)};

      \legend{lorem, ipsum, dolor}

      \end{axis}
    \end{tikzpicture}
  \end{figure}
\end{frame}
\begin{frame}{Quotes}
  \begin{quote}
    Veni, Vidi, Vici
  \end{quote}
\end{frame}

{%
\setbeamertemplate{frame footer}{My custom footer}
\begin{frame}[fragile]{Frame footer}
    \themename defines a custom beamer template to add a text to the footer. It can be set via
    \begin{verbatim}\setbeamertemplate{frame footer}{My custom footer}\end{verbatim}
\end{frame}
}

\begin{frame}{References}
  Some references to showcase [allowframebreaks] \cite{knuth92,ConcreteMath,Simpson,Er01,greenwade93}
\end{frame}

\section{Conclusion}

\begin{frame}{Summary}

  Get the source of this theme and the demo presentation from

  \begin{center}\url{github.com/matze/mtheme}\end{center}

  The theme \emph{itself} is licensed under a
  \href{http://creativecommons.org/licenses/by-sa/4.0/}{Creative Commons
  Attribution-ShareAlike 4.0 International License}.

  \begin{center}\ccbysa\end{center}

\end{frame}

\begin{frame}[standout]
  Questions?
\end{frame}

\appendix

\begin{frame}[fragile]{Backup slides}
  Sometimes, it is useful to add slides at the end of your presentation to
  refer to during audience questions.

  The best way to do this is to include the \verb|appendixnumberbeamer|
  package in your preamble and call \verb|\appendix| before your backup slides.

  \themename will automatically turn off slide numbering and progress bars for
  slides in the appendix.
\end{frame}

\begin{frame}[allowframebreaks]{References}

  \bibliography{demo}
  \bibliographystyle{abbrv}

\end{frame}

\end{document}
